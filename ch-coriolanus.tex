\chapter{He did it to please his mother\label{ch:Coriolanus}}
If Shakespeare's \emph{Richard II} meditates on the mutability of historical discourse and the ironies inherent to narratives written about the past, \emph{Coriolanus} takes a different route to the intersection of history and tragedy.
In this later play, Shakespeare reframes history as not merely a discourse to be mastered but as a matter of public concern and consumption.
It is the public, specifically, and its relationship to history and historical actors, that animates the action of the play and the political philosophy to which it speaks.
From the ceremonious style of \emph{Richard II}, a play that retains a vestige of tragedy's sacrificial undertone and early modern drama's liturgical ancestry, we move to the blunt frankness of the theater as a public forum, where history is transcribed into the public imagination.
Shakespeare's latest Roman play treats history as in some sense a secular revelation: the disclosing of private affairs to public scrutiny.
There is no history, we learn from \emph{Coriolanus}, without a public and conversely no public without a history to articulate its collective identity.
The Jacobean \emph{Coriolanus} was better placed to engage with this spirit of publicness---and its consequences--than the Elizabethan tetralogy plays. Not only had theatrical interests shifted, but monarchical idolatry had become displaced enough for new social antagonisms and new ways of looking at them to emerge.

In \emph{Richard II}, Shakespeare depicts a king who enacts his private tragedy as a public display.
Formally rendering private experience in a public language, the play collapses the distinction between them.
King Richard is represented as having a solitary introspectiveness that is indistinguishable from his outward persona.
There is, as it were, no separation between his appearance and his substance.
The poetic ornamentation of his speech works to project his inner life onto the outer world of the play, and this play consistently reminds us that all worlds---and all histories---are linguistically constituted and therefore manipulable and negotiable.
While the word-hoarding Bolingbroke is dubiously tactical with his language, Richard makes no pretense of masking his feelings, beliefs, or intentions behind an inscrutable, kingly mien.
He behaves as one who occupies the singularly paradoxical role of king perhaps ought: he is, in the fullest possible sense, autopoietic, evincing the conviction that, like a poet makes verses, his words make the world.
Put another way, he fully inhabits his role whereas Bolingbroke appears cynically detached from his.
Given the versified form of the play \emph{Richard II}, Richard the king could be forgiven for thinking his reality coextensive with his imagination: he is the play's cleverest versifier, even if the astuteness of his wordplay outstrips what his political situation, and nearly what language itself, can bear.
No matter to whom he is speaking, even if to no one (and it is sometimes hard to distinguish), Richard leaves nothing unsaid, and what he says aloud to one or none is as good to him as a proclamation to all.
He behaves as if even his soliloquies have an illocutionary effect, a meditative utterance no different from a royal edict.
Maintaining no corner of hidden thoughts or private subjectivity, Richard thinks he speaks with history's authentic voice, a playwright-historian whose words are his deeds.

The king of twenty thousand names contrasts markedly to Caius Martius Coriolanus,\footnote{For the sake of consistency, I will refer to ``Martius'' when discussing Shakespeare's character specifically and to ``Coriolanus'' when discussing the play and in all other contexts.
Quotations, however, remain unaltered.} a character who at every opportunity disclaims his.
Though usually classified as a tragedy or, vaguely, as a ``political play,'' there is cause for looking at \emph{Coriolanus} through a historical lens, as it reflects through its overt political content the way Shakespeare felt about history as a public commodity that imperfectly captures private lives.
Unlike Richard, who revels in the poetic artifice of kingly self-representation, Martius wants nothing to do with collective processes of meaning-making, and \emph{Coriolanus} theorizes the public differently.
What brings Martius to a crisis in this play is his failure to grasp that a state cannot exist without a public to populate it---without other subjectivities that conflict with and complicate one's own---and a public is comprised of a heterogeneous citizenry.
The heterogenous, however---the aberrant, unknown, unreliable other---disgusts him, is the target toward which his own simultaneously self- and other-denying language is aimed.
If Richard's language is defined by multiplicity and the efflorescence of presence, Martius's is by its singularity and the parsimony of absence.
As we will see, his resistance to public appropriation is inscribed into the play text itself.

% Delete the following section, because the "Pray now no more" quote is repeated later

When Martius is welcomed back to Rome following the conquest of Corioles, he is granted, by both Folio text and grateful citizenry, the eponymous agnomen he would prefer to repudiate (2.1.156ff).\footnote{Although editorial tradition has generally changed Martius's speech prefix after 1.9.66, where he is first addressed as ``Martius Caius Coriolanus,'' the First Folio---also the first printed edition of the play---waits until his return in triumph to Rome.}
His response is characteristic:
\begin{vq}
No more of this, it does offend my heart.\\
Pray now no more.\\
\hfill(2.1.163--164)
\end{vq}
If Richard acts as if he were himself the author of his own text, Martius wants to leap off the page entirely (and actually does at one point).
His reputation that he acts ``as if a man were author of himself'' (5.3.36) is true to a point, but it might be more accurate to say that he wishes to ``unauthor'' himself---to be ‘written' by no one and subject to no one's authorship, including his own.
It is ironic that \emph{Coriolanus} depicts exactly the sort of heroic individual on whose shoulders historiography has traditionally stood, but it is a purposeful irony.
Representing Martius as wanting no part of historical discourse, Shakespeare creates for himself a dramatic/textual conundrum: a character who does not want to be written about and an actor who loathes to be on stage.
These are the necessary conditions, however, for the play to test the stubborn resilience of a different kind of autopoietic individual against the irresistible desire-to-know of the public.
That desire-to-know is a fragmentary force.
It is, in fact, the modus operandi of Renaissance reading: all texts were subject to disintegration, recompilation, and refutation.
Where there is a text, that is, there is interpretation, and once a text is disseminated, interpretations can proliferate far beyond what any author may have intended or subject may have desired.
In \emph{Coriolanus}, Shakespeare depicts this ineluctable subjection of private singularities to public mediation and the impossibility of ever escaping the multiplicity of interpretation: the many-headed reader, who seizes the body of the text and renders it on behalf of the body politic.

Whether this is a frightening image of our inability to protect our reputations or of the necessity of putting public interests ahead of private egos, \emph{Coriolanus} demonstrates the consequences of resisting interpretive appropriation.
What the play suggests is that in order for history to perform its work, the heroic or exemplary individual must be integrated into a greater, more comprehensive narrative, into the story of the people on whose behalf he acts.
This assimilation of private person to public act implies a degree of selection but also, for the person involved, annihilation.
Once the text outlives the person, the text essentially becomes the person.
As an exemplary figure from the classical past himself, Martius likewise only has significance to the (early modern) present insofar as he stands for or against a certain kind of virtue.
His life exists entirely within the bounds of the texts that remain to carry his memory---Plutarch, Livy---and his continued existence is predicated on his absorption into a historical imaginary.
His fate has been for the story of his life to end up serving as an ideological token, to be passed around like currency.
It is precisely to this debasing situation that Shakespeare's Martius objects.
Preferring to keep his achievements out of circulation, he denies their exchange value with moral abstractions.
He appears not to want his victories to mean anything at all beyond their physical enactment, as if allowing them to be accounted for in the historical record were a form of prostitution: eternal glory at the expense of the depreciation of his name.
In every instance in which his deeds are heralded, he is quick to nullify their importance.
At the height of his achievement, he calls them ``nothings'' (2.2.75).
His wounds, the measure by which the Romans judge their heroes, he entirely discounts, calling them
\begin{vq}
Scratches with briars\\
Scars to move laughter only\\
\hfill(3.3.50--51)
\end{vq}
even when they might have saved him from banishment.
This combination of Martius's nation-defining heroism with his reluctance to have that heroism culturally encoded is the dilemma of the play.
On the one hand, Martius is a military hero and savior of the state who both deserves and is obliged to endure official commemoration.
On the other hand, he is directly hostile to any form of public circumscription of his deeds and to the suggestion that he accomplished them on behalf of ``the people.''
By standing firm in contravention to himself, he contravenes against the society that made it possible for him to be who he his
He errs in believing that his virtue is a private possession rather than a public good, that it signifies outside a public context.

Martius, unlike Richard, does not enact a desire to control his own historical publicness.
To the contrary, he wants to escape his public role, escape ``representative publicness,'' entirely.
Richard's abnegation draws the rest of the world on after him. Martius's entails retreat into a private, inarticulate space on which the world, the public, does not impinge.
In the English history plays, we see kings bodily interred into history.
Martius, however, resists such bodily integration into history and also resists its performance: like few actual actors, he does not want to be seen, and he rejects our applause.
Yet he is more ``bodily'' overdetermined than any of Shakespeare's English kings.
He is, in a way, all body.
His body even becomes a grisly stylus when as a ``thing of blood'' he is raised like a sword before the gates of Corioles.
After his victory, it bears a chronicle of wounds that everyone except him wants to inscribe into the public record.
But he does not want his wounds to be common property, to be symbolically transferred from his body to the body of the state.
Rejecting the metaphor of the body politic trumpeted by Menenius, he prefers to restrict the legibility of his body and its wounds to the private domain dominated by his mother.
As if the writings of a secret language shared only with her, he resists the appropriation of these signifying lacerations by the artificial, poetic body of the historical person publicly dubbed ``Coriolanus.''
What he finds so repellent about the people is that they might have---through the voice granted to them, through their participation in the power of language---power over the language that constitutes him.
He fears, that is, an incursion into his language world, the sort of incursion he cannot fight off with weapons.
Leonard Tennenhouse emphasizes the importance of this linguistic aspect of the play's dramatic set-up.
The beginning of the play is really a new beginning for Rome: the founding of the Republic and the enfranchisement of the plebeians.
The state in these circumstances is newly defined and so, according to Tennenhouse, ``the actions which constitute public service, and the means by which power is exercised must also be newly defined.''
This new, more public faculty of language, a faculty, I would add, that also newly opens onto history, is what Martius distrusts.
He is equally skeptical about giving the people a voice and allowing that voice to shape the future discourse of the city and himself.
Unable to control this language, he fears the dissemination of his actions into speech that will grow beyond his capacity to control it:
``His abhorrence of public speech and his distrust of words are functions of his obsessive quest for a personal integrity which can only be concretely realized in physical action.
Language to him is a private faculty which at best serves to speak his anger''~\cite[223]{tennenhouse_coriolanus:_1986}.

Shakespeare's dramatic historiography overall is concerned less with the recording of deeds than their assimilation into a language world of shared meanings.
The ``deeds'' for which Martius is celebrated have no meaning outside the Roman public that sanctions them, delineates their normative value, and rewards them with the name that fixes his achievement within time, place, and language.
To give them a public significance, however, is also to make their significance contestable.
Thinking them mute facts that speak to no one, Martius at first refuses their reduction to language.
He is loathe to make any part of himself available to the divergent opinions of the people.
If the people are able to ``read'' his wounds, he believes, his wounds will become their possession.
Of course, this is exactly what they want: it is their prerogative as the witnesses of state history to make his private deeds public.
There is no choice, either, whether to enter into the public memory, because regardless of Martius's disdain for the rites of civic life, ``there is no getting away from being named by others, and thus no getting away from engagement with their laws.
And therefore, there is no getting out of being human, if that means to be zoon politikon''~\cite[124]{pfannebecker_cyborg_2012}.
Participating in politics, in the life of the city, is what Martius is supposed to want, however, and it is what his patrician circle prepares him for.
Instead, horrified by the idea of such exposure, he evades public ownership of his body and the incorporation into a public history it would entail.

A rejector both of the people and the publicness of history, Martius will act in warfare as an instrument himself, but he cannot bear to be used as an instrument by others.
He is as withholding of his public identity, provender to the Roman political dispensation, as he would prefer his state to be withholding of grain to feed the masses.
What the play alleges, however, is that he does not have a choice: the irresistible commons will feed on him and grain, both.
The consequences of his obstinacy are twofold: for the state, in that the founding hero of a society is irreconcilable to living in one; and for the founding hero himself, whose own history does not belong to him.
The sort of heroism exhibited by Martius and celebrated by Rome is at once superhuman and inhuman, an exceptional nation-building force that can only operate when unbound by restrictions that make the normal functioning of nations possible.
His vision for Rome, that it be constantly at war so as to forestall a reckoning with plebeian demands, is exemplary of Giorgio Agamben's ``state of exception,'' the contradictory circumstance in which a nation's most defining laws and values can only be guaranteed by means of their permanent suspension.
The temporary state of emergency creeps by degrees to become a permanent state of exception, when the nation discovers that it defines itself in the negative, as one always under an existential, and therefore exceptional, threat.\footnote{See Giorgio Agamben. \emph{State of Exception}. Trans. Kevin Aell. Chicago: University of Chicago Press, 2005. Print.\nocite{agamben_2005}}
That Martius would prefer such a state is indicative of his distaste for politics and routine governance.
His Rome would be frozen in time, without a public to articulate its identity or constitute its memory---a Rome, therefore, without a history.
But Shakespeare's play also intimates that history does not entirely belong to the great men who animate it.
The people, whom Martius both fights for and despises, ultimately come to possess the history he creates in spite of him.
They lack for recognition individually, but as ``the people,'' they are more important than any one person, including him.
He belongs to history, to a specific and finite time.
Just as his body and his wounds are totems of a temporally-bound present, the body of ``the people,'' history's ultimate inheritor, extends beyond time, into a future that Martius cannot depopulate.
And this is one source of his anger.
He coyly attempts to keep his wounds hidden, to prevent the people from speaking their history through them, but he cannot escape the grasp of history so easily.
Martius's place at the beginning of Roman history is predetermined.
The irony of his own dramatic situation is absorbed into the way Shakespeare characterizes him.
He cannot opt-out of his historical function any more than he can keep his wounds hidden and his body inviolate.

Martius is a unsympathetic antagonist for whom public display is anathema, but he is also an exemplary figure from the most venerated historiographical tradition of Shakespeare's time.
Shakespeare characterizes in \emph{Coriolanus} a celebrated historical actor, but the play also explores what it means for such an actor to be subsumed into history; it investigates the historical process, partly in the guise of a political process, and puts to the question who makes history and who gets to decide how history is made.
Such an exercise in historical reconstruction and contemporary appropriation aligns the play with the larger humanist project of the Renaissance, to re-conceive the boundaries of what is knowable in the present through a confrontation with knowledge recovered from the past.
As Clifford Ronan points out, there is precedent for the use of drama in historical explication by the Romans themselves in the \emph{fabulae praetextae}, or political stories.
Ronan argues that ``to write a Roman history play must have been a central humanist endeavor: a re-representing of the genre in which Rome itself re-presents its history''~\cite[7]{ronan_antike_1995}.
What Shakespeare accomplishes with \emph{Coriolanus}, in particular, is the importation of this Roman idiom into English times and the English tongue. It is at once a nostalgic re-creation of what England imagined to be its cultural and imperial antecedent and a challenge to its historical sovereignty.
And despite its appearance in an era of dramatic production that did not universally place a high value on historical accuracy for its own sake, \emph{Coriolanus} is nevertheless an impressive attempt at verisimilitude.
As put by Terence Spencer, noting the preciseness of detail with which Shakespeare depicts Rome in this play, ``dozens of poetasters could write plays on Julius Caesar or on Cleopatra.
Dozens did.
But to write \emph{Coriolanus} was one of the great feats of the historical imagination in Renaissance Europe''~\cite[35]{spencer_shakespeare_1948}.
This is a remarkable endorsement for a play that is not typically regarded with very much fondness.
By making Roman actors speak with English voices, the play proposes an extension of the English historical horizon even as far back as the \emph{terminus a quo} that Rome in the time of Caius Martius Coriolanus represents.
This is in keeping with the common perception among the early modern English that Roman history was one of the principal founts of their own---a direct line of descent passing back to Troy through the blood of Aeneas
It was widely believed that Britain was Rome's, and ultimately Greece's, true cultural inheritor (London was ``Troynovant'' just as Rome was a new Troy).
Shakespeare's dramatic art exploits this affective dimension to history by using the strangeness of the past to illuminate the dark corners of the present.
But he does not necessarily do this as a cover for some sort of commentary on his own time, as Bart Westerweel argues, but uses ``his sources in such a way that his audience would be convinced that they were watching a play representing an era and culture remote from their own but, at the same time, touching on events and themes, that had a `local habitation and a name'''~\cite[203]{westerweel_plutarchs_2001}. 
\emph{Coriolanus} is neither a naïve anachronism nor a self-conscious historical allegory.
Like Shakespeare's other history plays, it is a poetic adaptation of historical materials.
It continues Shakespeare's exploration of the idea of history as a poetic object, a manipulable medium through which he examines the process by which the past is rendered meaningful to the present.
With the English history plays and even the earlier Roman plays, his focus was more on the knowing exhibitionism of his characters, of their ironic recognition of the inescapability of their historical situation.

More than a commentary on the Jacobean present by means of a thinly veiled classical setting, the play evinces a subtle reflectiveness on the process of historicization by which private actions become the substance of public memory.
In addition, and as with so much historical writing in the early modern period, it operates in a realistic mode to convey as factual material derived from fictions.
Caius Martius Coriolanus probably never existed, after all, though the authenticity of his story would hardly have been doubted, or relevant, at a time when historical truth was more generously defined.\footnote{On the historicity of Martius, see Kytzler and Salmon.
Martius's story was first declared a fiction by Theodor Mommsen, who, found it to partake more of the poetic than the historical: ``Wer in diesen Erzählungen nach einem sogenannten geschichtlichen Kern sucht, wird allerdings die Nuss taub finden; aber von der Grösse und dem Schwung der Zeit zeugt die Gewalt und der Adel dieser Dichtungen, insbesondere derjenigen von Coriolanus, die nicht erst Shakespeare geschaffen hat''~\cite[26]{mommsen_erzahlung_1870}.
``Whoever searches for a so-called historical kernel in these tales will indeed find the nut to be deaf; but the force and the nobility of this literature attest to the grandeur and the momentum of the time, especially those of Coriolanus, which Shakespeare was not the first to create.''} 
Although not usually aligned with the English tetralogies as history plays in their own right, and even listed under the ``tragedies'' heading in the First Folio, Shakespeare's Roman plays arguably form their own historical tetralogy.
Indeed, the Roman histories have much in common with the English histories, such as their derivation from tragedy, their multiplicity of voices and perspectives, and their moral ambiguity.
Robert S. Miola writes about this diversity of topical material and the sense that English and Roman citizens alike lived ``in a tense, conflicted present, shaped by the pressures of a mythic past and by those of a destined future''
\cite[193]{miola_shakespeares_2002}.
In the English histories, there is always a strong dramatic irony at work.
The fundamental irony, of course, is that a poetic realization of history is entirely fictitious.
It may be derived from accepted facts, but its composition--the arrangement and versification of those facts---is all invention and suggests there is a certain degree of inventiveness to history itself, to both the ways we remember and the ways we record it.
A further irony of historical drama, one that links it to tragedy especially, is that it has a witness: the audience projected by the text even when the text is not serving a public occasion.
But the dramatic irony of the history plays also serves as a meta-historical commentary, and this applies to both the English and Roman histories.
The writing order of the Roman histories is comparable to that of the English histories.
As with the latter, Shakespeare chronicles in the former the most ``recent'' episodes first.
\emph{Titus Andronicus}, set during the decline of the Roman Empire, is the earliest of these episodes and exhibits some of the historical circularity of the English plays by looking both forward, dramatically, and back, temporally, to \emph{Coriolanus}:
\begin{vq}
Arm, my lords! Rome never had more cause.\\
The Goths have gathered head, and with a power\\
Of high-resolved men, bent to the spoil,\\
They hither march amain, under conduct\\
Of Lucius, son to old Andronicus,\\
Who threats, in course of this revenge, to do\\
As much as ever Coriolanus did.\\
\hfill(4.4.62--68)
\end{vq}
\emph{Titus} is another play about a stubborn general who declines public office, rejects the advice of those close to him, and ends up the sacrificial victim of the state he fought for.\footnote{Geoffrey Bullough thinks Shakespeare based much of \emph{Titus Andronicus}, including Titus's refusal to be named emperor and his son Lucius's exile and return, on Plutarch's ``Life of Coriolanus''~\cite[6:24]{bullough_narrative_1957}.} 
His son Lucius, like Martius, is banished from Rome and then returns to revenge himself on the city, though unlike Martius is beloved by the common people and makes his peace with them.
Both plays sit at the intersection of private passions and public, political violence.
As in so many history plays, the affairs of the public are orchestrated by figures who cannot reconcile their private passions with their public offices.
This is, indeed, one of the prevailing dilemmas of the dramatic-historical genre: what happens when private and public interests conflict.

The historical ironies of \emph{Coriolanus} are subtly embedded in the disposition of its language.
The frequency of certain charged terms, many of them unique to this play, along with several sets of telling antitheses indicate that what's at stake is very much the ownership of language and historical identity.
In this sense, the language of the play is idiosyncratically ``public,'' in that it is oriented toward public affairs but it is also language that the play positions as contestable.
The granting of titles, the displaying of wounds, the giving of voices: so many words in \emph{Coriolanus} come across as practically tangible, as if coextensive with the acts they represent, or as tokens to be traded around in the practical business of political negotiation.
One can understand why Martius so objects to using them---his hatred of the people includes their language---and is so often exasperated by the way others do.
In fact, literary critics frequently note that certain words appear more often in \emph{Coriolanus} than in any other play: ``country''~\cite[150]{knowles_shakespeares_2002}; ``grain''~\cite[201]{westerweel_plutarchs_2001}; ``power''~\cite[141]{patterson_shakespeare_1989}; ``Rome'' and ``Roman''~\cite[165]{chernaik_myth_2011}; ``voice(s)''~\cite[250]{kermode_shakespeares_2000}; ``wounds''~\cite[47]{holland_coriolanus_2011}.
One could add to this list ``people,'' which occurs more than five times as frequently as its nearest rivals, \emph{Titus Andronicus} and \emph{Julius Caesar}, and hardly at all elsewhere; ``noble,'' its opposite, which occurs more often in \emph{Coriolanus}, by far, than in any other play; and ``citizen,'' which is noteworthy for even occurring at all.
``Mother,'' an obvious contender, is eked out by \emph{Hamlet} and \emph{King John} and ``pride,'' another one, by \emph{Troilus and Cressida}.
\emph{Coriolanus} comes close to tying with the friendless \emph{Timon of Athens} for ``friend(s)'' but sails past all the rest with ``common,'' ``enemy,'' ``wars,'' and ``work'' (which appears even more often than the psychoanalytically-charged ``wounds'').\footnote{To obtain these results, I matched a selection of the most semantically meaningful top hits in a concordance of \emph{Coriolanus} against the ``Open Source Shakespeare'' online concordance of the complete works.}
Power, wounds; country, people; friend, enemy; noble, common; wars, work: all words that indicate the concepts, several of them near-antitheses, which the play struggles to define.
The vocabulary of Coriolanus is structured by them, and the political orientation of the play is reflected by such oppositions and in the contestable identities they evoke.

Power, and who should have it, is coordinate with its vulnerability.
In order to hold office in Rome, its leaders must expose to the people the marks of mortality they earned on their behalf.
Country is the referent for defining both the external enemy and for testing the true Romanness of the people; allegiance to Rome is not necessarily compatible with allegiance to class.
Martius characteristically denies that the people truly are Romans, typically describing them with language that dehumanizes them entirely:
\begin{vq}
Though in Rome littered; not Romans, as they are not,\\
Though calved i'th' porch o'the' Capitol.\\
\hfill(3.1.240--241)
\end{vq}
Neither proper citizens, nor much more than beasts whose only function is to breed, the people to Martius are the binary opposite of both ``country'' and ``noble.''
Mortality is at issue here, too, because a true Roman's life, as seems to be the case in this play, belongs to the state, and citizenship turns out to be less a voice in power than mute assent to the ``old lie,'' \emph{dulce et decorum est pro patria mori}: Volumnia's claim that if she had a dozen sons ``I had rather had eleven die nobly for their country than one voluptuously surfeit out of action'' (1.3.23--25) is merely the sentimental analogue to Martius's more practical opinion of war, as
% I like this quote here, maybe just delete the later mention of it
\begin{vq}
means to vent\\
Our musty superfluity.\\
\hfill(1.1.220--221)
\end{vq}
Indiscriminate breeding thus excuses the state's treatment of them as indiscriminate, ignoble chaff.
Indeed, the ``people,'' a term that history since Shakespeare has perhaps overdetermined, is at odds in \emph{Coriolanus} with ``country,'' the Leviathan that is more than the sum of its parts---or Menenius's body politic, insensible to its individual constituents wailing for bread.
Certainly, the tribunes's claim that ``the city is the people'' is an alien concept to Martius's patrician sensibility and a violation of his pragmatic, martial ethics.
But Martius does not recognize the people as a class: to him, they are
\begin{vq}
woollen vassals, things created\\
To buy and sell with groats, to show bare heads\\
In congregations, to yawn, be still and wonder\\
When one but of my ordinance stood up\\
To speak of peace or war.\\
\hfill(3.2.10--14)
\end{vq}
They have, that is, no identity, motivation, or even culture of their own.
While the text's representation of them is, on the hand, fluid (are they plebeians? citizens? just ``the people''?), Martius bespeaks the darker perspective that regards politics as, at best, a kind of husbandry.
In his estimation, the people are not really people.
They are only passive recipients of the cultural leavings from the tables of the mighty, exchangers of both small coins and small-minded gossip.
The plebeians may conjecture about the city's supply of grain, about ``what's done i'th'Capitol,'' and about ``who thrives and who declines'' (1.1.187--191), but that is the limit of their capacity.
In his stentorian denunciation of them at the opening of the play, Martius frames the feckless irrationality of the plebeians in similar terms:
\begin{vq}
What would you have, you curs,\\
That like nor peace nor war? The one affrights you,\\
The other makes you proud.\\
\hfill(1.1.163--165)
\end{vq}
He considers them infantile, even animalistic, and his lack of faith in their ability to reason about all that lies between peace and war makes his question more than rhetorical.
He answers for them, because they cannot, in his estimation, answer at all.
More to the point, one cannot predict their behavior, because they do not act in accordance with propriety.
Whether peace or war ``affrights'' them or makes them proud, they are as equivocating as Martius's grammar is ambiguous.
They are ``curs'' that would eat each other if the senate did not keep them under control.
They are, in other words, as easy to put in awe as they are irresolute in their opinions.
When he describes them as a ``mutable, rank-scented meinie'' (3.1.68), it is significant both because he expresses his fear of granting power to a heterogeneous, disputatious populace and because it is Shakespeare's only recorded use of the rather Spenserian epithet ``mutable,'' an apt adjective for a crowd that is characterized, as we will see, as a close cousin of Spenser's own allegory of uncontrollable slander.
The idea that such creatures be given a share in the determining of state affairs would obviously make no sense to someone of Martius's opinion.
Like an unpredictable beast, the people do not think; they can only react.

``Peace'' and ``war'' appear to cover the range of ideological positions available to Martius if not to the Romans, in general.
From either extreme to the other, all discourse is contained, and none of it is available to the plebeians, whose own voices are only so many tokens to be offered mutely up when demanded.
Despite the prevalence of the word ``voices'' in the play, and the onus put on a potential consul's winning of them, they are little more than interchangeable, inarticulate acclamations.
Collectively, they have the power to ratify state decisions, but individually they can do nothing and say nothing, and there is very little they say in the play except to one another.
Possessing no real value, they exist only to be surrendered, just as the plebeians themselves exist only to be sent to the slaughter in war.
Their ``voices'' serve only to signify their actual voicelessness.
Interestingly, Martius also associates the people with custom, to him the frustratingly senseless traditions that hold even the powerful in thrall.
In one of his few moments of solitary reflection, he wonders aloud why he must petition for the ``needless vouches'' of ``Hob and Dick.''
We may like to imagine him channeling Pindar's oft-quoted line that ``custom is king of all'' when he laments that
\begin{vq}
Custom calls me to't.\\
What custom wills in all things, should we do't,\\
The dust on antique time would lie unswept,\\
And mountainous error be too highly heaped\\
For truth to o'erpeer.\\
\hfill(2.3.115--119)
\end{vq}
With his rejection of error and resort to sententious couplets, Martius sounds here almost like a humanist.
Even as he recites a familiar-sounding maxim---and it is hard, coming from him not to read it ironically---he nevertheless decides to ignore his own advice and proceed with what custom demands, which is to act as if matters of peace and war really do belong to public discourse.
We are not surprised that he can only keep the act up for so long.
What is particularly vexing about his condescending solicitation of the people, perhaps about his entire characterization, is his honesty, how accurate his view of politics appears to be.
The patricians placate the plebeians, because they must, but their strategy is to offer them a form of representation that only mirrors existing relationships of domination.
The tribunes they allow the people to elect understand as well as the aristocrats the imperative to manipulative the crowd.
With Martius, they at least know what they are getting, but it is his inadequacy to conform to the public discourse that is only right because it does control the crowd that dooms him.

We can trace Martius's invective against the people and his aversion to public life to the private language style imposed on him by his mother.
For a character as archetypically masculine as Martius---his family name even invokes the Roman god of war---the absence of any paternal figure for him is striking.
His seeming disinterest in his own paternal role parallels this missing element in his back story.
Instead of a father, Martius has a mother---a particularly masculine mother, perhaps, in that she rejects the insular security of the family in favor of the civic affairs generally reserved in the Roman (and early modern) world for men.
But she is still a mother, and her molding of Martius's disposition derives equally from his ``biological'' birth and from their metaphorical near-marriage.
The first words Volumnia speaks, in scene 1.3, provocatively close the gap between husband and son: ``If my son were my husband, I should freelier rejoice in that absence wherein he won honour than in the embracements of his bed'' (1.3.2--4).
Later, as Martius is preparing to depart from Rome, he reminds his mother of this sentiment.
Promising to ``exceed the common'' (4.1.32) in what he accomplishes, he uses grammatically similar language to chide her in the same way that Volumnia, at the beginning of the play, rebukes his wife:
\begin{vq}
you were wont to say,\\
If you had been the wife of Hercules,\\
Six of his labours you'd have done and saved\\
Your husband so much sweat.\\
\hfill(4.1.16--19)
\end{vq}
The repetition of conditional phrasing reinforces his reminder to his mother that they share a vocabulary, a common supply of apothegms, perhaps, but also the secret dialect of husbands and wives.
The distinction between husband and son narrows further in the lines that follow:
\begin{vq}
Farewell, my wife, my mother.\\
I'll do well yet.\\
\hfill(4.1.20--21)
\end{vq}
Perfunctory in his salutation to his actual wife, who is everywhere in the play an afterthought, it is the adjacency and grammatical parallelism of ``my wife, my mother'' and the encapsulating internal rhyme of ``yet'' with ``sweat'' that indicates the extent to which Volumnia is closer to occupying this role and to closing with Martius a proprietary linguistic circle.
Absent a father, Martius would seem to be entirely the creation of his mother, a work of ideal Roman sculpture for which she is solely responsible.
The bond between mother and son is like that between artist and art, or poet and poem.
Once the artwork is in the world, of course, it tends to take on a life and significances of its own.
Martius is as confused that his mother does not approve of his actions in the marketplace as he is averse to exposing himself to public scrutiny.
He cannot go beyond the singular text Volumnia has taught him, even when she would like him to adopt a certain amount of interpretive flexibility toward it.
Martius, however, is not interested in this sort of ambiguity.
In his resistance to political contingency, he acts as though he were identical to the text of Roman values that most people only pay lip service to.
And his capacity to be so uncompromising is greater than the city's to live up to the Rome of his imagination.
He even manages to frame his banishment from Rome as a banishment \emph{of Rome}, as if the most Roman act would be to abandon---and then seek means to destroy---a Rome that has become its own antithesis.
His actions always reference an extreme self-possession, a knowing of himself that cannot permit being known by others and, divorced from a reciprocal social context in which identity is intersubjective, turns inward.
One of the reasons Martius, as we will see, comes across as paradoxically prideful and self-abnegating at the same time is this stubborn self-reinforcement.
At the core of his stubborn resistance to dialogue, there is, in fact, nothing: only the language his mother taught him and the text it continues to write, spiraling centripetally away from the outside world.

If the counterpoint to the language of politics and public identity in \emph{Coriolanus} is the private linguistic domain of Martius and his mother, we should expect to find some evidence of this bifurcation in the text.
As dialogic as the play's language is, public and private are not themselves terms that Shakespeare appears to be much interested in.
They do not occur often in any of his plays, even though, as concepts if not political constructs, they determine the very substance of his poetic art.
The public declamations and rhetorically rich language that are native to a genre meant to be performed before an audience are balanced by those idiosyncratically Shakespearean moments of reflection that we are meant to regard as internal thoughts, private though not in the sense of property.
In \emph{Coriolanus}, there are comparatively few soliloquies, but there is a sense of propriety about public versus private language.
There is, of course, the public language of civic life: the negotiations, appeals, and honoraria with which the play is filled.
Then there is the private language of the family, of sentimentality, certainly, and of social strategy, but also the private language that is concealed, inferred, or---as in Martius and Volumnia's final meeting---cannot be put into words.
The word ``private'' itself often refers off the page, to an unwitnessed space.
While the showing of wounds is supposed to be public, Martius offers a private audience instead, as if he does not want us, the public and his posterity, to see them, either:
\begin{vq}
I have wounds\\
to show you which shall be yours in private.\\
\hfill(2.3.75--76)
\end{vq}
There is also a sense in which private and public identities are distinct. 
One's private interests could indeed be at odds with the public welfare, and private affairs are often in Shakespeare's plays impugned as a corruption of what should be public processes.
With Martius before the gates of Rome, this sense of private privilege is invoked by the emissaries who endeavor to treat with him.
When Cominius reports on his attempt to dissuade him from attacking the city, he says,
\begin{vq}
I offered to awaken his regard\\
For's private friends. His answer to me was\\
He could not stay to pick them in a pile\\
Of noisome musty chaff.\\
\hfill(5.1.23--26)
\end{vq}
Since Martius's private interest is only in the language world he shares with his mother, he does not even entertain the insiderism of his own class.
Where private friends normally might intervene in public diplomacy, Martius refuses to acknowledge the usual channels, a move that earns a compliment from Aufidius that almost seems back-handed.
Martius, he says,
\begin{vq}
Stopped your ears against\\
The general suit of Rome, never admitted\\
A private whisper, no, not with such friends\\
That thought them sure of you.\\
\hfill(5.3.5--8)
\end{vq}
Martius is above the machinations of elite society, but he also betrays his inability to express loyalty to anyone.
At this point in the play, he has abandoned all possibility of this sort of friendship, but his volte-face is only be the fulfillment of his characterization from the beginning.
He has no friends in either sense of the word: as fellow aristocrats shaping society according to their whims or as intimate partners privy to his council
Before the entrance of Volumnia and Virgilia that will lead to his undoing, Martius disclaims that he will hear entreaties ``nor from the state nor private friends'' (5.3.18).
Neither public, ``state'' entreaties, will change his mind, nor the pleading of his peers.
His mother will turn out to be the exception, because she occupies an entirely different realm of privacy for Martius.

``Private'' often means that we, readers and audience, do not get to hear what is spoken, either.
Or, we are not supposed to know.
Once we do know, if a supposedly private utterance is delivered on stage, it is longer private.
This is the view into history that historical drama offers: the publicity of private exchanges.
When, for dramatic purposes, language is concealed, it is not as if characters are speaking to one another off the page.
Gestures to extra-dramatic action are tantalizing, but only insofar as they foreshadow a public revelation to come.
The withholding of information beyond the end of a play is as frustrating as the withholding from the public of what it wants to know: which is, in short, everything.
The private, in other words, is threatening.
It is anti-social, on the one hand, a realm of secrecy that works toward its own ends.
This is the accusation leveled by the plebeians against the patricians, the fundamental tenet of the revolt that creates the \emph{res publica}, the public entity, in the first place.
On the other hand, the private is sinister, an unexposed area of consciousness that can offer space for contemplation but just as often indicates the perfidy of hidden motives and bad faith.

In a drama, the private can take various forms, each a degree further removed from the theoretical public gaze until it is removed from the theatrical gaze, as well.
We see both the strategy sessions of the aristocrats and the councils of the tribunes in \emph{Coriolanus}.
We even see in passing the colloquy of the plebeians, who possess their own private opinions about the significance of public language.
What is private can also insinuate.
Julius Caesar has a public excuse but also gives private reasons to Brutus for not going to the Senate; Antony wonders publicly about the private griefs between them after Brutus persuades him to go anyway.
The private is frequently the reserve of the villain, too, who implicates us in his plot by revealing to us his plans or motives.
Richard III, whose private speech makes us counterparts to his crimes---the machinations of a dangerous, private ambition---contrasts markedly to Richard II, who misrepresents his own private interest as identical with the public, until Bolingbroke reenters the scene with private interests of his own.
And Bolingbroke, of course, introduces the private speech of no speech at all, or speech we do not witness.
It is only with the necessary aid of historical retrospect that we can know what he is about.
If Henry V regards the difference between kings and private men to be nothing but ``ceremony,'' Richard II nevertheless suffers the consequences of eliding this distinction, even as Henry IV is perhaps not ceremonious enough.
In these respects, Martius is not politic about what he says, where, and to whom.
He is not exactly a Shakespearean villain in the usual sense, either, precisely because he does not keep any part of himself hidden.
He may refuse to recognize the existence of a public at all, and he may prefer to retreat into the idiolect of his mother's devising, but to us he all surface and no depth.
With no hidden motives to conceal and no subtle introspectiveness to reveal, he has only the fixed form of a stereotype.
Impatient and self-assured, his language is entirely public in the dramatic sense, with no hidden reserve of complex subjectivity, while it is entirely private, even privative, in the way in which he uses it.
He lets it bluster forth at the slightest provocation, having no stomach for long term plans and the slow unfolding of a deliberate plot.
He is neither a would-be playwright, like Edmund, Iago, or Duke Vincentio, nor even a neurotic improviser, like Macbeth, Hamlet, or King Lear.

The public that is not merely the theatrical public of the playhouse but a category of social experience has its own significations.
The word itself appears seldom in Shakespeare, but touches on a range of meanings where we do find it, from collective concern, accountability, and memory, to spectacle and shame.
When not paired with some ``good,'' ``weal'', or ``benefit,'' the public is the space of civic deliberation, commendation, and illocutionary occasion.
It is also a venue of accusation, justice, and even farce.
The only occurrences in \emph{Coriolanus} are when the tribunes variously refer to the ``public benefit,'' the ``public weal,'' and ``public power.''
The idea of a collective benefit is conventional enough, though here it is tinged with the resonance of class.
``Public power'' almost seems like an oxymoron in the Roman context, an amorphous, disjointed authority that opposes power's traditional, patrician seat.
It is a strength in numbers that coalesces around the tribunes's exhortations: feckless, perhaps, in most circumstances but irresistible when it is able to be guided.
Elsewhere in Shakespeare, we encounter ``public streets'' and ``public haunts'' where private people may go to debase themselves or else melt into the undifferentiated crowd.
It is this sense of public more than any other than may indicate Shakespeare's own distaste for the crowd, however much that could be said to have translated over to his plays.
A well-known argument about his Sonnet 111 supposes that it was one of a handful written early during the reign of James.
In it, he quite possibly expresses his true feelings about the public and his disappointment that, despite his gifts, he must consort with social inferiors in order to make a living:
\begin{vq}
O, for my sake do you with Fortune chide,\\
The guilty goddess of my harmful deeds,\\
That did not better for my life provide\\
Than public means which public manners breeds.\\
\hfill(\emph{Sonnet 111} 1--4)
\end{vq}
Although there is debate about the exact meaning of these lines, it is clear enough that the speaker thinks little of his ``public means.''
While some critics have gone so far as to impute these lines to the poet's frustration that, while he may have performed at court, he was never a part of court society, it is enough to state here that the poem at least references a common assumption about the merits of anything qualified as public.\footnote{On the subject of the sonnet, its usefulness as biographical evidence, and the possibility that it was part of a poetic dialogue in the early seventeenth century, see MacDonald P. Jackson. ``Shakespeare's Sonnet cxi and John Davies of Hereford's ``Microcosmos'' (1603)''. The Modern Language Review 102.1, 2007: 1--10. Print.\nocite{jackson_2007}}
Without having to guess at Shakespeare's own opinion, it is evident from his language here and elsewhere that the public is not universally presumed to be a neutral social category, much less the most important social category.
Having to engage with the vulgar public was an unfortunate necessity, avoided if possible, not embraced in service to some sort of activist dramatics.
Although it may receive the largesse of the state as its benefit, that is, we do not often find writers referring to the people or the public as a legitimate political actor.
More typically, the public has the detractive quality suggested by the negative parallelism in Sonnet 111.
``Breeds'' in this poem cuts two ways, alluding to both animalistic reproduction and indoctrination into polite society.
Except ``public manners'' are implied to be the opposite of good breeding.
It is as disgraceful to be public as it is to be common, and the poem indicts the breeding of public manners, just as the plebeians in \emph{Coriolanus} are accused by the patricians as being unproductive, yet ever reproducing, burdens on the true citizens of the Rome. It is possible that Shakespeare is only echoing a widespread sensibility that he does not share, but given the number of reverberations, evidence to the contrary is far more persuasive.
It is in \emph{Hamlet}, we should remember, where Shakespeare himself possibly invents the term ``groundlings'' to describe the better part of his own customers.

Where the people do act collectively, they are most often portrayed, following the traditional metaphor, as a body.
The ``public ear'' is appealed to, the ``public eye'' witnesses, and the ``public body'' as a whole may be a metaphor particularly suited to Shakespeare's Roman plays, so much does the city itself act in them as a character in its own right.
In \emph{Coriolanus}, the body politic moves in its own course, separate from the individual people who comprise it, and we should not confuse the two.
The public body, in this play, is like a beast that must be appeased, even if the plebeians as individuals come across as perfectly guileless.
\emph{Julius Caesar} arguably set the precedent for deploying the public as a sounding board for justifications and official record-keeping.
And history hinges on it: after Brutus gives the official, public reasons for Caesar's death, Antony is granted leave to speak from the ``public chair'' on the same subject.
A dramatic irony perhaps unintended by Shakespeare is the tableau that Antony dares to enact in the ``public eye,'' thus proving to Octavian the irreconcilable importunity of Antony himself in \emph{Antony and Cleopatra}:
\begin{vq}
Contemning Rome, he has done all this, and more\\
In Alexandria. Here's the manner of 't:\\
I' th' market-place, on a tribunal silvered,\\ 
Cleopatra and himself in chairs of gold\\
Were publicly enthroned: at the feet sat\\
Caesarion, whom they call my father's son,\\
And all the unlawful issue that their lust\\
Since then hath made between them. Unto her\\
He gave the stablishment of Egypt; made her \\
Of lower Syria, Cyprus, Lydia,\\
Absolute queen.\\
\hfill(\emph{Antony and Cleopatra} 3.6.1--11)
\end{vq}
It is only in public that socio-political relationships in these plays become instantiated.
The ceremony of public investiture grants politics its value and politicians their legitimacy.
Even if the public is only a passive witness or, as in \emph{Coriolanus}, the guarantor of laws and traditions, it is nevertheless a kind of mirror in which the powerful need to see themselves reflected.
It is also a public concern what the story of the state and state actors is going to be.
What its heroes performed on its behalf form the core of the founding myths that determine what it means to be a citizen and how best to be a Roman.
The future of the state is thus imagined, and worried over, by its own past, and the past becomes for the present a justification for its self-fulfillment.
The combination of dramatic past with theatrical present keep both temporal perspectives in dialogue and in tension.
The irony of Cleopatra fretting about her portrayal on stage by a squeaking boy is not so different from the speculation twice entertained in \emph{Coriolanus}, once by Aufidius and once by Volumnia, about ``the interpretation of the time'' (4.7.50 and 5.3.69)

In attempting to deduce the significance of the unique vocabulary of \emph{Coriolanus}, we might speculate more broadly as to its political significance, as well.
Politics, as this play confesses, provides a particularly suitable arena for dramatization, but is politics in this play merely content, or does \emph{Coriolanus} demand of us a more sensitive account of its impingement on an area of life that seems transhistorical?
The frequent recurrence of charged words has meant that \emph{Coriolanus} uniquely lends itself to explicitly partisan readings.
Moreover, \emph{Coriolanus}'s long-standing reputation as strange and unlikeable---attributed variously to Shakespeare's declining poetic powers, increased resort to collaboration, pandering to changing audience tastes, or some combination thereof---has meant that the play seems to need not only explaining but some kind of accounting for as an otherwise odd member of the corpus.
The bulk of critical interest in \emph{Coriolanus} has concentrated on the political context in which it was written and which it has often been presumed to represent, and in the psychodynamics of Martius's relationship with his mother.\footnote{For an overview of recent criticism, see Lee Bliss. ``What Hath a Quarter-century of \emph{Coriolanus} Criticism Wrought?'' \emph{The Shakespearean International Yearbook} 2 (2002): 63–75. \nocite{bliss_what_2002}
Print.} 
As little as we might like to be reductive either about Shakespeare's politics or his dramatization of interiority, the issues involved have always seemed so blatant, and, \emph{a fortiori}, so patently relevant to major ideological discourses of our own time---Marxism, Fascism, Freudianism---it is almost impossible to write about \emph{Coriolanus} without, as it were, choosing sides in the ongoing debate as to the play's political bias.
But these kinds of teleologies inevitably cast our own prejudices back onto a text that predates the party politics and class consciousness that informs them.
We may today be hardly able to brook the thought of a Shakespeare unsympathetic to the working man (or rude mechanical), but this play seems more a project in historical introspection---who were these people and what do their actions mean to us---than a piece of barely disguised propaganda.
Indeed, prior to the recent turn in early modern literary criticism toward the New Historicism, readings of \emph{Coriolanus} tended to focus more on the play's tragic, rather than political, qualities.
To the extent that politics in the play was touched on at all, it was generally treated as one among many possible topics of dramatic content if not merely banal proof of Shakespeare's bourgeois class snobbishness.
William Hazlitt considered Shakespeare ``to have had a leaning to the arbitrary side of the question,'' representing both patricians and plebeians relatively evenly, warts and all~\cite[15]{hazlitt_coriolanus_1995}.
In this, he is in agreement with Coleridge, who superciliously remarks in a marginal note on ``the wonderful philosophic impartiality of Shakespeare's politics''~\cite[177]{coleridge_coleridges_1989}.
A. C. Bradley, who in his lecture on the play thinks it ``extremely hazardous to ascribe to him any political feelings at all, and ridiculous to pretend to certainty on the subject,'' regards the representation of the people as undoubtedly copied over from Plutarch, probably conventional, and in any case only the catalyst of the play's tragic design: they are the force with which Coriolanus unavoidably and hopelessly comes into conflict~\cite[5]{bradley_coriolanus_1970}.
Annabel Patterson, whose own reading of the play does posit a hidden political agenda, nevertheless perceives \emph{Coriolanus} as a challenge for critics today precisely because the political dimension is so hard for us to ignore.
And given the emphasized role of history in literary studies in general, she finds that we must negotiate between the deterministic source criticism the play is felt to demand and a relativistic skepticism that threatens to evacuate it of political specificity entirely.
This is partly a problem of disciplinary approach.
Without recourse to history as the bedrock of all interpretive structures, suggests Patterson, and the literature profession's faith in historical explanation having been undercut by postmodern reasoning, we are left flailing between a smug assurance that all textual witnesses are ideologically compromised and an ever-vanishing nostalgia for the sureties of an objective method.
While Patterson notes that these are extremes, ``and the case of \emph{Coriolanus} demands a peculiarly exacting poise between them''~\cite[122]{patterson_shakespeare_1989}, we are still left by this sort of theoretical positioning with the uncomfortable choice between historical determinism and the slipperiness of internal evidence.

Nevertheless, it can seem irresponsible to either avoid the question entirely or ignore how much the text demands political appropriation.
Indeed, there is potentially attractive material in \emph{Coriolanus} for holders of any political position: partisans of either the right or the left, in the reductive terminology of today, who want to use the text as ideological justification; historical revisionists who claim Shakespeare tacitly supports the people; cynics who believe he is coldly representing a political reality in which no one is sympathetic; and ironists who believe he is simply lampooning the excesses of political life.
Frequently, adaptations of the play for performance also resort to choosing a side explicitly, whether as an opportunity to promote a specific agenda or lest they appear mealy-mouthed.
The Tory playwright Nahum Tate was already doing this as early as 1682, discerning in \emph{Coriolanus} a political situation comparable to that of late seventeenth century England and in Martius an analogue to James, the embattled Duke of York.
Tate was not at all reluctant to be heavy-handed in his version of the play, to which he gave the not-so-subtle title \emph{The Ingratitude of A Common-Wealth}.
In his preface, he is unabashed: ``Upon a close view of this Story, there appear'd in some Passages, no small Resemblance with the busie Faction of our own time.
And I confess, I chose rather to set the Parallel nearer to Sight, than to throw it off at further Distance''~\cite[3]{tate_dedication_1995}.
By placing ``the Parallel nearer to Sight,'' Tate evidently meant delineating his character's stock functions so baldly as to permit little interpretive license.
Whereas Shakespeare's \emph{Coriolanus} is able to leave us feeling the distaste inherent in all statecraft, and uneasiness about the imperfections of justice, Tate's play strikes at us with gratuitous, sensationalistic violence.
In case a spectator had any doubts about Martius's victimhood, Tate invents a new character, Nigridius, who plots against him Iago-like, ultimately forcing Martius to witness the death of his wife and the dismemberment of his son before he dies himself.
Whatever we may think of Tate's play as a literary work, and the reviews have not been stellar, we are much less in the dark about his own political inclinations than we are about Shakespeare's.\footnote{For an overview of the Coriolanus tradition and Tate's place in it, including a critical edition of his adaptation, see Ruth McGugan. \emph{Nahum Tate and the Coriolanus Tradition in English Drama With a Critical Edition of Tate's The Ingratitude of a Common-Wealth}. New York: Garland, 1987. \nocite{mcgugan_nahum_1987}
Coincidentally, the three Shakespeare plays adapted by Tate, \emph{Richard II}, \emph{Coriolanus}, and \emph{King Lear}, are also the three with which the present study is concerned.}
Where Shakespeare's text has the sort of writability that permits a dialogue about its political valence---in keeping with its ideologically equivocal, dialogic nature---Tate's forecloses ambiguity and gruesomely ends the conversation.

While interpretive ambiguity is fine for a reader of the text, however, it is a dilemma for theatrical producers who feel that the political content of \emph{Coriolanus} demands a performative response with contemporary relevance.
Tate's eager embrace of this dilemma is not without more recent analogues.
Bertolt Brecht, whose adaptation of the play, \emph{Coriolan}, is among the most noteworthy, embraced the dilemma in his own way.\footnote{For the original text, see Bertolt Brecht. ``Coriolan''. \emph{Gesammelte Werke}. Vol. 3. Frankfurt: Suhrkamp, 1967. 2396–2497. Print.\nocite{brecht_coriolan_1967}; and Bertolt Brecht. ``Coriolanus''. \emph{Collected Plays}. Ed. Ralph Manheim \& John Willett. Trans. Ralph Manheim. Vol. 9. New York: Random House, 1972. 57–146. Print. \nocite{brecht_manheim_1972}} 
Rather than nuancing his characterization of Martius, Brecht goes in the opposite direction from Tate, using him as a bludgeon to subvert what he believed to be the playgoing audience's tendency to over-identify with tragic heros.
What he removes from his version are those mollifying and mitigating moments of psychological complexity that would put a strictly political interpretation of the play at variance with what appears to be Shakespeare's more subtle dramatic project.
Martin Scofield, in his critique of this strategy, suggests that Brecht's modifications actually undermine his own intention: ``In focusing more exclusively on the class struggle, Brecht has not only lost the full dimensions of the individual tragedy which he is presumably content to lose but also a political dimension.
What is important to Brecht is the collective dilemma, the tragedy of a people that has a hero against it.''~\cite[335]{scofield_drama_1990}.
As most critics have noted, Shakespeare makes the psychological basis of Martius's behavior explicit.
The elaboration of psychological depth can be humanizing, but it can also hold out the possibility for rehabilitation and reintegration into society, a conciliatory complexity that is not part of Brecht's mission.
Without this extra dimension to Martius's characterization, specifically, according to Scofield, we lose the sense that politics begins with personal relationships, not with an abstract calculation about class divisions.
Equally skeptical about Brecht's political equation was Günter Grass.
Grass's own adaptation of Shakespeare's play, \emph{The Plebeians Rehearse the Uprising}, is actually posed as a dramatization of Brecht making of \emph{Coriolan}.
It features a decidedly Brecht-like figure, ironically named ``The Boss,'' who takes advantage of the real-life uprising in East Berlin in June 1953 to help him work out a staged uprising for his production of \emph{Coriolanus}.\footnote{Günter Grass. ``Die Plebejer Proben Den Aufstand.'' \emph{Werke}. Göttinger Ausg. Vol. 2. Göttingen: Steidl, 2007. 357–429. Print; \nocite{grass_plebejer_2007} Günter Grass. \emph{The Plebeians Rehearse the Uprising: A German Tragedy}. Trans. Ralph Manheim. New York: Harcourt, Brace \& World, 1966. Print. \nocite{grass_plebeians_1966}} 
``The Boss,'' however, turns out to be just as uncompromising as Martius himself, which suggests that Grass's dramatic argument is the opposite of Brecht's: over-identification with ``the people'' is just as problematic as over-identification with a tragic hero.
In his play, the tragic protagonist turns out to be the same sort of ``heroic'' individual whose inability to compromise is what makes him a hero in the first place (he is an artist with a vision) but also alienates him from the people on whose behalf he thinks he is working.
If Grass's Boss, and by extension Brecht, is merely an opportunist, it is no great insight of Grass to have pointed out the contradictions in Brecht's attempt at literary appropriation or the incompatibility of the contrived political calculations of drama and actual, real life politics.
As dramatic as politics may be, a theme of all these plays, politics does not have a plot.
There is no narrative arc bending inexorably toward resolution.
As Brecht understood, drama---tragedy specifically---has the potential to subvert political intent.
But as Grass understood, speaking on the people's behalf can be just as totalitarian as silencing them.
We can only recover something of that intent for \emph{Coriolanus} if we attempt to gauge the predominant tone of its language, somehow incorporate the tragic denouement into a reading that succumbs to neither pity nor exultance, and do our best to step back from our own historical context.
If we assume Shakespeare's sympathies were somewhere on a spectrum comprehensible to our own, we risk making him, too, into a Grassian opportunist.
Most readings of \emph{Coriolanus}, however, struggle with its treatment of politics, whether a political interpretation is practicable at such a remove and what happens to politics when it is historicized.

Although the binarism that defines the political scene in the Anglo-American world today is hardly analogous to that which obtained in early modern England, \emph{Coriolanus} is, among all of Shakespeare's plays, the most ready-made for political appropriation.
Cast into modern terms, the conflict between the patricians and the plebeians appears to be the Roman original of all subsequent class struggle in history.
Indeed, to our democratic sensibility today, the complaints of the plebeians sound entirely justified.
Shakespeare represents them as having an abstract desire for an equitable distribution of wealth and an enlarged franchise, even as the concrete demands they express themselves are for the alleviation of immediate needs.
He invites us to infer their qualities as a public by what the patricians say about them, how their own tribunes regard them, and only lastly by what they individually, and without individual identity, have to say for themselves.
Their goals, if our inferences ascribe to them contemporary political motives (and not even motives more contemporary to Shakespeare's time than to that of the early Romans), are precisely the goals of the modern, liberal state.
A more conservative reader, however, could just as easily interpret the play as an apology for enlightened oligarchy, patriotic militarism, and the cult of the heroic, self-sufficient individual.
Such a reading is not necessarily anti-democratic, or does not have to be.
The rub is how to translate the idea of the commonwealth into a constitutional basis for pragmatic action.
The welfare of the people, and how to bring it about, invokes a spectrum of philosophically incompatible, if not quite contradictory, positions: the freedom that entails the responsibility to determine one's own destiny is as attractive as the freedom that enables sufficiency by provisioning surplus resources to the less fortunate.
A debate along these lines, however, can easily overlook the far more primitive political situation of \emph{Coriolanus}'s Rome.
As if a sort of Hobbesian experiment, this earliest of urban populations figures out, before our attentive eyes, the necessity of articulating a body politic as well the consequences of that body having to contend with disaffected members.
We see this disaffection at the outset with the people's revolt, but we also see it in Martius's intransigence.
As a protagonist, Martius's stark portrayal cannot easily be redeemed by mitigating nuances.
Depending on the reader's own political persuasion, his denunciations of the people mark him as a political reactionary of the most despised variety or as the heroic embodiment of martial virtue and libertarian autonomy.
If one believes Shakespeare to be standing up for traditional, even conventional, beliefs, then Martius is indeed a tragic figure who falls prey to a fickle, ungrateful populace.
If one sees Shakespeare as more radically subversive, however, then Martius's bombast is a self-indictment of aristocratic arrogance and excess.
His role in the play would then be ironic as opposed to epic and his death a victory for the people rather than the inevitable fate of misplaced, romantic heroism.

It is also possible to stand aloft from politics and consider Martius as virtuous if only because he is so resolutely attached to his convictions.
Pointing to the scene of his death and his scornful repudiation of Aufidius, R. B. Parker, in the introduction to the Oxford edition, admires his resoluteness, even if it is in service to personality traits we might otherwise find distasteful.
According to Parker, ``one cannot help but empathize with this last, defiant gesture, because Martius stands suddenly again for what he has always represented at the deepest level of the play''~\cite[69]{parker_coriolanus_1994}.
Along the same lines, but coming to the opposite conclusion, an unsympathetic interpretation of Martius might even characterize him as so excessive in his bombast and pomposity as to be risible.
No less a theatrical authority (and advocate of the people) than George Bernard Shaw considered the play ``the greatest of Shakespeare's comedies'' because Martius so admirably represents an ``instinctive temperament''~\cite[509]{shaw_man_1962} and precisely because he remains so obdurately ``true to himself.''
Shaw thinks of Martius, that is, as more a caricature of a stock cynic than a worthy adversary representative of any discernable political philosophy. 
Many critics have taken Martius seriously enough, however, to consider Shakespeare's depiction of the plebeians unjust, finding in him a conservative, even elitist, attitude which does not comport with modern assumptions about class conflict.
The degree to which this makes us uncomfortable often determines the extent to which we are willing to ascribe these attitudes to the dramatist directly or else find a convenient way out.
Looking back to the early seventeenth-century with the political perspective of today, it is possible to apologize for Shakespeare by adjusting the focus of interpretation.
In so doing, we can find in him a liberal humanist who invites his audience ``to contemplate an alternative political system'' and a dramatic debate about the nature of representative government: ``who shall speak for the commons''~\cite[127]{patterson_shakespeare_1989}; or a Marxist demystifier, who---with the helpful aid of New Historicist suspicion---figures the ``primal bargain'' of representation, which would appear to be the plebeians's best hope, as not a revolution but instead ``the containment of mutiny, the transformation of mutiny into an empty ritual''~\cite[191--192]{arnold_third_2007}.
Or, by making history into an excuse altogether, we could blame the cultural context in which Shakespeare had to work.
Earlier critics who did take up the political content of \emph{Coriolanus} as more than a colorful background tended to go this route.
C. C. Huffman argues that \emph{Coriolanus} encodes a response to contemporary Jacobean politics, in which the absolutism of royal power contested with the humanist ideal of freedom.
Since it was unviable, not least because he was a member of the King's Men, for Shakespeare to openly advocate democratic principles, he chose a Roman setting, according to Huffman, as a neutral venue for working out what was a classic problem in politics: extreme positions that will brook no compromise.
In this formulation, it is the nature of politics to inevitably destroy the steadfastly virtuous.
Though Huffman makes more of the play's politics than his predecessors, his approach ultimately makes \emph{Coriolanus} seem more anodyne than we (or he) would like, a dramatic exercise in humanistic intellectualizing that advances no specific political agenda of its own and therefore tacitly supports the status quo ``to the despair of generations of later audiences''~\cite[222]{huffman_coriolanus_1972}.
Critiques predicated on sympathy for the proletarian plebeians may or may not excuse Shakespeare's treatment of them, which is hardly sympathetic itself.
Such readings of the play can at best figure it as a graceless mirror of elitist sentiment and at worst indict its author as a collaborator with oppressive authoritarianism.
Alternatively, a radical reading of the play could resist Shakespeare's putative intention and celebrate exactly the plurality and multivocality of the people which Martius (possibly Shakespeare himself) finds so dangerous.

There is some evidence from historical research of \emph{Coriolanus}'s unusual degree of accuracy in its political representations and treatment of contemporary events.
Many recent historicist readings of \emph{Coriolanus} have placed substantial stock in Mark Kishlansky's approving citation of the play as accurately representing in its Roman setting the procedures of Parliamentary selection in early modern England.
The depiction of Martius's standing for office, notes Kishlansky, deviates entirely from what Shakespeare found in Plutarch and Livy:
\begin{bq}
The episode of Coriolanus's consulship is entirely of Shakespeare's devising.
Moreover, these scenes so accurately portray the process by which officeholders were selected in the early seventeenth century that one must conclude that Shakespeare had first-hand experience, either of wardmote selections to the London Common Council or of parliamentary selections themselves.
It is rare to have the testimony of so acute an observer.
Thus it is worth reflecting on the central tenets of selection as Shakespeare recreated them.~\cite[5]{kishlansky_parliamentary_1986}
\end{bq}
Just as Martius is soliciting the \emph{assent}, not the \emph{consent}, of the people for his accession to the consulship, based not on campaign promises but on the strength of past deeds, Parliamentary offices in Shakespeare's time would have been filled in advance as rewards for services rendered.
The only role of the ``electorate'' would have been a rather meaningless ratification of what was a predetermined outcome.
The failure of the people to grant their assent was regarded, as it is in \emph{Coriolanus}, as a catastrophic failure of the system and not a regular part of electoral process.
But this was already changing at the beginning of the seventeenth century.
More and more, local Parliamentary selections were butting up against middle class disenchantment with the intractably absolutist king.
As he puts it in his account of English political development, Kishlansky argues that
\begin{bq}
assent would become choice, and the dire consequences of rejection would recede.
For the first time election---that is, contests among candidates for majority decisions---would become an important element in the system by which men were chosen to Parliament.
A process of social distinction would give way to one of political calculation, and along the way England would be brought as close to collapse as was \emph{Coriolanus}'s Rome.~(8--9)
\end{bq}
The crux of \emph{Coriolanus}, for Kishlansky, is the conflict between the upstart plebeians, who want a greater voice in political affairs, and the old order of patricians, who want to retain their traditional privileges; the increasing power of Parliament at odds with a king who preferred keeping his subjects subjugated is thus dramatized by Shakespeare as the classic political struggle of republican theory.

Since long before Kishlansky's supposition about the play's historical accuracy, critics have customarily identified \emph{Coriolanus} with the anti-enclosure riots in the Midlands in 1607.\footnote{For the connection of the Midlands Revolt, as well as other contemporaneous uprisings, to \emph{Coriolanus}, see Holland 69--70. For the original argument, see also E. C. Pettet ``\emph{Coriolanus} and the Midlands Insurrection of 1607''. Shakespeare Survey 3 (1950): 34–42. Print. \nocite{pettet_coriolanus_1950} If Shakespeare was as interested in the anti-enclosure riots as critics have long maintained, however, it is worth pointing out that 1) ``it is not clear whether the rising corn prices, which reached their height in 1608, were sufficiently felt in the midland counties during the early part of 1607 to form one of the immediate causes of the revolt''~\cite[213]{gay_midland_1904} and 2) in the play itself, the rising of the plebeians occupies hardly more space than the length of the first scene---unlike in Jacobean England, the plebeians actually get what they want---after which the dramatic focus re-centers onto Martius, the war with the Volscians, and the political machinations of the tribunes.
This is possibly a case in which a dated supposition has, through overuse, become received wisdom.
According to Steve Hindle, however, ``depopulation and dearth were, it seems, intimately linked in plebeian consciousness''~\cite[27]{hindle_imagining_2008}.
Hindle also sees the play as valuable a source for historians as does Kishlansky: ``Whatever the play may or may not reveal about Shakespeare's own attitudes, his familiarity with the idioms associated with the political economy of grain supply suggest that \emph{Coriolanus} is no less valuable a source than the digger broadside, King James's proclamations, Wilkinson's Sermon or Bacon's essay for historians seeking to understand the dynamics of popular protest in the early seventeenth century.
The play not only echoes and rehearses the discourses associated with the Midland Rising, but also explores the parameters of what it was possible to think about hunger, about paternalism, about protest and about punishment in Jacobean England'' (47--48).
Hindle's case rests almost entirely on the first scene of the play, as arguments must that make of the entire play a commentary on popular rebellion.
Hindle's point, however, is not that the play is subversive of the contemporary political establishment but a rejoinder to both popular discontent and official corruption in favor of good governance: ``Indeed, it is arguable that 1.1 of \emph{Coriolanus} imagines not insurrection itself but the prevention of insurrection by the timely redress of grievances, a policy subsequently endorsed by Francis Bacon and conspicuous by its absence from the counsels of James I in 1607, at least until the crown's hand had been forced by civil commotion'' (49).}
By dating the play to this event, we might also consider it marking a shift in Shakespeare's political dramaturgy.
While it is plausible that the enclosure problem and the corn riots contributed ``to an attitude of receptivity for such a play''~\cite[127]{stirling_populace_1949}, there are a few problems with readings that go much beyond such vague correlations.
One is that, however much the constituency of Parliament was gaining power in the early seventeenth century, there is little evidence that the common people per se were ever considered by either Parliamentarians or Royalists to be anything other than the ignorant mob Shakespeare depicts them as.
An alternative political system may or may not have been in the offing, but it is being too hasty to conflate the nascent public sphere of urban elites with the ``commons.''
Another problem is that such a reading, as flattering as it is to our own sensibilities, requires Shakespeare to have been himself a proto- (or crypto?) republican.
If that were the case, we would have to ironize all of Martius's fulminating against the plebeians, which is not completely unappealing in its vitriolic intensity, and somehow account for both the genuinely labile nature of Shakespeare's citizenry as well as the duplicity of their representatives.
Whether it proves the validity of his point of view or only the consistency of Shakespeare's dramatic depiction of them, they bear out Martius's accusations entirely.
One might also point out that Martius is ultimately banished for his words, not his deeds, and by decree rather than any kind of genuine trial---hardly indicative of an emergent liberalism.\footnote{Andrew Hadfield, who has done the most work on the question of Shakespeare and republicanism, can only conclude murkily that Shakespeare was definitely interested in politics.
Though hardly a solution to the problem of his personal views---and it is an open question whether we can or should ever find one--this is still a bold statement in contrast to critics of long ago, who generally regarded him an apolitical author. See Andrew Hadfield. \emph{Shakespeare and Republicanism}. Cambridge: Cambridge University Press, 2005. Print.}
David George, rehearsing much of the evidence and historical work brought to bear on the dating and contextualization of \emph{Coriolanus}, thinks that the play's politics may have been more of a marketing ploy than anything else.
He corroborates the consensus position that Shakespeare directly references the Midlands Revolt but does not thereby impute to him, in doing so, a specific political motive.
He argues instead that the inclusion of such topical allusions was ``an artistic method of creating immediacy for his audience and of winning their involvement.
Simply, they would not have understood references to ancient Roman customs and practices''~\cite[72]{george_plutarch_2000} with the explanatory mechanism of contemporary analogies.
Clifford Ronan acknowledges the political potential of Roman plays in particular but concludes that any specific agendas that might be imputed to them were likely dissipated amidst the heterogenous audiences that witnessed their performances: ``Like history writing proper, a drama of history advances partisanship less than it encourages relativism and tragic ambivalence''~\cite[51]{ronan_antike_1995}.
Without party platforms on which to base political appeals and given the upper-class bias of most plays, according to Ronan, it seems more sensible to regard dramatic crises that happen to be political as simply one among many varieties of dramatic content that, at best, channeled subversive energies that might otherwise have sought more destructive outlets than visits to the playhouse.

Returning to my primary topic, what the political content and all this critical politicizing of \emph{Coriolanus} point to is, if not a disguised political agenda on the part of Shakespeare himself, the emergence in early modern England, as an accompaniment to a historical consciousness, of a concept of the public.
However one might relate the events of the play to the events of the day or enforce on the text ideologies from centuries in its future, it is undeniable that the play opens out onto a field of action in which private citizens combine their voices in a public entity.
The theory of the public that Shakespeare dramatizes is set at the very beginning of the play in opposition to the organic model of the state proposed by Menenius.
Whereas the public is a collective construct of free citizens negotiating their private boundaries, the body politic illustrated by the famous ``fable of the belly'' is a political fiction that mystifies political process---the \emph{arcanum imperii}\footnote{The source of the expression \emph{arcana imperii} is Tacitus, \emph{Annals} 2:36.
The histories of Tacitus were increasingly popular at the end of the sixteenth century for their realistic portrayal of power politics and imperial dissimulation.
For an extensive treatment of this subject, see Peter Samuel Donaldson. \emph{Machiavelli and Mystery of State}. New York: Cambridge University Press, 1988. Print. 111--140. \nocite{donaldson_machiavelli_1988}
For an interesting discussion of the concept's relevance to early modern English culture, particularly to magic, automata, and other wondrous devices that operated according to analogous principles of obfuscation, see Jessica Wolfe. \emph{Humanism, Machinery, and Renaissance Literature}. Cambridge: Cambridge University Press, 2004. Print. 63--68. \nocite{wolfe_humanism_2004}}
---and accommodates patrician prerogative behind the veil of a convenient parable.
Menenius would like to catechize the crowd with his doctrine, a dogma that invokes abstract relationships that have little bearing on its immediate, practical demands.
He appeals to them to blame ``the gods, not the patricians'' (1.1.68), as if the organization of Roman society were as eternal as the natural world and destined to continue on its course forever.
This vision of society is essentially apolitical, with no space for debate and no latitude for change.
Menenius may be anticipating the future continuance of Roman power, but his comparison of the state to the ineffability of heaven suggests that Rome exist in a permanent present.
The plebeian demand for participation in power is a demand by the commons to realize the fullness of their subjectivity as historical agents.
If history is made by people, not preordained by the heavens, \emph{Coriolanus} presents us with the founding myth of ``the people'' asserting their control of it.
Their insistence that ``the people are the city'' (3.1.200) counterposes Menenius's condescending paternalism as well as Martius's outright hostility.
According to Menenius, who speaks as though he were a neutral arbiter in this debate, the maintenance of class relationships is essential to the survival of the state and its inhabitants:
\begin{vq}
No public benefit which you receive\\
But it proceeds or comes from them to you,\\
And no way from yourselves.\\
\hfill(1.1.147--149)
\end{vq}
Though more indulgent than Martius's preference to suppress them, it is only a milder, more deceptive form of symbolic violence to exclude the plebeians from the genuinely public function they desire through mollifying propaganda.

What is most remarkable about Shakespeare's rendition of the fable of the belly, however, is that it singularly fails to convince.
Like Martius's refusal to display his wounds in the marketplace, this too is a Shakespearean innovation that is not hinted at in his sources.
The plebeians do not simply take Menenius at his word and disperse.
Instead, they argue with him.
They force him to defend his assertions.
They demonstrate through their very disputatiousness what a public dissemination of ideas would look like, totally thwarting his assumption in his own historically-granted powers of persuasion---historically-granted because he himself ``knows'' the story of Menenius Agrippa as well as we do.
One gets the sense that the plebeians have heard it before, too.
As is so often the case, Shakespeare deviates from the traditional reading of this episode.
Although humanist writers generally used Menenius's fable as a traditional example of the power of speech, \emph{Coriolanus} comes at perhaps too late a date to take it seriously.
Even Sir Philip Sidney, who is not exactly credulous about such matters, reports it as a ``notorious'' exemplum in his defense of poetry:
\begin{bq}
Infinite proofs of the strange effects of this poetical invention might be alleged; only two shall serve, which are so often remembered as I think all men know them.
The one of Menenius Agrippa, who, when the whole people of Rome had resolutely divided themselves from the senate, with apparent show of utter ruin, though he were (for that time) an excellent orator, came not among them upon trust of figurative speeches or cunning insinuations, and much less with far-fet maxims of philosophy, which (especially if they were Platonic) they must have learned geometry before they could well have conceived; but forsooth he behaves himself like a homely and familiar poet.
He telleth them a tale, that there was a time when all the parts of the body made a mutinous conspiracy against the belly, which they thought devoured the fruits of each other's labour; they concluded they would let so unprofitable a spender starve.
In the end, to be short (for the tale is notorious, and as notorious that it was a tale), with punishing the belly they plagued themselves.
This applied by him wrought such effect in the people, as I never read that only words brought forth but then so sudden and so good an alteration; for upon reasonable conditions a perfect reconcilement ensued.~\cite[41--42]{sidney_defence_1966}
\end{bq}
As it appears in Shakespeare, Menenius's famous tale is indeed rather like a poetic interpretation of history, but one that has passed through a number of filters before arriving in \emph{Coriolanus}: it initially presented itself to Shakespeare as an example from the classics of the power of rhetoric; it is presented to the plebeians in the play as a casuistic rationale; which includes its ironic presentation to Shakespeare's audience as a familiar example of the power of rhetoric; and finally, in its presentation in its original context, but anachronistically burdened by its subsequent career, it ends up a farce.
Rather than occupying the role of sage elder, Menenius gains the reputation in this play of sententious fool.
He is, however, exactly the sort who would glean conventional wisdom from history for application to the present moment, though he hardly manages himself to convince anyone of anything.
Has Shakespeare, by the early seventeenth century, tired of such maxims?
Or is it his own public that is no longer consoled by them and more entertained by a knowing subversion of them?
The contrast between the imagery evoked by Menenius's fable and the political reality that presses against it sets the tone for the play, and there is, throughout it, much cynical rebuttal of this sort of language, by Martius especially.
The intersection of political fictions with fictions of subjective autonomy in this play consistently demonstrate the inadequacy of both: ``The facts of Roman life on both sides, including his own character, give Menenius' parable the lie.
What we see is no organic body politic but a Rome torn by factional strife, Machiavelli's politics of the power struggle, reflected in the play's imagery of bodily fragmentation''~\cite[47]{parker_coriolanus_1994}.
The ``fable of the belly'' is no more tenable in an actual society than man being author of himself.
``Wholes'' in the political model suggested by \emph{Coriolanus}, whether composite social constructs or the self-consistent integrity of individuals are belied by the essential discontinuity of wills and identities that political life entails.
Martius, who fundamentally denies this reality, speaks more truthfully than he realizes when he mockingly calls the people ``fragments'' (1.1.217), and less self-consciously than he is aware when he impugns their mutability: his own will have far more dire consequences.
Menenius could not convince him with his fable, either, even if he considers himself Martius's own biographer,
\begin{vq}
The book of his good acts whence men have read\\
His fame unparalleled, haply amplified.\\
\hfill(5.2.16--17)
\end{vq}

Tested against political realities they seemed unable to account for, such classical platitudes were found wanting.
James Holstun suggests that the play reflects a genuine break made in the Jacobean period from the more self-assured concepts of social order up until then predominant: ``In this play, Shakespeare moved out of the Tudor conception of the body politic into the seventeenth-century critique of the body politic as an outmoded fiction''~\cite[492]{holstun_tragic_1983}.
Similarly, Jonathan Dollimore sees the play as indicative of a new kind of historical understanding which, focusing on state power, social conflict, and the struggle between true and false discourses, exposes the contradictions between traditional ideas and the present political circumstances that put them to the lie.
For Dollimore, \emph{Coriolanus} is a symptom of the decentering of the humanist subject in the early modern period, a result of the newly coherent class antagonisms that revealed the radically contingent nature of both identity and history.\footnote{Jonathan Dollimore. \emph{Radical Tragedy: Religion, Ideology, and Power in the Drama of Shakespeare and His Contemporaries}. Brighton: Harvester, 1984. Print. 218--230. \nocite{dollimore_radical_1984}} 
Although Shakespeare evinces in \emph{Coriolanus} much skepticism about traditional political fictions, he also tests the possibility of abandoning political fictions entirely and the consequences of trying to make such a futile escape.
\emph{Coriolanus}'s textually ironic skepticism toward its own status as a source of lessons from history is revealed most starkly when Martius is preparing to leave Rome and finds his mother forgetting the tough-minded wisdom she raised him to adopt:
\begin{vq}
Nay, mother,\\
Where is your ancient courage? You were used\\
To say extremities was the trier of spirits;\\
That common chances common men could bear;\\
That when the sea was calm all boats alike\\
Showed mastership in floating; fortune's blows,\\
When most struck home, being gentle wounded craves\\
A noble cunning.
You were used to load me\\
With precepts that would make invincible\\
The heart that conned them.\\
\hfill(4.1.1--11)
\end{vq}
We have already seen how Martius references the linguistic world he shares with his mother, and how it makes itself felt in the language he uses.
Here, he reminds her of what she taught him, and yet his nostalgia is mixed with detachment.
Given his relatively unsophisticated bitterness throughout most of the play, it would be difficult to read these lines as entirely without irony.
The vague universalism and grammatically clever construction of these apothegms come across as glib, even aggressive, when insensitively delivered by Martius to his mother.
But they are part of the ``text'' of Roman culture that he tears up when he banishes the city.
As with all the customs he disdains to follow, it is fitting that one who would be ``author of himself'' should have no need of \emph{historia magistra vitae}.

Martius fears being reduced to the banality of these humanist sayings, of himself becoming a part of them.
It would be tantamount, in his estimation, to being appropriated into public discourse, registering his life itself as a text in the common store of public history.
He reviles, that is, exactly the situation in which he finds himself---written into a history play, exposed to common spectators, and vulnerable to the interpretive appetites of readers.
His fate, to be literally torn apart---and then tritely commemorated---is emblematic of this fear.
His body is violated in the same way that texts are marked up, ripped, and recombined.
Neither remains integral.
A text, once public, undergoes constant transmutation, as pieces are removed and repurposed, original intent having little bearing on subsequent needs.
If it is one of history's primary functions to make ``public'' its own processes, to render the opaque in the past both visible and usable to the present, the people want to do exactly this: bring the unseen, unspoken mechanisms of power and statecraft into an arena in which they can be scrutinized and either validated or repudiated.
It is this public exposure, this surrendering to an uncontainable, hermeneutic gaze that Martius reviles.
He fears becoming exactly what he already is: an exemplary personage fictionalized by historiography and then fed to the masses as gustatory edification.
In a play so preoccupied with images of hunger, eating, and appetite, this is hardly an exaggeration.
As the third citizen in the ``selection'' scene vividly explains:
\begin{bq}
If he show us his wounds and tell us his deeds, we are to put our tongues into those wounds and speak for them.
So, if he tell us his noble deeds, we must must also tell him our noble acceptance of them.
Ingratitude is monstrous, and for the multitude to be ingrateful were to make a monster of the multitude, of the which we, being members, should bring ourselves to be monstrous members.~(2.3.5--12)
\end{bq}
This is a visceral and even disgusting image for what is meant to be a mundane political process, but it effectively evokes the horror that Martius feels about having his ``nothings monstered'' (2.2.75) by the ``many-headed'' perception of the public.
To let the plebeians put their ``tongues into those wounds and speak for them'' would put his wounds into public circulation, rather like a book that is passed from reader to reader, each one scribbling a different commentary in the margins.
They would no longer be marks written in a silent language legible only to Martius and his mother but would be made to signify, be given voice, on behalf of the state.
They would be brought, that is, into history.
Martius, however, prefers to remain in a private, unarticulated world of his own.
He regards his body as his private property, not a vehicle of the state, and his deeds as private actions.
When he finally ``banishes'' Rome, he imagines himself the master of his own destiny.
He thinks ``there is a world elsewhere'' (3.3.134) beyond the reach of the public and the grasp of its historical imagination.
His banishment of Rome is analogous to his denial of his name, to his refusal to let his wounds speak, and most of all to his inability to speak a civil language.

Martius's reputation for stubborn inarticulacy is a personality trait about which Plutarch gives a clue but is elaborated by Shakespeare into a defining characteristic.
In Sir Thomas North's translation of Plutarch ``he had an eloquent tongue''~\cite[5:543]{bullough_narrative_1957} and in Philemon Holland's Livy he is a judicious participant in civil affairs, ``right politicke of advise, active besides''~\cite[5:498]{bullough_narrative_1957}.
While his sources for the play agree in describing him as a well-spoken if irascible person, Shakespeare's Martius ``continually demonstrates his inadequacy as an orator and, in so doing, his inability to fulfill the primary social and civic duty of a Roman citizen''~\cite[185]{miola_shakespeares_1983}.
Unlike the Martius he gleaned from his sources, Shakespeare's cannot observe public forms or speak a public language.
This does not mean that he does not possess his own sort of eloquence, however.
Instead of depicting him as a particularly hot-headed political disputant, Shakespeare follows the suggestions in Livy and Plutarch as to Martius's verbal capacity in order to fill out his psychological composition along a different dimension.
He is withholding, but not silent, and the more he talks, the more he seems to be holding himself back.
In contrast to Cordelia, who cannot heave her heart into her mouth (\emph{King Lear} 1.1.91--92), and Bolingbroke, who is unable to breathe out ``the abundant dolour'' of his (\emph{Richard II} 1.3.256), Martius has no problem bridging the gap:
\begin{vq}
His heart's his mouth.\\
What his breast forges that his tongue must vent.\\
\hfill(3.1.259--260)
\end{vq}
Paul A. Cantor notes the relevancy to \emph{Coriolanus} of Aristotle's idea that man is a political animal and that, without a city, a person is either a beast or a god: ``Growth into true humanity requires the city because because it is contingent upon speech, which can only be developed through human association''~\cite[101]{cantor_shakespeares_1976}.
But Martius's inarticulacy is also reflected in the lack of depth with which he is characterized.
Unlike most of Shakespeare's protagonists, he is not given over to very much introspectiveness.
As Maurice Charney reminds us, ``Coriolanus has only thirty-six lines of soliloquy: the same number as \emph{As You Like It} and the fewest in the Shakespeare canon.
This does not prove anything by itself, but it keeps us aware of the lack of inwardness in the play, and the fact that Coriolanus is the least articulate of Shakespeare's tragic heroes.''~\cite[79]{charney_dramatic_1970}.
He neither wants to speak nor be spoken about, as if he rejects the world of words completely.
James Kuzner relates his aspersion of public identity directly to his martial obsession.
Engagement in warfare, for Martius, represents the exact opposite of engagement with the public.
Instead of having to self-identity, he chooses to self-abnegate:
``Battle turns the corporeal self inside out; it does what the public square does not, makes of him surfaces without depths, a being no longer clearly or only Martius (as he is then called) but undifferentiated''~\cite[190]{kuzner_unbuilding_2007}.
Being all surface also means being entirely in the present.
Martius has no use for the past and no concern for the future: he exists only to act in the here and now.
His denial of any kind of diachronic identity for himself reaches its head when Cominius reports his encounter with him outside the gates of Rome, the erstwhile hero returned as a threatening conquerer:
\begin{vq}
`Coriolanus'\\
He would not answer to, forbade all names.\\
He was a kind of nothing, titleless.\\
\hfill(5.1.11--13)
\end{vq}
His linguistic ineptitude notwithstanding, Martius has something of Richard II in him, a solipsistic character who thinks he can will through words his own world into being.
Whereas Richard is absorbed in the generative powers of his imagination, however, Martius uses his language like a weapon.
They both illustrate a particularly Shakespearean drive to create worlds out of language, to fill the gaps between ideas and reality with words.
Martius's tactic is not to proliferate meaning but to attack it.
His banishment of Rome and search for a ``world elsewhere'' exemplifies his determination to brutalize the world into a shape that fits his nature and silence all the voices that threaten to fragment it.
Finding this impracticable, his only resort in the end is to seek the death of the outside world, the political death of Rome, and ultimately the death of his own, arrant subjectivity.
But his death drive toward historical erasure is no more effective than Richard's self-chronicling.

As reared by Volumnia, Martius's public self is the creation of his private life, but he prefers to remain private to the point of denying the public entirely and even attempting to destroy it.
Volumnia herself, sent to make peace with him, has to warn Martius that his history hangs in the balance whether he succeeds in conquering Rome or not:
\begin{vq}
Thou knowst, great son,\\
The end of war's uncertain; but this certain,\\
That if thou conquer Rome, the benefit\\
Which thou shalt thereby reap is such a name\\
Whose repetition will be dogged with curses,\\
Whose chronicle thus writ: `The man was noble,\\
But with his last attempt he wiped it out,\\
Destroyed his country, and his name remains\\
To th'ensuing age abhorred.'\\
\hfill(5.3.140--148)
\end{vq}
His name, as she explains, is not his own.
History made him Coriolanus but can unmake him, too.
No name is not a choice.
Future ages will have their way with every word that makes up who he is.
This is one of the most historically ironic moments in the play.
We witness this exchange as part of a history in which the composition of that history is discussed.
Martius's reputation is already determined, if not by Shakespeare himself in the process of dramatizing him then by the historically-aware auditors and readers of his play, who have now to make up their own minds as to how the chronicle is ``writ.''
Self-authorship is not available to Martius, because he is authored by history already, and autonomy from history is impossible.
This is one aspect of the tragic composition of his character, his paradoxical drive to become socially unbound and textually unlimited:
``Only by severing his relationship to nature and abstracting himself from all claims of kinship, will he become absolutely autonomous''~\cite[110]{greenblatt_shakespeares_2010}.
Or so he thinks.
Autonomy of this kind, however, does not exist.
There is no organic state, as insincerely or naively posited by Menenius, to separate himself from.
And there is no extra-social realm to which Martius may retreat.
Extra-textual might be a better word for it, as this is a textual point, not a phenomenological one.
There is no practicable division between public and private selves in a text, only verbal reaching toward one, and this a paradox that is integral to Shakespeare's project.
Although Martius thinks it outside his integral identity to traffic with the people in the forum, he is even then already acting out the text that has already been written for him.
When he has performed his obligation satisfactorily, he puts off the robe of humility and reassumes his former mien, ``knowing myself again'' (2.3.145).
His conversation with the people, however, shows they know him well enough, too.
He is already, as a text---written over with battle scars---their possession whether he reveals them or not.
When he later threatens to destroy the city, it is this inescapable intersubjectivity that he really wants to annihilate:
``In pursuit of a totally autonomous self which is privately grounded in the public language of the state,'' he confronts the contradiction that his private self can only ever emerge from public discourse~\cite[229]{tennenhouse_coriolanus:_1986}.
He thinks that by destroying Rome, he will also destroy his public self; his inability to do so bespeaks the dependency on that public self that his private self retains.

As much as Martius tries to avoid incorporating himself into public discourse, he also resists being made a subject of discourse by others.
As I have been arguing, a central theme of the play is exactly the interaction between private self and public representation but also the inherently public quality of historical writing.
It is a conceit of the play that its primary character does not want to act and a conceit of Shakespeare's engagement with history that this character, though entirely textual, does not want to be written about or remembered. 
To Martius's frustration, he is almost constantly being ``historicized'' by everyone else.
This chronicling always occurs either in his absence or else with his explicit disapproval.
We encounter these scenes of history in the making as though the public fiction of ``Coriolanus,'' the poeticized history that will bear Martius's legacy, is being written before us.
Particularly galling to Martius is that his deeds are celebrated as if he were consciously serving a public purpose, rather than acting as the instrument of a spontaneous, destructive compulsion.
Killing his enemies is not so far for him from killing language itself.
When he rushes into the city of Corioles alone, he is all but given up for dead by his compatriots.
Titus Lartius offers the first of what will turn out to be a series of eulogies for him:
\begin{vq}
O, noble fellow,\\
Who sensibly out-dares his senseless sword\\
And, when it bows, stand'st up! Thou art left, Martius.\\
A carbuncle entire, as big as thou art,\\
Were not so rich a jewel. Thou wast a soldier\\
Even to Cato's wish, not fierce and terrible\\
Only in strokes, but with thy grim looks and\\
The thunder-like percussion of thy sounds\\
Thou mad'st thine enemies shake, as if the world\\
Were feverous and did tremble.\\
\hfill(1.4.57--65)
\end{vq}
Hardly has the play begun and Lartius already sets its pattern.
Martius has not yet adopted his eponymous surname but, presumed dead in his moment of triumph, he achieves here his apotheosis: more daring than his inanimate sword, more valuable to the state than a man-sized jewel, and the very image of the ideal Roman warrior according to Cato the Censor, the upholder of timeless Roman virtues who had not yet even been born.
Lartius makes of Martius an exemplum, a model of behavior so superhuman it is practically inhuman.
Martius's response to this, when he miraculously re-enters the scene, is typical: ``Sir, praise me not'' (1.5.16).
His objection is greater still when Cominius, his general, informs him that his honor does not belong to him but to the state.
Rejecting Martius's pleas to leave his heroics uncelebrated, Cominius makes it clear, in his good-natured rebuke, that the public is author of everyone:
\begin{vq}
You shall not be\\
The grave of your deserving; Rome must know\\
The value of her own. 'Twere a concealment\\
Worse than a theft, no less than a traducement,\\
To hide your doings and to silence that\\
Which, to the spire and top of praises vouched,\\
Would seem but modest. Therefore, I beseech you---\\
In sign of what you are, not to reward\\
What you have done---before our army hear me.\\
\hfill(1.9.19--27)
\end{vq}
Cominius explains to Martius that his valor is a commodity and must be priced as such, so that his comparative value as an asset, as a ``sign'' of Roman identity, will be understood by the Roman public.
So great is the need for Martius's achievement to be advertised and thus inscribed into Roman history, in fact, that Cominius regards it as treasonous to think otherwise.
There is no virtue, after all, without models of virtuous behavior, and Cominius is trafficking in what was for the early modern period one of the most valuable sources of models.
Martius, naturally, will hear none of it:
\begin{vq}
I have some wounds upon me, and they smart\\
To hear themselves remembered.\\
\hfill(1.9.28--29)
\end{vq}
But memory is entirely the point.
If his wounds are not given textual form, worries Cominius, they will mean nothing, and they will do nothing.
History, however, requires that they perform.
The state, in fact, requires them to be the public text of a history that will serve to establish the very Roman identity that at the beginning of Roman history did not yet exist but that Shakespeare's latter-day ``Romans'' already take for granted.
If Cominius regards Martius's wounds as traces inscribed on the body politic, however, Martius prefers that they stand mute, or invisible: writing without meaning or legibility.
Unable to appreciate Cominius's point---he is practically speaking a foreign language---Martius considers approbation flattery, not a tool of state, and rejects the acclaim of his peers as ``acclamations hyperbolical'' (1.9.50), which is, perhaps, true enough about a heroic Roman who probably never existed and whose life is therefore entirely hyperbole.
But Cominius will be appeased here no more than the citizens of Rome will be later, when Martius refuses to make his wounds legible to them.
He argues that Martius does too much damage to himself, as if impugning his own reputation is tantamount to suicide---he says he will bind him in manacles, if he must, until he will listen to reason.
Though he appears to accept the title ``Coriolanus'' cordially enough, the first thing he says after the assembled soldiers shout his new name is rather telling about his true regard for such ceremony: ``I will go wash'' (1.9.66).
And when he is welcomed back to Rome as a conquering hero with further shouts of his new name, he is equally reticent to acknowledge it:
\begin{vq}
No more of this, it does offend my heart.\\
Pray now no more.\\
\hfill(2.1.163--164)
\end{vq}

The most significant moment of eulogizing of Martius, of course, is the official ceremony at which Cominius relates his deeds in public.
The speech does little to propel the plot forward: it seems to exist only so Martius can walk out on it, so he can refuse ``to hear my nothings monstered'' (2.2.75).
This is not simply false modesty, I would argue.
Though the sense of ``monstered'', a likely Shakespearean coinage, might include ``demonstrated,'' it also evokes the image of the many-headed multitude Martius so despises.
And as he departs, Menenius addresses them directly as ``multiplying spawn'' of which a thousand are worth ``one good one'' (2.2.76--77).
This is an unusual scene.
Martius acts as though not hearing Cominius's speech, the public transcription of his service, will somehow nullify it.
As he will later refuse to display his wounds and allow them to be given a voice, he cannot brook hearing them be given voice in his presence.
And so he literally walks off the page.
If he could, one imagines he would walk out of history, too.
It is ironic that Cominius claims to lack the voice needed on this occasion, that
\begin{vq}
the deeds of Coriolanus\\
Should not be uttered feebly.\\
\hfill(2.2.80--81)
\end{vq}
This is ironic, because Martius does not want them uttered at all and feebly is exactly how he himself will proclaim his merits himself in the marketplace.
Manfred Pfister argues that Martius's ``self is primarily a public self, a self enacted in public, constantly aware of the image it is projecting and the effect it is having on the other, and even the disregard for his impact on others, his famous 'noble carelessness', is a carefully studied mise-en-scene''~\cite[42]{pfister_acting_2009}.
This, I think, is only accurate in the sense that Martius is a public property in spite of himself.
That his behavior is calculated, and not impulsive, grants him far more self-consciousness than the text allows.
He is not merely ``acting out'' impulsiveness, either, not as a deliberate strategy, at any rate.
That is, Martius's self \emph{is} a public self, but this is the very reality that he resists.
Volumnia might have taught Martius how to act the Roman, but for him it is not acting.
 On the battlefield, in his rivalry with Aufidius, he finds the sort of authentic experience he mistakes for reality, and this is the source of his distrust for the paltering speech of the public forum and the insincerity required for survival in political life.
In fact, the eulogizing of Martius finally ends only when the play does, in the most ironic mode, of all, with his once and future nemesis Aufidius declaring that ``he shall have a noble memory'' (5.6.155).
Of what, we might ask, will his noble memory be comprised?
Does such an enemy of the ``common,'' those who will inherit it, deserve to be remembered as ``noble''?
The most ``noble'' remembrance he might receive could only be the exact sort of antiseptic eulogy he despises.
But this is why he deigns to be remembered at all and why he distrusts the multifariousness of the public.
As for the people themselves, his distrust of them is perhaps justified by the mockery that Brutus and Sicinius make of his family history in order to persuade them to rescind their support of his nomination to the consulship.
In what is itself a scene of rhetorical persuasion, the tribunes persuade the plebeians to pretend they had been convinced against their better judgments by the tribunes's own epideictic cant about Martius's lineage and accomplishments:
\begin{vq}
Say we read lectures to you,\\
How youngly he began to serve his country,\\
How long continued, and what stock he springs of,\\
The noble house o'th'Martians, from whence came\\
That Ancus Martius, Numa's daughter's son,\\
Who after great Hostilius here was king,\\
Of the same house Publius and Quintus were,\\
That our best water brought by conduits hither;\\
And Censorinus that was so surnamed,\\
And nobly named so, twice being censor,\\
Was his grest ancestor.\\
\hfill(2.3.232--242)
\end{vq}
The tribunes thus manage to pit Martius's ``own desert'' (2.3.64) against him, demonstrating both the public context of identity and reputation and---his greatest fear---how little bearing such facts and deeds have on them when twisted by false interpretation or put in service of an ulterior agenda.
Subject to the inconsistent voices of the people, Martius's ``history'' can only ever be the product of political manipulation, contingent circumstance, and popular capriciousness.
That he is unable to reconcile his speech with the speech of others is what makes of this perhaps banal fact a tragic situation.
We should not, therefore, be surprised by how much he bridles at the way others use language, responding with exasperation to the plebeians's claims about senatorial policies---``They say!'' (1.1.185); to the tribunes's presumptuous commands---```Shall'?'' (3.1.92); to the way the nobles think he should acquit himself with the people---```mildly''' (3.3.146); to the multiple accusations of treason made against him---``How? `Traitor'?'' (3.3.66) and ```Traitor'? How now?'' (5.6.87); and to Aufidius's sudden insults at the end of the play---```Martius'?'' (5.6.89) and ```Boy'!'' (5.6.117).
These verbal echoes reflect not only his shock at how language is being used against him---an ironic position for such a weaponizer of language to be in---but also his incomprehension of the way public language functions.
The mutability and negotiation that animate public discourse is incompatible with someone like Martius, who wants to keep his words even as he speaks them.
In a way, he wants to speak without being heard or, rather, wants his words to take the shapes of action and not assume new shapes in the mouths of others.
He can hardly even bring himself to speak the formulas of public ceremony and mocks the suggestion that he utter platitudes ``in wholesome manner'' to the people:
\begin{vq}
What must I say?\\
`I pray, sir'? Plague upon't, I cannot bring\\
My tongue to such a pace. `Look, sir, my wounds!\\
I got them in my country's service, when\\
Some certain of your brethren roared and ran\\
From th'noise of our own drums.'\\
\hfill(2.3.48--53)
\end{vq}
Public speech to him is all insincerity, and he scoffs at its empty pretension.
It is also to him a kind of contamination, the mingling of a false ``Coriolanus'' with the true Martius.
The citizens come to offer him their ``own voices'' with their ``own tongues,'' to join with him in a public discourse that would combine their assent with his magnanimity, but he prefers to keep his tongue sealed behind clenched teeth.
One has to imagine an incredulous exclamation point after practically everything he says.

As all of this eulogizing, and his reaction to it, demonstrates, the formation of Martius's identity is consistently represented as in the hands of others.
In spite of his own temperament and protestations to the contrary, he is actually quite a passive character.
Outside of a military context, his bluster bespeaks his essentially powerless condition.
Unlike other Shakespearean tragic heroes, Martius is not a figure of authority.
He is not even the highest ranked general in the Roman army.
As much as he thunders his hatred of the plebeians, he is unable to act on it.
His position is essentially one of frustration.
Beholden to the law and to a social structure he can hardly tolerate, he is neither sanctioned to vent his anger in any meaningful action nor does he take the initiative to deliberately violate law or custom.
He is proud but also petulant.
All he can do is fulminate against the commons and hope for new wars to grant him release from the social norms that hold him in check.
His entrance scene, in which he famously assaults us with his argument against democracy is utter bombast, more Gorgias than Plato and less persuasive.
By the time he arrives to meet Menenius in mid-appeasement, the demands of the people have already been met and their tribunes appointed.
His language is not a preamble to action but the snipings of impotent rage:
\begin{vq}
Would the nobility lay aside their ruth\\
And let me use my sword, I'd make a quarry\\
With thousands of these quartered slaves as high\\
As I could pitch my lance.\\
\hfill(1.1.192--195)
\end{vq}
Not only is this a blatant example of what Janet Adelman refers to as Martius's ``phallic exhibitionism''~\cite[132]{adelman_angers_1980}, it is also indicative of his manipulable and instrumental nature: he would if he could but he is not allowed.
Likewise, in his finest hour at the battle of Corioles, he is not at the front of the action but must wait for news of Cominius's parley with the enemy.
And when the people turn on him, he pointlessly claims that ``on fair ground I could beat forty of them'' (3.1.244).
Menenius's corroboration that he, too, could tackle two of their best only drives home how utterly powerless Martius is and untenable his rage.
He is as impotent as an old man long past his fighting years and entirely without an object (including Aufidius) adequate to his anger.
As Cominius usefully points out to both of them,
\begin{vq}
Manhood is called foolery when it stands\\
Against a falling fabric.\\
\hfill(3.1.248--249)
\end{vq}

The most salient indication of Martius's passivity is his relationship with his mother.
The product of her careful rearing, Martius is fundamentally incapable of imagining an autonomous identity for himself, however much he tries.
His self-authorship is itself a maternally derived construct.
If his way of ``writing'' his reputation is to be wounded in battle, and to reserve the display of those wounds explicitly for his mother's gaze, it is a self-authorship of a profoundly nihilistic variety.
At the same time that he renounces his signifying in society he also repudiates the claim to autonomy it is meant to manifest: his ``wound-writing'' is a form of invention by means of subtraction, denying his pretension to self-authorship while mortifying his flesh.
To believe, in other words, that one's identity has a solid foundation that is in no way subject to the shaping mechanisms of upbringing or social interaction is to deny identity itself, to subtract from it until there is nothing left.
Martius is compelled, by this sort of subjective death drive, to strip away everything from his own identity that is in any way contrived, or artificial, or dependent on others: to make of identity, in its structure and consistency, an insubstantial, unreadable void.
His rationale, if we suppose that he has one, is if it does not exist in language, it cannot be spoken of.
Coppélia Kahn points out how Martius's ``self-canceling''~\cite[152]{kahn_roman_1997} identity is built on a series of such negations, that he defines himself against precisely the sort of artificial nature that formed him ``if both the warrior's ferocity and the politician's `insinuating nods' are the man's part, and he learns them both from a woman who thereby serves as his cultural father'' (155).
Volumnia, in contrast to her son, regards the wounds positively, as additive: the emoluments of his soldier's office, earned in service to the state and exchangeable in the political marketplace for higher office.
She counts them up like currency and ascribes to them a value that can only exist in a social context.
Her attitude toward Martius's wounds resembles her attitude to his reputation.
If wounds can be converted into the dissimulation of reputation, then the dissimulation of political performance can be converted into power---as long as Martius can smooth ``his rougher accents'' (3.3.54) into ``gentle words'' (3.2.60).
She appeals to him to set aside his warrior's temperament and assume a more flexible aspect, to think, as she does, that his wounds were not earned for their own sake but for the purpose of social action:
\begin{vq}
I prithee now, my son,\\
Go to them, with this bonnet in thy hand,\\
And thus far having stretched it---here be with them---\\
Thy knee bussing the stones---for in such business\\
Action is eloquence and the eyes of th'ignorant\\
More learned than the ears.\\
\hfill(3.2.73--78)
\end{vq}
This is rather too much for Martius, for whom action is action, not eloquence, and it would also seem to contradict Volumnia's otherwise martial fixations elsewhere.
Interestingly, Volumnia's strategy represents the people as, perhaps, less canny about appearances than they may be.
In the remarkable conversation between the officers at the Capitol, Shakespeare suggests that the people understand politics very well indeed.
One of the officers even praises Martius for being honest to a fault about his awareness of the people's fickle disposition and dismisses as irrelevant to his achievements his course personality: ``He hath deserved worthily of his country, and his ascent is not by such easy degrees as those who, having been supple and courteous to the people, bonneted, without any further deed to have them at all into their estimation and report'' (2.2.23--27).
At least some of the people would be unmoved by a politician appearing, bonnet in hand, to solicit their support.
What all this theorizing about political posturing points up is a play concerned with the fashioning of appearances but also the impossibility of ever reigning in that fashioning and disciplining it to a singular purpose.
Volumnia, in her paradoxical role as both the maternal artist of Martius's identity and the ``cultural father'' who represents essential truth, wants to orchestrate Martius's reputation.
She wants him, that is, to be a model of unassailable Roman virtues but at the same time to contradict those virtues by letting himself be manipulated so that he can learn how to manipulate others.
The tribunes, of course, have their own ideas about manipulating him, and the end result is that the last person to have any power over Martius's public identity is Martius himself.

Martius's passivity, or what we might term his aggressive-passiveness, is congruent with his role as a product of the historical imagination he would rather not be a part of.
If the play demonstrates that history's grasp is inescapable, its tragic inflection is Martius's defiance of what it dramatizes, in a variation of the fatalistic tragic mode, as inexorable fact.
He is not the product of his own autonomous spontaneity any more than he has any ability to control how others perceive him and construct representations of him that have political force.
Even before we meet him in the play, the essential truth of the malleability of historical reputations is contained in a casual remark by one of the rioting citizens in the first scene in response to another's defense of Martius's patriotic service: ``he did it to please his mother'' (1.1.35).
This is, of course, his literary reputation, as well, but we should bear in mind that it is a reputation constructed in the play, and constructed in advance, and also constructed within a dramatic examination of the very process of its constructedness.
What is left is passed along to us to accept or deny as citizen-readers ourselves.
Though it seems a throwaway remark, it hits too close to the mark, and foreshadows too much about the play's plot, to dismiss entirely.
While it is true that Martius wishes to please his mother, it is really the mother as imaginary linguistic origin that seduces him, the author of the text to which he constantly returns.
As if fulfilling a prewritten outcome, it is also the text that gets him killed.

One further question worth addressing is what relationship Martius's self-abnegating rejection of self-representation has to his vitriolic hatred of the people.
What, that is, do they represent to him outside any pat system of class antagonism we might perceive in the play as a whole?
Why does he hate them?
A clue is given in the first scene, discussed in part above, when the plebeians's demand that the patricians ``yield us but the superfluity'' (1.1.15) is echoed later in the same scene by Martius, on hearing the news that Volscians are again in arms:
\begin{vq}
I am glad on't.\\
Then we shall ha' means to vent\\
Our musty superfluity.\\
\hfill(1.1.220--221)
\end{vq}
For someone so otherwise obsessed with fighting, it is surely as significant that this is his first reaction, even before any mention of his delight in battle, as it is that he uses the same word, ``superfluity,'' to describe the people's relationship to the city that the people used to describe the patricians's exploitation of it.
They are, to him, excessive, unnecessary, a surplus.
Their wavering opinions represent a surplus of interpretive possibility, too, an uncontainable abundance of voices that Martius's discipline cannot hope to muster.
At every turn, Martius's hatred of the people is predicated on this unpredictability and unreliability.
They are unpredictable and unreliable as members of the polity but also as the bearers of his legacy.
They have, for him, too many voices, all crying out at once, and to purposes that are too often crossed.
In their diversity, they represent the heterogeneity of interpretation, the defining quality of historical reception itself, which threatens his private impetus to remain self-contained and unread.
To expose his wounds to public scrutiny, as it were, would be to give them a false signification.
They would be no longer his bodily property but the common property of Rome and all future generations who might desire to put their tongues into them, as well.
Faced with this prospect, he would rather his wounds remain unread and unreadable and, as for his reputation, he makes his feelings on that subject as explicit as we might expect:
\begin{vq}
Think upon me? Hang 'em!\\
I would they would forget me, like the virtues\\
Which our divines lose by 'em.\\
\hfill(2.3.56--58)
\end{vq}
If to know is to possess, he is less interested in what the people think about him or what influence he has over them than he is that they not know him at all.
If the people are worried that Martius
\begin{vq}
would depopulate the city, and\\
Be every man himself.\\
\hfill(3.1.266--267)
\end{vq}
Martius would prefer only knowing himself to the exclusion of anyone else knowing him or having a purchase on his identity at all.
Not even the nobles can draw him from his single-minded epistemology.
They desperately want him to know what he is doing when he provokes the people into banishing him.
With a familiar Ricardian pun, however, Martius responds, ``I'll know no further'' (3.3.86), expressing both the limit he has placed on his understanding and a stubborn insistence (i.e.
``I'll know `no' further'')  about the extent to which he will pursue negation.
Banishing the city from his consciousness, a variety of the pre-mirror stage fallacy of ``if I can't see you, you can't see me,'' he vows to remain in exile ``what is like me formerly'' (4.1.53), a nameless, illegible singularity outside the boundaries of historical perception.
If the people to him are like a threatening monster, from this moment he seems akin to a monster himself, ``like a lonely dragon'' (4.1.30).
Cityless, and therefore even less human than the plebeians, he reverts to the most debased animalistic state of all.
The psychological dimension to banishment is clearly something that interested Shakespeare.
Martius's ``I banish you'' (3.3.122) is not far from Gaunt's advice to Bolingbroke,
\begin{vq}
Think not the King did banish thee\\
But thou the King.\\
\hfill(\emph{Richard II} 1.3.279--280)
\end{vq}
and Kent's mental accommodation to his own dismissal by Lear: ``Freedom lives hence and banishment is here'' (\emph{King Lear} 1.1.182).
In \emph{Coriolanus}, however, and paradoxically, the social negation of banishment has a positive, empowering property for Martius.
Denying his citizenship, like denying his name, is a means to prevent others from knowing him, just as refusing to reveal his wounds---``I will not seal your knowledge with showing them'' (2.3.106)---is a withholding of his body from epistemological assimilation by the many.
As Cynthia Marshall puts it, ``the question of exposing his wounds is less one of compliance to ceremony or conformity to public expectation than of allowed knowledge, of the extent to which his wounds cease to be his own and become more generally available---whether, as he puts it, they 'shall be yours'''~\cite[102]{marshall_wound-man:_1996}.

Threatened by a superfluity of individual voices all with the power to read his wounds and decide what they mean, Martius's only recourse is to retreat into a solipsistic idiocy.
He may spit and bluster about using his sword to ``make a quarry'' of the plebeians as individuals, but what he fears is the people in the abstract, the unkillable image of their undifferentiated, compounded beastliness.
In depicting the people, Shakespeare adopts a familiar Renaissance metaphor.
They are consistently reckoned, even by themselves, as ``a many-headed multitude'' (2.3.15); ``the mutable, rank-scented meinie'' (3.1.68); ``Hydra'' (3.1.94); ``the multitudinous tongue'' (3.1.157); a ``common cry of curs'' (3.3.119); and the tribunes of the people, ``the tongues o'th'common mouth'' (3.1.22).
The image of the people as a many-headed monster was deployed with great frequency in the early modern period and is ubiquitous across all genres of writing.
Despite critical attempts to locate Shakespeare's soft spot for the common man, there does not seem to be anyone contemporary with him who writes with very much sympathy for this monstrous, metaphorical aggregate.
Even the arguments putatively presented on the people's behalf, such as we find in Book V of Spenser's \emph{Fairie Queene}, are presented only for the sake of refuting them.\footnote{This is the episode of the leveling giant, \emph{The Faerie Queene} 5.2.51--52. All Spenser citations are from Edmund Spenser. \emph{The Faerie Queene}. Rev. 2nd ed. Ed. A. C. Hamilton, Hiroshi Yamashita, \& Toshiyuki Suzuki. Harlow, England: Pearson Longman, 2007. Print. Longman Annotated English Poets. \nocite{spenser_faerie_2007}
During the Civil War, a Royalist pamphleteer actually reprinted it as anti-Leveller propaganda, demonstrating the power of such literary images on the political landscape of the time.
For text and commentary, see John N. King. ``The Faerie Leveller: A 1648 Royalist Reading of The Faerie Queene, V.ii.29--54.'' Huntington Library Quarterly: A Journal for the History and Interpretation of English and American Civilization 48.3 (1985): 297–308. Print. \nocite{king_faerie_1985}} 
And in Book VI of that work, the slanderous power of the mob is captured in that most unusual of Spenserian creations: the Blatant Beast.
Like the multitude in \emph{Coriolanus}, the Blatant Beast is many-tongued, unpredictable, and speaks through wounds.
As the object of \emph{The Faerie Queene}'s final quest, it uniquely evades permanent capture, too, and seems to persist as much outside the boundaries of the poetic work as outside the boundaries of any effective law.
Its brief appearance at the end of Book V exemplifies the futility of trying to curb its libelous intent or capture it within the margins of a text.
Just as Martius cannot pretend the historical imagination of the people does not exist, the Blatant Beast ravages the public text whether Artegall, the hero knight of justice chases it or not: ``Unaligned, highly generalized, the creature and its rages are at this point placed beyond the interest, as well as beyond the apparent control, of the patron of Law and Justice: Artegall actively tries not to regard it, holding Talus back from attack, and the thing is left behind as the hero returns to Gloriana's court''~\cite[105]{gross_reflections_1999}.
The beast, as a representation of the many-headed multitude, undoes Spenser's work from the inside, just as the people in \emph{Coriolanus} are the undoing of Martius's own extra-textual reputation.
The metaphor of the many-headed beast evokes less a politically subjected underclass than the uncontainable social energy of a pluralistic population.
Sympathy for such a bedeviller, no more than sympathy for the proliferation of anonymous verse libels in the period, was not forthcoming from the poets.\footnote{For an overview of the recent rediscovery of the Stuart verse libel, see Alastair Bellany. ``Railing Rhymes Revisited: Libels, Scandals, and Early Stuart Politics.'' History Compass 5.4 (2007): 1136–1179. Print. \nocite{bellany_railing_2007}} 
The Parliamentarians in the 1640s, Christopher Hill points out, were willing to at least manipulate the beast to serve its own ends, but it is doubtful that anyone would have countenanced such an outrage in prior decades.\footnote{Christopher Hill. \emph{Change and Continuity in Seventeenth-Century England}. Rev. ed. New Haven: Yale University Press, 1991. Print. \nocite{hill_change_1991}} 
Unfortunately for our own democratic sensibilities, Shakespeare was not likely to have been an exception.
Indeed, the multitude in \emph{Coriolanus} is as formulaic is any other: inconstant, mercurial, and easily cowed into submission.
The point of introducing the mob into his dramatic design was probably less to inspire audience identification with it than to pose, at best neutrally, the problem that Martius faces.
As a public figure, he is subject to public whims and public scrutiny.
His history will be written outside of his control, read and interpreted by the multitudinous readers of early modern Europe by whom he would prefer to be forgotten.
Shakespeare's fellow dramatists, whatever their politics, were no more sympathetic to the many-headed beast, either.
This should not be surprising since playwrights often regarded themselves among its principal victims.

A clear intellectual current runs through much of the dramatic and non-dramatic literature treating the subject.
Thomas Middleton has the Arthurian figure Vortiger excoriate the beast in his own historical romance, \emph{The Mayor of Quinborough}:
\begin{vq}
Will that wide throated Beast, the multitude,\\
Never leave bellowing? Courtiers are ill\\
Advised when they first make such Monsters.\\
How neer was I to a Scepter and a Crown?\\
Were casting glory, till this forked Rabble\\
With their infectious Acclamations\\
Poyson'd my Fortunes for Constantines sons.\\
\hfill\cite[5--6]{middleton_mayor_1661}
\end{vq}
Thomas Dekker is likewise hostile in both drama and prose, his frequency of attack more than can be chalked up to unserious ideas and unsympathetic characterizations.
Moreover, the language he uses to describe the beast resembles that of Shakespeare's in \emph{Coriolanus}.
For example, in \emph{The pleasant comedie of old Fortunatus}:
\begin{vq}
I scorn'd to crowd among the muddie throng\\
Of the rancke multitude, whose thickned breath,\\
Like to condensed Fogs doe choake that beautie,\\
Which els would dwell in every kingdomes chéeke.\\
\hfill\cite[sig. E1v]{dekker_pleasant_1600}
\end{vq}
In \emph{The pleasant comedie of patient Grisill}:
\begin{vq}
These two are they, at whose birthes envies tongue,\\
Darted envenom'd stings, these are the fruite\\
Of this most vertuous tree, that multitude,\\
That many headed beastes.\\
\hfill\cite[sig. L1r]{dekker_pleasant_1603}
\end{vq}
In \emph{Lusts dominion}:
\begin{vq}
I have perfum'd the rankness of their breath,\\
And by the magick of true eloquence,\\
Transform'd this many headed Cerberus,\\
This py'd Camelion, this beast multitude,\\
Whose power consists in number, pride in threats;\\
Yet melt like snow when Majestie shines forth\\
This heap of fools, who crowding in huge swarms,\\
Stood at our Court gates like a heap of dung,\\
Reeking and shouting out contagious breath\\
of power to poison all the elements.\\
\hfill\cite[sig. E3r]{dekker_lusts_1657}
\end{vq}
And in his pamphlet, \emph{The Belman of London}:
\begin{bq}
Who would not rather sit at the foote of a hill tending a flock of sheepe, then at the helme of Authoritie controuling the stubborn and unruly multitude? Better it is in the solitarie woods, and in the wilde fieldes, to be a man among beastes then in the midest of a peopled Citie, to be a beast among men.
In the homely village art thou more safe, then in a fortified Castle: the stings of Envy, or the Bullets of Treason, are never shot through those thin walles: Sound healths are drunke out of the wholsome woodden dish, when the cup of golde boyles over with poyson.~\cite[sig. B3r]{dekker_belman_1608}
\end{bq}
Samuel Daniel extends the trope into an epic simile:
\begin{vq}
Like when some mastiffe whelpe disposd to play,\\
A whole confused heard of beests doth chase,\\
Which with one vile consent runne all away,\\
If any hardier then the rest in place.\\
But turne the head that idle feare to stay,\\
Backe strait the daunted chacer turnes his face:\\
And all the rest with bold example led,\\
As fast runne on him as before they fled.\\
So with this bold opposer rushes on\\
This many headed monster multitude.\\
\hfill\cite[464--465]{daniel_multitude_1600}
\end{vq}
And Michael Drayton also employs it in a historical tragedy:
\begin{vq}
This monster now, this many-headed beast,\\
The people, more unconstant then the wind,\\
Who in my life, my life did so detest,\\
Now in my death, are of another mind:\\
And with the fountains from their teareful eyes,\\
Doe honor to my latest obsequies.\\
\hfill\cite[sig. K2v]{drayton_peirs_1594}
\end{vq}
Political theorists of the time were no less indisposed toward the multitude, which they regarded as a genuine threat to the peace and security of the state.
Sir William Corne-Waleys, sounding much like a cooler-headed Coriolanus himself, warns against popularity as a basis for reputation:
\begin{bq}
For with danger they stand that stand not upon themselues---his [the popular man's] foundation is the many headed multitude, a foundation both in respect of their number and nature uncertaine, and consequently dangerous, for who knowes not the divers formes of mens imaginations, as different almost as their faces, which showes them easily seperated, \& their forces being strong, no longer then whiles together incorporated, being so subiect to be severed, nay they going against nature, if holding a continued union, what can issue from this confidence, but danger? their natures, but by the pleasure of nature and their education is left ignorant, which impotencie leaves a wavering disposition easily seduced, and as easily reformed, apt to beleeve a fayre tale, and as apt to beleeve weake reasons, strong: spent in contradiction, this makes them inconstant, for their discourse not used to retaine things, makes them like any thing, because they are destitute of the use of comparison.~\cite[R5r--R5v]{corne-waleys_essayes_1600}
\end{bq}
The dramatist and translator Anthony Munday, writing anonymously against the Catholic League's attempt to overthrow Henry III in France, mentions Coriolanus by name as a victim of popular caprice in times of crisis:
\begin{bq}
In such civill divisions, the mishaps are so great, that without consideration of good turnes and benefites received, or the vertuous actions of excellent men: the people so furiously cast themselves upon them, as they cease not to pursue them, even to death or banishment.
As it happened in Athens to Themistocles, Aristides, Demosthenes, and Phocion: in Rome, to Coriolanus, Camillus, Scipio Affricanus, Cicero and others…

And in the ende, by unbridled libertie, in many places and Citties where the Rebels are, you shall beholde not any \emph{Democratia}, or populer estate, wel \& pollitiquely governed by the Lawes, but rather a most miserable \emph{Olocratia}, an insolent domination of the multitude, or rather a many headed \emph{Anarchia}, the oppression whereof is most horrible and pernitious.
For you knowe that the people either serve humbly, or commaunde imperiously, and tasting a little of the bayte of libertie, exemption of taskes, subsidies and charges: in furie they reject and throwe off the yoke of obedience to the King, Superiours and Magistrates, themselves weilding and managing the highest authoritie.

Then pretending an equalitie, they practise nothing els but seditions, mallice, robberies, spoyles, insolencies, and destructions: whereupon Plato thus spake very notably.
\emph{The whole Common-wealth shall decay and perrish, when it is to be governed by Brasse or yron, that is to say, by foolish men, such as are borne rather to serve and obey, then to rule and commaunde}.~(Munday and L. T. A. sig. Q1v–Q2r)\nocite{munday_falshood_1605}
\end{bq}
Given the intense interest of Protestant England in the French wars of religion, it is clear that fear of the many-headed monster was no joke to Shakespeare and his contemporaries.\footnote{For the Catholic League's subversive activities, which only ceased with the Protestant Henry IV's conversion to Catholicism, see Mack Holt. \emph{The French Wars of Religion, 1562-1629}. Cambridge: Cambridge University Press, 121--152. \nocite{holt_french_1995}} 
Machiavelli, in his commentary on Livy, also cites the example of Coriolanus.
In his case, the mob must be contained at all costs, and Martius turns out to have in his political theory the same sacrificial status that many critics ascribe to him in Shakespeare.\footnote{According to Brockbank, for example, \emph{Coriolanus} has a sacrificial design that marks the end of the time when \emph{virtus} could accommodate valiantness: ``He dies for a city that has made too much of the cult of the warrior but failed to recognize the nature of its dependence upon that cult.
His death is deserved as a climax and consummation of a life which exhausts the possibilities of a mode of virtue, and as a punishment because such a life cannot be reconciled with the larger and more vulnerable claims of human community; that community itself is purged, chastened, shamed, and renewed'' (66).
But see also Nicole E. Miller, who claims that the sacrifice is never fully absorbed into the symbolic system of the city:
\begin{bq}
Those who read Coriolanus's death as sacrificial---either effective or failed---posit him as something like Hegel's `world-historical individual,' whose demise brings about a new order.
In this reading, grace assumes concrete, political form.
Yet, I would argue, in the end we are not to be allowed this palliative: what \emph{Coriolanus} represents, finally, remains a suspended grace, a token withheld, a sign never fully either inscribed or understood, even as we are called upon to ``assist'' in his remembrance.~\cite[297]{miller_sacred_2009}
\end{bq}} 
The story of Coriolanus demonstrates for him the utility of offering ``an outlet by which the masses can discharge the anger they have formed against a single citizen,'' without which, mob violence could destroy the republic:
\begin{bq}
And as to corroborating this opinion with examples, I think this of Coriolanus is enough from the ancients.
Each one may observe from it how much ill for the Roman republic would have resulted if he had been killed by a mob, because thereby injury would have been done by individuals to individuals.
Such injury produces fear; fear seeks for defense; for defense partisans are obtained; from partisans rise parties in states; from parties their ruin.
But since the affair was managed by one who had authority over it, all those ills were avoided that might have arisen if it had been managed with private power.~\cite[212--213]{machiavelli_discourses_1989}
\end{bq}
The opinion that the people represent a potential threat to the state was shared by Machiavellians and anti-Machiavellians alike.
The Jesuit thinker Giovanni Botero, alarmed by the popularity of Machiavelli, published \emph{Della ragion di stato} in  response in 1589.
But he nevertheless agrees with Machiavelli on the necessity of keeping a curb on the commoners and suggests using the sort of political fictions employed by Menenius or else purging the superfluity with wars, as recommended by Martius:
\begin{bq}
Just as a doctor can relieve the disordered humours of the human body by diverting them elsewhere with cauterizing and blood-letting, so a wise prince can placate an enraged people by leading it to war against an external enemy, or by some other means which will turn it from its original evil intention.
As Horace says, the populace is bellua multorum capitum, and when it is troublesome it must be taken now by one of these heads and now by another; it requires most careful management, and the hand, the rod, the curb and the halter may all be needed in turn.
The great need here is for a fertile imagination, capable of thinking up expedients to inspire in the populace feelings in turn of pleasure, fear, suspicion and hope, so that they can be held in check and then reduced to obedience: those men are best fitted for this work who possess the affection of the rebels as well as the gifts of sagacity and eloquence.
Agrippa pacified the people of Rome by telling them the famous fable of the human body and its members.~\cite[112--113]{botero_reason_1956}
\end{bq}
Sir Thomas Browne also agrees, arguing about the multitude that ``their reason cannot rectifie them, and therefore hopelesly continuing in mistakes, they live and dye in their absurdities; passing their dayes in perverted apprehensions, and conceptions of the world, derogatory unto God, and the wisdome of his creation''~\cite[14]{browne_sir_1981}.
Browne says the people are better led by example than precept, proverbs than logical demonstrations.
He also says they are concerned only with surface meanings, what is most immediately apparent and that, unable to attain the ``second intention of the words, they are faine to omit their superconsequencies, coherencies, figures, or tropologies, and are not sometime perswaded by fire beyond their literalities''~(16).
Menenius would be quick to agree.
Both he and Browne recognize, as Shakespeare dramatizes, the dangerous capacity of the crowd to convert hastily adopted opinions into ill-conceived violence:
\begin{bq}
Their individuall imperfections being great, they are moreover enlarged by their aggregation, and being erroneous in their single numbers, once hudled together, they will be errour it selfe; for being a confusion of knaves and fooles, and a farraginous concurrence of all conditions, tempers, sex, and ages, it is but naturall if their determinations be monstrous, and many wayes inconsistent with truth.~(17)
\end{bq}
That many of the plebeians in \emph{Coriolanus} the play do have ``tongues'' of their own perhaps suggests that Shakespeare does indeed have a more complex notion of the multitude than most of his contemporaries.
As we saw with Menenius's fable of the belly, the sort of political fictions recommended by these commentators are subjected in \emph{Coriolanus} to withering skepticism.
If putting his most famous speech about the necessity of ``degree'' in social order in the mouth of the wily Ulysses is a subtle way of undermining it,\footnote{\emph{Troilus and Cressida} 1.3.85ff.} the fictitiousness of ``the body politic'' is, as E. A. J. Honigmann argues, brought right out into the open by Menenius's interaction with its actual, speaking members: ``The fable begins to break down as soon as the body's members are humanised beyond the conventional Aesopian minimum (attributing mere speech to animals and so on), for it then comes close to conceding that social roles are interchangeable, that one can do the work of another''~\cite[179]{honigmann_shakespeare:_2002}.
Adelman goes even further.
She makes a compelling case that we, as spectators, are incorporated by Martius into the many-headed multitude he despises:
\begin{bq}
Coriolanus seems to find our love as irrelevant, as positively demeaning, as theirs; in refusing to show the people his wounds, he is at the same time refusing to show them to us.
In refusing to show himself to us, in considering us a many-headed multitude to whose applause he is wholly indifferent, Coriolanus denies us our proper role as spectators to his tragedy.~\cite[144]{adelman_angers_1980}
\end{bq}
But the exposure of these fables for what they are does not mean Shakespeare is on the side of the people.
If anything, it means the necessity of restraining the interpretive activity of the people is both necessary and impossible at the same time.
The tension between the state's reliance on political fictions and the people's reluctance to believe them can never be fully dissipated.
Martius, who refuses to play this game, ends up inscribed into an entirely different sort of story than the one he perhaps envisioned for himself.
Our spectatorship at a vast historical remove figures exactly the sort of retrospective interpretive multiplicity he rejects.
We may feel in league with the people, because they are, in some sense, our surrogates, but Martius is victimized by a power from which no one is safe: the free-floating power of multiple imaginations all shaping history according to their own best ends.

Martius ends up absorbed into Roman legend as a founding hero but one whose anti-social tendencies threatened the social stability his legendary exploits guaranteed.
Rome could not have existed without him, but it also could not have continued to exist with him.
The imaginary Rome of English antiquarian nostalgia was summoned into being by this sort of historical contradiction.
But the ideal is always recovered from the contingencies that threaten it.
Early modern history was often shaped by the poetic desire of the present to recover an idealized past and realize the ideal again in the future.
But history also requires the sacrifice of private selves to the public narrative, and history provides no safeguards against the multiplicity of public voices.
Reading history, inasmuch as performing history, is the act of revitalizing all the dead flesh that made the circumstances of its reading possible.
As \emph{Coriolanus} shows, all history is to some degree a history of wounding, of subtracting from the present for the sake of the future.
Such a play reanimates the wounded bodies whose negation made history a positive possibility.
The death of Caius Martius Coriolanus, who thought he could remain so autonomous as to remove himself from history entirely, ironically strikes a hopeful note for the future at the beginning of Roman memory of the past.
We are here in the literary terrain of \emph{King Lear}, another play set at the dawn of historical memory.
While Lear's death presages the disintegration of a society that depends on the royal succession, on a history that must continue because it is pre-written, but cannot because it has been ruptured, Martius is sent to the slaughter as the fulfillment of a destiny, Rome's destiny, that he chooses to take no part in.
Lear thinks he can see his kingdom in his mind's eye.
He believes that nothing can escape his gaze, that, contrary to the common phrase, his map is the territory.
Martius, the quintessential Roman, defines himself by projecting his imagination outside Rome as a historical exile who carries his society with him.
Roman history will go on, however, in spite of him, and Martius, who wants to go unmarked by history is marked anyway.
Lear, on the other hand, can only exist in society if he is marked as king.
Unlike the fulsomely social world of \emph{Coriolanus} with its teeming hordes of citizen-historians, Lear wanders in a landscape that seems to have no society at all, as if it has been retroactively cleansed of anyone who could grant him the social significance he craves and that Martius, for whom it is available, spurns.
Also unlike \emph{Coriolanus}, which has all of Roman history ahead of it, \emph{King Lear} abruptly negates the British history that properly lies before it.

The final word on the status of history at the end of \emph{Coriolanus} is perhaps as banal as it is deflating of Martius's tragic denouement.
With the Volscian senators gathered over the corpse of their erstwhile commander, agog at what has befallen, one of them makes what is perhaps the most insipid, indecorous, and yet apropos remark in all of Shakespeare: ``Let's make the best of it'' (5.6.148).
There could be no better articulation than this anticlimactic shrug of history's inevitable recourse to poetry, to the affectively-charged fiction, always straining toward some kind of truth-greater-than-facts, that history was in the early modern period.
Intriguingly, the sentiment is echoed by Günter Grass, whose Brechtian ``Boss,'' confronted by the tragic reality of his own historical circumstances, voices a resignation that would not be unfamiliar to Shakespeare himself: 
``Nun gut. Vielleicht fallen bei all dem Elend Gedichte ab''---``Well, maybe some poems will hatch from all this misery'' (Grass, ``Die Plebejer'' 428; \emph{The Plebeians} 110).