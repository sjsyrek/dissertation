\chapter{The abstract (and brief chronicles of the time)\label{ch:Methodology}}
\begin{bq}
It certainly seems that Shakspeare's historic dramas produced a very deep effect on the minds of the English people, and in earlier times they were familiar even to the least informed of all ranks\dots Marlborough, we know, was not ashamed to confess that his principal acquaintance with English history was derived from them; and I believe that a large part of the information as to our old names and achievements even now abroad is due, directly or indirectly, to Shakspeare.\nocite{coleridge_literary_1836}
\begin{flushright}
---Samuel Taylor Coleridge
\end{flushright}
\end{bq}
When Hamlet flatters the visiting players as the ``abstract and brief chronicles of the time,'' whose ill report does more harm than a bad epitaph, he does more than evoke the historiographical function of theater by slyly referencing one of Shakespeare’s most celebrated dramatic genres. Although the figuration of the players as ``chronicles'' is a knowing nod to the history plays that so dominated the early English stage, it is also an extension to the theater of the Renaissance poet’s claim to the power of immortalization. History was often celebrated in the Renaissance, in accordance with Cicero, as ``truly the witness of the times, the light of truth, the spirit of memory, the teacher of life, the messenger of antiquity.''\footnote{``Historia vero testis temporum, lux veritatis, vita memoriae, magistra vitae, nuntia vetustatis''~\cite[2.36]{cicero_oratore_1979}.} But this is a commentary on history as an abstract discourse about the past, history in general. Hamlet’s remark suggests that, in appropriating this discourse for dramatic presentation, the dramatist-poet imbues his own historical text with poetry's capacity to sublimate particular circumstances into general truths. Shakespeare, in this fashion, yokes history and poetry together, suggesting that poetry---specifically, the poetry of the stage---is what breathes life into history. Living performers literally perform the parts of historical actors, but the dramatic-historiographical text itself lives on, too, immune to the obsolescence and supersession to which non-dramatic, non-poetic historiographies inevitably succumb. History, in other words, supplies the tenor of this dramatic memorializing while poetry provides the vehicle. Hamlet admonishes Polonius to treat the players well, lest they tarnish his reputation through their performances and lest their performances be captured into text (an ironic threat since Polonius is himself a theatrical construct). A performance is a poetic act in motion, after all, and a historical performance that raises the dead from the past can manipulate their representations on behalf of the future.

When the ``chronicles of the time'' are given parts to play from the actual chronicles of history, Hamlet's rebuke acquires added resonance. A fictional scenario performed theatrically does not threaten posterity, but our conception of the past is indeed at stake when actual figures from history are rendered in drama---are, in effect, made fictional. Although the distinction between \emph{fabula} and \emph{historia} had been recognized well before the early modern period, there was, as yet, no agreement on where exactly the dividing line should lie. Whether true or not, history shared with fiction among many thinkers the obligation to at least be morally useful. While historical investigation became increasingly oriented toward the criticism of original sources, standards were inconsistent and poetic material was not discounted on the basis of form any more than historical episodes were for being patently impossible. The Christian valence of history remained in the background, as well. To some extent, as pointed out by Peter G. Bietenholz in a discussion of Erasmus (one of few contemporary skeptics about the veracity of ancient myths), history was like fiction, in that both were related to stagecraft: while \emph{fabulae} were like the stories enacted in plays, \emph{historiae} were performed on the universal stage of God. As products of human invention, ``both terms shared a fluid quality of truthfulness; both might involve flattering and lying, and both might reveal an abyss of profound meaning’’~\cite[148]{Bietenholz}. As far as plays themselves were concerned, English historical drama evolved coterminously with the rediscovery of the ancient histories that would go on to redefine historical thinking throughout early modern Europe. Shakespeare’s kings and Caesars were storming across the stage at exactly the same time that Justus Lipsius was promoting Tacitus throughout the intellectual circles of the continent. 

Universal history of Polybius vs. the tragic, psychological Tacitus (rediscovered 
drama evolved at a time when history was being redefined by the rediscovery of actual histories, and by the circular logic that true histories were to be verified based on their antiquity

The personages depicted in a play might have been real, that is, as might the scope of their activity, but the rendering of history into drama transforms disinterested facts into an artificial representation: a work of art that may be aesthetically astute or intellectually stimulating but, unrestrained by objectivity or accuracy, remains prejudicially selective. It is in this vein that many early modern writers complained about the theater, that its performances were lies precisely because they involved actors pretending to be other people. When theatrical actors imitated historical actors, these polemicists could have judged the threat greater still, a contamination of England's sacred national memory if not of truth itself. They would not have been so incensed by the theater, however, had its fictions not been so effective. As a part of the era's historical discourse, the history play, specifically, was arguably much more effective in capturing the public imagination than the chronicles and classics from which they were derived (though, as I will argue below, early modern histories could be just as ``poetical'' as their dramatic representations if not as viscerally compelling). In the plays, as opposed to the historiographies, characters, personified by roles in the texts and ventriloquized by body doubles on stage, speak directly for themselves and directly to their audiences. Dramatists caricature their subjects with words and actions that were never exactly theirs, and actors infuse their parts with idiosyncrasies, recognizable personalities, psychological fullness---indeed whatever extra-historical flourishes they choose. Early modern drama related history not through detached narrators, the here-say of old documents, or even second-hand speeches, but through vivid representations of those who lived it, who, living again on both page and stage, indulged readers and audiences alike in their desire, coveted as much by Renaissance humanists as modern New Historicists, to speak with the dead.\footnote{The first sentence of Stephen Greenblatt's \emph{Shakespearean Negotiations} is, famously, ``I began with the desire to speak with the dead''~\cite[1]{greenblatt_shakespearean_1988}.}

The driving conceit of historical drama, of course, is that its dramatis personae did actually live. It is a conceit necessarily borne in mind by readers and audiences of plays and particularly vulnerable, therefore, to exploitation by playwrights. This was especially the case in England at the turn of the 17th century, when the subject of history was much in vogue and history plays themselves extremely popular. Since early modern history writing tilted more toward moral exemplarity than objectivity or evidence-oriented argumentation, playwrights had enormous latitude to dramatize the past without having to pay obeisance to accuracy or verisimilitude. The national history of England---increasingly a source of identity and pride---was even more labile. A discourse still in the early stages of its articulation, the writers of the time freely shaped it into whatever form they needed. If there is merit in Hamlet's praise of the players, it is as much because of its accurate assessment of the theater's opportunistic impact on the historical imagination of the period as it is a channel for the self-praise of a successful historian-dramatist.\footnote{The notion that the intellectual culture of early modern England in particular was historically-oriented is developed at length in Kelley and Sacks, \emph{The Historical Imagination in Early Modern Britain: History, Rhetoric, and Fiction, 1500--1800}.\nocite{kelley_historical_1997} D. R. Woolf has also written a number of books that explore the historical sensibility of early modern England including \emph{The Idea of History in Early Stuart England: Erudition, Ideology, and 'The Light of Truth' from the Accession of James I to the Civil War}\nocite{woolf_idea_1990} and \emph{The Social Circulation of the Past: English Historical Culture, 1500--1730}.\nocite{woolf_social_2003} Also relevant are older studies: T. D. Kendrick, \emph{British Antiquity}\nocite{kendrick_british_1950}; F. Smith Fussner, \emph{The Historical Revolution: English Historical Writing and Thought, 1580--1640}\nocite{fussner_historical_1962}; F. J. Levy, \emph{Tudor Historical Thought}\nocite{levy_tudor_1967}; Arthur B. Ferguson, \emph{Clio Unbound: Perception of the Social and Cultural Past in Renaissance England}\nocite{ferguson_clio_1979}; and Joseph M. Levine, \emph{Humanism and History: Origins of Modern English Historiography}\nocite{levine_humanism_1987}.}
Shakespeare implies that actors, and by extension the personages they dramatize could justly claim to be historians, ``chronicles,'' in their own right: ``so long as men can breathe or eyes can see,'' that is, so long lives the account of history created by his plays. It is perhaps ironic that he was correct not because traveling bands of actors would remain England's primary vehicle of information dispersal, and not because poetic lines are figuratively immortal, but because his works, and the historical ones not least of all, were in fact printed and ultimately pored over with as much fervor and editorial scrutiny as any actual chronicle.

This dissertation argues that the early modern theater had a particularly influential role in crafting the public understanding of the past and that the history play was a vital constituent of the historical imagination of early modern England. My argument is based on the following premises:
\begin{enumerate}
\item The literary culture of early modern England, specifically at the turn of the seventeenth-century, reflected a burgeoning interest in historical matters in the society at large, but the idea of history was also flexible and open to broad interpretation, on the one hand, and calculated exploitation on the other, making the methods of history seem much like those of poetry.
\item History plays, at the same time a cause and result of this expansion of the historical mentality and its poetic quality, were among the most popular products of the theatrical world, both as performances and as printed publications.
\item The close alliance of history and poetry as genres and the availability of historical ``content’’ to poetic ``form’’ meant that these plays were themselves a legitimate form of historiography, neither universally dismissed as fictional contrivances nor perceived as merely representations of some prior historical truth.
\item As popular historiography, these plays participated in the elaboration of the ``historical imagination’’ that was beginning to spread throughout early modern society, and their usefulness for gauging the contours of this historical imagination stems from their unique status as ``performance-texts,’’ variable theatrical productions that emerged as equally variable printed texts, striated by the cultural consciousness of their audiences and readers and by the entire panoply of theatrical practices and contexts in which that consciousness was embedded.
\item Insofar as history was supposed to be useful and inasmuch as printed books were ``used’’ by their readers, plays were also subject to pragmatic uses, even if those uses were less about annotation and textual repurposing and entailed more diverse, and more nebulous, modes of reception and dissemination.
\end{enumerate}

The history play combined drama's sensitivity to the role of language in fashioning reality with the uniquely ``writable'' \footnote{See Barthes on writeable.\nocite{Barthes}.} quality of early modern historiography to supply to readers and audiences a uniquely illuminating and affective encounter with the past. Rather than specifying the historical context of a particular form of literary activity or assuming the priority of historiography to poetry in a society's articulation to itself of its past, I wish to propose instead a specifically literary context for the historical consciousness of the period. While this may look, in part, like a recursion to Sir Philip Sidney's own elevation of poetry over history, my intention is rather to treat the writing of history itself, contrary to Sidney's low opinion of it, as a practice no less informed by the methods of poetry and the sensibilities of poets than poetry itself. The heyday of Shakespeare's interest in historical topics, the late sixteenth and early seventeenth centuries, happens to have been a singularity in the development of historical discourse. A newly nationalistic England had begun to take its own history seriously even as, or perhaps because, it clung to the myth of its historical exceptionality, but before history itself became a discipline in the modern sense with methodological standards, disinterested analyses, and endowed university chairs. Secondarily, by focusing on three plays---\emph{Richard II}, \emph{Coriolanus}, and \emph{King Lear}---I demonstrate the relevance of my thesis to a cross-section of the most prominent historical modes: historical, classical, and mythical. These plays form a discernible arc of development in Shakespeare's own historiographical project, as well, his career as a dramatist having been particularly concerned with the relationship between past and present. In elaborating this argument, I have drawn on insights gleaned from work on the nature of historical discourse, on the historical culture of early modern England, and on the history of books as it pertains to reading practices pertinent to plays and playgoing. In addition, and in keeping with the literary orientation of this project, I consider the formal qualities of these plays as the working out, in poetic language, of a kind of historical thinking rather than second-order representations of an a priori rationale or set of expectations about history in the abstract. That is, the case I want to make is that the formal features and linguistic characteristics of Shakespeare's history plays have as much to do with his inferences about the meanings of history as their generically historical content. Obviously, early modern people thought many different, often contradictory things about history---what it was and what counted as true history or not---so it is impossible to establish a concept of history as either a firm ground the dramas built on or as a definitive discourse toward which they thrust. What the dramas do manage to represent, however, is the capacity for a historically-inflected imagination to hold contrasting positions and incompatible beliefs simultaneously. History plays were in a unique position to capture this uniquely heterogenous quality of historical thinking in early modern England, its multiple voices and the multifarious uses to which it could be put.

I have chosen to concentrate my efforts on Shakespeare not merely because he wrote more history plays than any other contemporary playwright, but since the extant body of his work in historical drama, and what I am including in that category, spans an exceptionally lengthy period of time during which we can trace his thinking about history and its relevance to the milieu in which he wrote. An unexpected argument in favor of Shakespeare as the primary example of dramatist-historian rests, ironically, on the extent of his collaboration with other writers working in the theater. While it is convenient, as it will be for me, to speak of ``Shakespeare'' as the author of these plays, it is perhaps closer to reality to regard his authorial function as an editorial convenience, a symbolic substitute for the host of compositional contexts and practices that went into the creation of any dramatic work in early modern England. Some of his earliest plays, as attribution studies have shown, were very likely co-authored, and a number of these are history plays. And while such studies argue that Shakespeare himself appears early on to have dispensed with this practice until the end of his career, we can never be sure exactly how much of his plays were his work alone. So many of them exist in multiple and often irreconcilable versions, satisfying conclusions about his own particular intentions are elusive. Even if textual fixity were a possibility, and not merely a contrivance of editors, each performance itself would have represented a different ``reading'' of that text and a potential spur to revision. From one to another, parts could have been swapped in or out, speeches amended in advance or extemporaneously, or lines simply forgotten. Also, the full texts of many plays were too long to ever have been performed in full. They would necessarily have been cut down or even excerpted, depending on the nature of the venue, the mood of the censors, the expectations of the audience, and the indeterminate, unpredictable whims of casts and crews. And this is to say nothing of their reception, whether by audiences at different performances or readers of different published editions. Furthermore, Shakespeare does not even appear to have been particularly concerned with print publication, so often the desideratum of literary analysis, at least of his plays.\footnote{Except see Erne for a refutation of this conventional assumption and Honigmann (1965: 189--192) for an earlier countervailing assertion that the absence of evidence of Shakespeare's interest in publication does not entail evidence of absence. My own argument, that the text of a play was an elaboration of many minds and that lay athwart more numerous cultural attitudes and theatrical practices than could be reduced to one man's singular creative vision, makes this point somewhat moot. Whether Shakespeare was or was not involved in the publication of the plays that bore his name is an economic argument that does not bear as much on the literary qualities of the plays as I have characterized them.} The concept of authorship being so different then, if it existed at all, he probably did not regard them as entirely his own, either as creative works or personal property. The characters he devised were not simply the idols of his imagination, after all. He wrote them as parts for specific members of his company (as partly indicated by the careless prompting of actors as opposed to the characters they portrayed in some of the plays, e.g. ``Will Kemp'' for Dogberry in \emph{Much Ado about Nothing}), and the contingencies of performance meant that his colleagues and business partners were as much involved in the production of his dramas as he was himself. In fact, his company owned the play scripts and most likely weighed in on their development. Why shouldn't they be considered as much ``authors'' of them as Shakespeare? When the scripts were eventually passed on (or leaked) to publishers, the publishers took over the copyright, as it were, since they owned whatever they printed. The usual practice of these publishers, when they printed plays, was to ascribe ``authorship'' to the companies that produced them, not the playwrights who happened to supply the words. If they mentioned Shakespeare's authorship at all, it was to sell more copies, not because the printed works in any way represented a final and ideal text handed down by him to posterity.\footnote{See Bentley (264--292) on some of the intricacies of the ownership and publication of play texts.\nocite{bentley_profession_1984}}

However much stock our modern notions of authorship place in the fantasy of Shakespeare's intention, his plays were, in their own time, a far more open medium than has until recently been allowed, inscribed by all the social practices of the theater and by the social pressures of theatrical production. It is for this reason that his dramatic histories occupied such a distinctive position in the early modern historiographical landscape. Inasmuch as all early modern plays both catered to and educated popular taste, the history plays reflected as well as shaped the Renaissance obsession with the past and the pastness with which the present was felt to be imbued. This was partly a function of the popularity of the theater and partly a function of the ascendency of historical discourse, in general. Within the gap between nostalgia for a lost cultural origin and the recognition of the lingering traces of that origin was the anxious anachronism by which the Renaissance has so often been defined---anxious because the past lived on in the present as a reminder of what could not be recovered and what could not hope to be its match. Although the awareness of historical difference did not necessarily appear suddenly in the Renaissance, it did evolve into an issue of markedly greater intellectual interest as texts became newly available and a self-selected, international community of scholars began tentatively to rebuild the classical legacy. Both the quantity and quality of historical writing, as well as its critical rigor, certainly surpassed anything that had come before.\footnote{Peter Burke provides a comprehensive introduction to the Renaissance discovery of anachronism that focuses, with extensive excerpting, on the historiography of the period.\nocite{burke_renaissance_1970} The various works of Woolf cited above complicate Burke's account by extending the awareness of anachronism to the capacity for, and specific modes of, more sophisticated historical thinking than had obtained during the Middle Ages.} Moreover, history had become, by the early seventeenth century, a substantial component of the imaginative life of the nation and not purely a scholarly matter. Any reading of the early modern English history plays must contend with these problematics. What these complications engender, however, are not obstacles to arriving at authorial intentions or historical/ideological representativeness, but a critical sympathy for the kind of intellectual struggle that historical topics precipitated in the early modern period. Coming from a textual tradition that was gradually losing both its authority and the reassurance that came with it and venturing into new terrain where little was known for certain and there were more questions than answers, the people who attended and read history plays did so to gain some purchase on where they might be headed. Indeed, I think these plays present an opportunity for considering not what ``Shakespeare'' thought but what sort of historical thinking they, for lack of a better word, channeled. They were, as both performances and printed literature, an important outgrowth of a society starting to think about history in creative, if not quite recognizably modern, ways. 

As a way to consider the history play's treatment, through language, of its historical subject, I think it is useful to regard early modern history as a species of poetry. It is not, obviously, a mode of writing that necessarily employs meters and stanzas, rhythm and rhyme, though it sometimes did. Rather, it is the product of the same poetic inclination that colored so much of the period’s writing, whether that writing positioned itself as ``poetical’’ or not. Poetry, in the sense I mean here, is not simply writing in the form of verse or even just writing that is avowedly ``fictional,’’ as Sidney defines it, but an attitude toward writing that accepts that 1. all writing, including historiography, is ``made’’ by writers and its materials adapted and shaped by them in a ``poetic’’ fashion and that 2. the character of such writing may be ironic with regard to its formative processes but is not, on that account, directed solely toward artifice or deception. The sort of writing that we find in early modern histories, including dramatic histories, may evince a self-consciousness about its narrative strategies but can nevertheless be sincere in its explication of a certain kind of truth. I am not suggesting, that is, that early modern writers approached their task with the same detached cynicism as our own, post-structuralist attitude toward the text, an attitude that presupposes a text’s inevitable rhetorical undoing of itself. They were capable of their own forms of cynicism, to be sure, but their appreciation of language’s slipperiness combined an indulgence in its polysemous variety with a recognition of its limitations that was canny but not resigned: they may have found it necessary to constantly revisit and replenish their representations of the ``truth,’’ but they did not have any such concept as the illimitable circularity of all utterance that they stopped aiming in that direction. What I want to argue instead is that poetry supplied Shakespeare with a semantics of understanding for the interpretation of history. The poetic process at work in his dramatic historiography was more than a process of appropriation whereby a distinct literary genre has poured into its thematic shell the hearty sustenance of an equally distinct historical one; likewise, Shakespeare's plays were not simply versions of history, adaptations of the real thing, or fanciful dramatizations everyone knew to be false. Poetry, rather, is the means by which Shakespeare's plays worked to give history meaning. Though they were certainly interpretations of available source materials, they would not necessarily have been received as anything other than true and accurate. The boundaries of historical understanding he would have shared with his audiences and readers were different than they are now, and the definition of historical truth much more flexible than even subsequent decades would allow. Strict standards of evidence did not exist, and the protestations of individual scholars, however ahead of their time they may have been, had as yet little influence on popular mentalities. Since the aims of history and poetry remained so similar, even if they were not always formulated as such, they borrowed freely from each other; as usual, in the Renaissance, discretion and credulity could coexist happily. And at a time when the past's primary value was in its applicability to matters of present concern, Shakespeare's method was to use history to practice his ideas in ways no less legitimate than would-be councilors of state, who gave wings with history to their own flights of poetic fancy.

My intention here and following is not to argue that history is poetry and poetry is history, as if distinctions do not matter and as if early modern writers had no sense of or opinions about their differences, but rather that their practical uses of the terms, as well as the overlaps in their perceived utility, means that neither in its early modern manifestation is prior to the other. The very fact of their family resemblance surely made any attempt to define one in terms of the other (even in opposition) an exercise in exactly the sort of rhetoric from which they both descended. I bear in mind throughout Sir Philip Sidney's remark that ``many times [the historian] must tell of events whereof he can yield no cause; or, if he do, it must be poetically''\cite[36]{sidney_defence_1966}. This is more often than not the case. If the terms were not exactly used interchangeably, the debate over the relative merits of history and poetry, which goes back to Aristotle and was continued by Renaissance scholars, indicates their common intellectual heritage and close semantic relationship.\footnote{See Kelley, ``Theory of History,'' for an overview of this debate as it was (inconsistently) pursued in the Renaissance.\nocite{kelley_theory_1988}}

What is history?

The analysis of history strives to distinguish between history as the practice or conceptual framework of a historian and history as the object of study, a content that appears neutral as to that framework. The methodology, however, is inseparable from the object of its analysis, because they are mutually deterministic, the one setting in advance the parameters for the interpretation of the other. That is, history does not, like philosophy, advance abstract and universal propositions; the arguments of historiography rest on some purported basis of facts or data---the materials of investigation---however mediated by subjective processes of selection, which delimit the scope of its scrutiny inasmuch as their arrangement is itself determined by the procedures of that analysis. In his famous essay on the subject of history, E. H. Carr writes that history is neither ``a hard core of facts surrounded by a pulp of disreputable interpretation'' nor ``a hard core of interpretation surrounded by a pulp of disputable facts''~\cite[23--24]{carr_what_1962} but ``a continuous process of interaction between the historian and his facts, an unending dialogue between the present and the past'' (30). Going further along these lines, Paul Hamilton notes the opening that skepticism about the truth claims of historiographic rhetoric creates for literary criticism:
\begin{bq}
Even after we as readers have ceased to be convinced, looking back at dated historical interpretations, what we notice are the master-tropes employed, the strategies for persuading us that evidence is being used in the proper sense, the mechanics of articulation. The justification of an interpretation is lodged in its expression. Explanation and historiography, history and its writing, appear to have become the same thing.~\cite[18]{hamilton_historicism_1996}
\end{bq}
While Carr's definition can serve just as well for literary criticism as for history, it aptly describes the no less dialectical process inherent in the poetic rendering of history. Hamilton merely collapses a distinction that, for early modern writers at least, was tenuous at best. According to Michel de Certeau:
\begin{bq}
History is not an epistemological criticism. It remains always a narrative. History tells of its own work and, simultaneously, of the work which can be read in a past time. Besides, history understands the latter only as it elucidates its own productive activity, and reciprocally, it understands its own work through the set of productions, and the succession of productions, of which this history is itself an effect.~\cite[43]{certeau_writing_1988}
\end{bq}
Though lacking the sophisticated methods and metrics of inquiry since developed by historians, particularly in the scientifically-oriented twentieth-century but no less impressively by their immediate successors in the seventeenth, writers of and on history contemporary with Shakespeare would have recognized the broad outlines of this problem. They could have asked, as we continue to, ``What is history as distinguished from other kinds of writing, and what is it good for?'' The content of history in early modern Europe would have been unanimously agreed to be the corpus of Greek and Roman historians. The usual method of analysis would have been to mine it for morally edifying exempla. This, at least, was the starting point for any humanist writer. Toward the late sixteenth-century, an energetic wave of writers began rethinking the means and aims of historiography. They did not subscribe to a unified program, however, and past models mingled freely with the latest thinking. Even myths and forgeries had a place in the understanding of what counted as history. As William Camden argues, the fiction that a people is descended from a noble race is useful, because it at least implies the importance of virtue.

``History is truly the witness of the times, the light of truth, the spirit of memory, the teacher of life, the messenger of antiquity.''\footnote{``Historia vero testis temporum, lux veritatis, vita memoriae, magistra vitae, nuntia vetustatis''~\cite[2.36]{cicero_oratore_1979}} 
We also find this exact sentiment expressed in the commendatory poem that prefaces The History of the World by Sir Walter Raleigh, who translates it as ``Times witness, Herald of Antiquitie, / The light of Truth, and life of Memorie''~\cite[N. pag.]{raleigh_history_1614}
historia magistra vitae ``Until the eighteenth century, the use of our expression remained an unmistakable index for an assumed constancy of human nature, accounts of which can serve as iterable means for the proof of moral, theological, legal, or political doctrines. Likewise, the utility of our topos depended on a real constancy of those circumstances implying the potential similitude of earthly events. If there were a degree of social change, it occurred so slowly and over such a period that the utility of past examples was retained. The temporal structure of past history bounded a continuous space of potential experience''~\cite[28]{koselleck_futures_2004}.\footnote{``Bis zum 18. Jahrhundert bleibt die Verwendung unseres Ausdrucks ein untrügliches Indiz für die hingenommene Stetigkeit der menschlichen Natur, deren Geschichten sich zu wiederholbaren Beweismitteln moralischer, theologischer, juristischer oder politischer Lehren eignen. Aber ebenso beruhte die Tradierbarkeit unseres Topos auf einer tatsächlichen Konstanz jener Vorgegebenheiten, die eine potentielle Ähnlichkeit irdischer Ereignisse zuließen. Und wenn ein sozialer Wandel strattfand, dann so langsam und so langfristig, daß die Nützlichkeit vergangenen Geschichte begrenzte einen kontinuierlichen Raum möglicher Erfahrbarkeit''~\cite[40]{koselleck_vergangene_1979}. An important piece of Koselleck's evidence for this semantic transition is the gradual displacement in German of Historie by Geschichte, a linguistic differentiation lacking in English but indicative of an ideological process then occurring throughout the learned community of Europe.}

(compare Collingwood and Grafton, What was History? who points out that not everyone bought this). For the decline of the \emph{ars historica}, see Grafton, \emph{What Was History?} 
189--254.\nocite{grafton_what_2007} Practical history never entirely died out, however. Bolingbroke, in the eighteenth century, was still defending it as history's principal purpose in incendiary terms: ``That the study of history, far from making us wiser, and more useful citizens, as well as better men, may be of no advantage whatsoever; that it may serve to render us mere antiquaries and scholars; or that it may help to make us forward coxcombs, and prating pedants, I have already allowed. But this is not the fault of history: and to convince us that it is not, we need only contrast the true use of history with the use that is made of it by such men as these. We ought always to keep in mind, that history is philosophy teaching by examples how to conduct ourselves in all the situations of private and public life; that therefore we must apply ourselves to it in a philosophical spirit and manner; that we must rise from particular to general knowledge, and that we must fit ourselves for the society and business of mankind by accustoming our minds to reflect and meditate on the characters we find described, and the course of events we find related there''~\cite[47--8]{bolingbroke_letters_1752}.

Much ground between the Ciceronian ``magistra vitae''
Sidney: ``bare was''
``the historian in his bare \emph{Was} hath many times that which we call fortune to overrule the best wisdom. Many times he must tell events whereof he can yield no cause; or, if he do, it must be poetically''~\cite[36]{sidney_defence_1966}.

Hayden White, Metahistory:

``In this theory I treat the historical work as what it most manifestly is: a verbal structure in the form of a narrative prose discourse. Histories (and philosophies of history as well) combine a certain amount of ``data,'' theoretical concepts for ``explaining'' these data, and a narrative structure for their presentation as an icon of sets of events presumed to have occurred in times past. In addition, I maintain, they contain a deep structural content which is generally poetic, and specifically linguistic, in nature, and which serves as the pre critically accepted paradigm of what a distinctively ``historical'' explanation should be. This paradigm functions as the ``metahistorical'' element in all historical works that are more comprehensive in scope than the monograph or archival report.''~\cite[ix]{white_metahistory:_1975}

Spenser's forms of history:

``The long-established argument for the Renaissance recovery of such concepts as 'anachronism', 'awareness of evidence', and 'interest in causation' holds within it an important element of truth: irrefutably, late sixteenth-century England was gradually moving towards a new understanding of its distant past. On the other hand, what such accounts of progressive struggle find more difficult to convey is the simultaneous highly diverse picture presented by a synchronic approach. If, in retrospect, it is possible to trace such developments as the decline of the chronicle, or the abandonment of myths about the country's Briton heritage, these conclusions were inevitably beyond the reach of late sixteenth-century observers''~\cite[8]{van_es_spensers_2002}.

The method and substance in Early Modern history writing ``appear to have depended greatly on one another. Although 'truth' was something to which almost universal claim was made, it was in practice a malleable quality. For the authors of humanist histories, for instance, truth was often as much instructive or aesthetic as it was factual... The balance between competing factual, political, or moral 'truths' was--at least in part--a question of genre. The author of an antiquarian discourse (a mode that made little claim for its ethical content) could judge a story on criteria very different from those of a chronicler''~\cite[12--13]{van_es_spensers_2002}.

``Forms of history'' according to Van~Es varied depending on the genre of the text with no priority granted to one over another in a period of intense confusion about what counted as true history and what didn't and intense interest in dialogue with the past (vs. detached observation from a critical, clinical distance).

To read Shakespeare historically is not only to check his plots against hard evidence, interpret his plays as fantasies that concealed contemporary personages in antique garb, or determine what nexus of material and ideological culture determined the rhetorical disposition of the plays's contents, but is to consider the historical content as one component in an array of imagery that points but does not contain. ``In formal imitation, or Aristotelian mimesis, the work of art does not reflect external events and ideas, but exists between the example and the precept. Events and ideas are now aspects of its content, not external fields of observation. Historical fictions are not designed to give insight into a period of history, but are exemplary; they illustrate action, and are ideal in the sense of manifesting the universal form of human action. (The vagaries of language make ``exemplary'' the adjective for both example and precept.)''~\cite[84]{frye_anatomy_2000}. History was something that could be explored in a speculative way, such that the distance between a historical fiction and historical fact was not a practical one.

Joel Altman proposes that Renaissance drama were ``fictional realizations of questions'' that raised but did not answer them, in keeping with the rhetorical education of the time, in which students were conditioned to argue on both sides of the question, resulting in the ``frequently disconcerting shifts of viewpoint'' characteristic of so much writing at the time. Might be used (c.f. Used Books) by audiences to construct their own hypothesis or as bases for learned conclusions or prejudiced opinions.~\cite[3]{altman_tudor_1978}

Historiographical methods in the early modern period could be scrupulous and farcical at the same time, with no tenor of irony undermining suppositions that were more the product of ingenuity and deductive improvisation than what we would recognize as sound principles of research. This is not, however, to be critical of those writers who indulged their historical imaginations. The early modern world, after all, was itself becoming ever more historicized: there was a general thirst for history and for historical explanations that could make sense of the present by means of the past. Geography, architecture, economics, politics, institutions both secular and religious, laws, lineage, and names had all acquired for people a veneer of temporal significations. In the devising of explanations for a society that was newly alert to its own historical contingency, but had as yet neither standards for the authentication of evidence nor a practiced objectivity in approaching it, history writers groped after answers using the critical tools that were available. In so doing, they cleared as much new intellectual terrain as they reinforced bulwarks of traditional knowledge. William Camden speculates about the origin of the name of the British people much along the lines of Isidore of Seville, the Medieval encyclopedist who was canonized the same year, 1598, that Shakespeare and his colleagues were securing funding for the Globe, and whose \emph{Etymologies} remained enormously influential:
\begin{bq}
What if I should conjecture, that they were called Britans of their depainted bodies? For, whatsoever is thus painted and coloured, in their ancient countrey speech, they call \emph{Brith}. Neither is there cause why any man should thinke this \emph{Etymologie} of Britaines to be harsh and absurd; seeing the very words sound alike, and the name also as an expresse image representeth the thing, which in \emph{Etymologies} are chiefly required. For \emph{Brith} and \emph{Brit}, doe passing well accord: and that word \emph{Brith} among the Britans, implieth that which the Britans were indeed, to wit, \emph{painted}, \emph{depainted}, \emph{died}, and \emph{coloured}, as the Latine Poets describe them; and Aαἰολονώτων, that is, \emph{having their backs pide}, \emph{or madly coloured}, as Oppianus termeth them.~\cite[26]{camden_britain_1637}
\end{bq}
Camden glosses this passage with a reference to the \emph{Cynegetica}, a verse hunting treatise from the turn of the third century A.D. ascribed, probably erroneously, to Oppian of Apamea. In this work, there is a passing reference to the tracking dogs of the ``Βρετανῶν αἰολονώτων,'' or ``painted Britons''~\cite[1.470]{oppian_1928}. Isidore's own etymology of ``Britain,'' to which Camden refers on the page following this passage, invokes the Roman poet Virgil as well as another popular myth about the origin of the British people: ``Brittones quidam Latine nominatos suspicantur, eo quod bruti sint, gens intra Oceanum interfuso mari quasi extra orbem posita. De quibus Vergilius (Ecl. 1, 67): Toto divisos orbe Britannos''~\cite[9.2.102]{isidore_of_seville_etymologiarum_1911}; ``Some suspect that the Britons were so named in Latin because they are brutes (\emph{brutus}). Their nation is situated within the Ocean, with the sea flowing between us and them, as if they were outside our orbit. Concerning them, Vergil (\emph{Ecl}. 1.66):
The Britons, separated from the whole world''~\cite[198]{isidore_of_seville_etymologies_2006}. They are brutish, that is, \emph{because} they are separated from the Mediterranean world, the civilized ``orbit'' of Isidore and his contemporaries. The authority of Virgil and the resemblance of ``British'' to ``brute'' will, of course, serve as the foundation for yet another origin myth, Geoffrey of Monmouth's \emph{History of the Kings of Britain}, which is itself the origin of the story of \emph{King Lear}. Camden puts this myth into question at the outset of his history, though he hesitates to deny its credibility entirely. 

There is contained in this intertextual side excursion a remarkable encapsulation of early modern knowledge transmission. A church father, Isidore, quotes Virgil---and does so frequently---as if a pagan poet were a reputable source of knowledge; a humanist historian, Camden, himself cites an ancient poet, Oppian, to illustrate his own Isidorean free association---and only to offer an alternative, with respect, to the popular Brutus legend, which also has a Virgilian antecedent. Christian and pagan, ancient and modern: they remain to the early modern mentality part of the same intellectual culture, a culture in which truth is formed out of  correspondences and inferences that take place among a web of texts all talking to one another.

Historical Culture

Dorislaus, Cambridge's first lecturer in history, relieved of his post for delivering a lecture on Tacitus that was, according to the intellectual bureaucrat and courtier Matthew Wren, ``stored with such dangerous passages (as they might be taken) and so applicable to the exasperations of these villainous times, that I could not abstayne before the Heads there present to take much offense that such a subject should be handled here, and such lessons published, and at these times, and E cathedra theologica, before all the university'' (qtd. in Mullinger 87)\nocite{mullinger_university_1911}

Indeed, the history play's idiosyncratic purchase on the historical imagination of early modern England was the product of an entire culture that had itself become increasingly historical. Moral and political lessons drawn from the Greek and Roman historians had long held an honored place in the humanist's repertoire, and historical narratives and antiquarian analyses of the English past were coming into vogue.

``A historical culture consists of habits of thought, languages, and media of communication, and patterns of social convention that embrace elite and popular, narrative and non-narrative modes of discourse. It is expressed both in texts and in commonplace forms of behavior---for instance, the resolution of conflicts through reference to a widely accepted historical standard such as 'antiquity'. The defining characteristics of a historical culture are subject to material, social, and circumstantial forces that, as much as the traditionally studied intellectual influences, condition the way in which the mind thinks, reads, writes, and speaks of the past. Above all, the notions of the past developed within any historical culture are not simply abstract ideas, recorded for the benefit of subsequent generations… Rather, they are part of the mental and verbal specie of the society that uses them, passing among contemporaries through speech, writing, and other means of communication.''~\cite[9--10]{woolf_social_2003} There is evidence that these changes extended well beyond the much-studied bounds of a rapidly-expanding London, as well. Much of England's medieval period housing stock, for example, was rebuilt and reequipped in the early modern period, effecting, in a country of entirely reconstructed villages, a greater degree of visual change from the past than had hitherto been experienced. See W. G. Hoskins, \emph{Rebuilding,} 44--59.\nocite{hoskins_rebuilding_1953}

``Historical discourse was increasingly a key part of sociable relations, including casual conversation, playfulness and courtship, human interactions that run a gamut of feeling from the sublime to the ridiculous, and which embrace an even wider range of conversational contexts from the political and economic to the sexual and drunken.''~\cite[131]{woolf_reading}

``One reason it is so hard to write at large about Renaissance historiography is that there were so many competing and overlapping notions of what `history' was or ought to be---but, at the same time, from this muddle emerged meaningful experiments in establishing ways of recording and commenting on societies changing over time.''~\cite[95]{colie_resources_1973}

François Baudouin, Institution of Universal History and Its Conjunction with Jurisprudence (1561) (see also No Island is an Island), Jean Bodin, Method for the Easy Comprehension of History (1566) (see also Grafton noting that Bodin's Republic was popular in England). Baudouin's indirect influence on Sidney, Puttenham, and other defenders of poetry, see Carlo Ginzburg, No Island is an Island, 26--42. \nocite{ginzburg_no_2000}

Error, ars historica, forgery

Francis Rigolot and the ``cultivation of error'' which has a certain aesthetic truth. Humanists interested in seeking out errors, but errors were also deliberately created for the explication of a different sort of truth.~\cite[1219--1234]{rigolot_renaissance_2004}

``In its heyday, the new art of history seemed to carry all before it. In the decades just before and after 1600, the ars historica glowed with all the prestige and charm that can invest a fashionable genre... Bliss was it, at least for Bodin, to be alive in 1580 in Cambridge, where every desktop sported a copy of his Republic---a work that he saw, for all its theoretical departures from the Method, as a formal continuation of the earlier book. Baudouin, Bodin, and the rest convinced the erudite patricians who managed universities and learned gymnasia across Europe to see history, as they did, as a formal discipline, one comparable to law in utility and status.''~\cite[192--193]{grafton_what_2007}

History is not a background against which anything is explained for early modern subjects---their own peculiar psychology or the present state of affairs in domestic or international politics. Identity is a nebulous word to describe a condition of being that is alleged always to be at once inherent and spontaneous and also shaped, limited, or accumulated by outside forces (these either deliberate, as in the case of propaganda or official ideologies, or unintended consequences of ideologically-driven acts---subversive, unconscious, or otherwise). There is a sense in which history is as ineluctable as a god, but history is a gnostic god hiding behind the scenes and which had to be discovered by a humanity desperate to confront reality in its empirical thingness, with no illusions. To the early modern humanists, illusions abounded in the world made up of texts in which they lived, and they saw it as their job to penetrate them, striking down shibboleths large and small, offering to unravel society's quilting points with one hand while performing the most incidental atheteses with the other. These humanists struggled with history, struggled to make sense of an incomplete historical record that had desperately to be clawed back from the oblivion of the Middle Ages against which they defined themselves.

Problem: poetry tends to cling to traditional structures of knowledge, folklore, myth, there is a kind of poetic sentimentality about these ancient structures and stories, an inherent conservatism that creates a kind of shared ground of experience that also makes it difficult to discover in literature of the period influences of the most cutting-edge contemporary thinking. There were new thoughts in the air, but people did not want to give up the traditional ones. As with Annius of Viterbo (see Grafton on wanting to cling to these things). Also Spenser who could write The Faerie Queene using the Tudor myth but also A View of the Present State of Ireland which is wholly practical in its design.
Annius of Viterbo's Commentaria super opera diversorum auctorum de antiquitatibus loquentium (1498) was ``the most complex, meticulous, and systematic effort of literary counterfeit ever recorded''~\cite[102]{stephens_giants_1989}.\footnote{For his more extensive account of Annius of Viterbo, see Walter Stephens, ``When Pope Noah Ruled the Etruscans: Annius of Viterbo and His Forged `Antiquities.'''\nocite{stephens_when_2004}}
(this is where I justify the popularity and publication of history plays as part of the historical-minded culture of the period)

``The rediscovery of the classical tradition in the Renaissance was as much an act of imagination as of criticism, as much an invention as a rediscovery; yet many of the instruments by which it was carried out were themselves classical products rediscovered by the humanists. Paradox, contradiction, and confusion hold illimitable dominion over all.''~\cite[103]{grafton_defenders_1991}

Relationship of history to poetry

(Perhaps combine this section with Nash, above, with reference to: In the most recent Shakespeare Survey, devoted to ``Shakespeare's English Histories and their Afterlives'' (fix citation below), Jean-Christophe Mayer marks off the the appreciation that many of the period's writers had for the singular ability of history plays ``to make the dead present through remembrance and to renew the acts of the dead by action''~\cite[20]{mayer_decline_2010}. Mayer discovers in this appreciation a species of practical fiction-making that frequently takes the form of an ``as if''' analogy. Thomas Heywood's \emph{Apology for Actors} he cities as paradigmatic of this sort of thinking.

Letter from Philip to Robert Sidney on poetry and history: ``In that kind you have principally to note the examples of virtue and vice, with their good or evil success, the establishment or ruins of great estates, with the causes, the time, and circumstances of the laws then written of, the enterings and endings of wars, and therein, the stratagems against the enemy, and the discipline upon the soldier; and thus much as a very historiographer. Besides this, the historian makes himself a discourser for profit, and an orator, yea a poet, sometimes for ornament. An orator, in making excellent orations 'e re nata,' which are to be marked, by marked with the note of rhetorical remembrances: a poet, in painting forth the effects, the motions, the whisperings of the people, which though in disputation one might say were true, yet who will mark them well, shall find them taste of a poetical vein, and in that kind are gallantly to be marked: for though perchance they were not so, yet it is enough they might be so.''~\cite[220--221]{sidney_correspondence_1912}

``If the oscillation of Renaissance writers between fact and fiction disconcerts modernity, it becomes intelligible once we recognize, in the history and poetry of that time, not distant or opposing activities but alternative and complementary means of instruction, which alike brought material beyond the real and present world to bear upon it. In the educational practices and the adult reading of the time, history and fiction reinforced each other's lessons.''~\cite[77]{worden_historians_2006} Blair Worden. Compare with Peter Mack, who points out the interrelations among histories, conduct manuals, and romances: ``Texts in all three genres reuse material from earlier writings and in turn present subject-matter for further reuse''~\cite[135]{mack_elizabethan_2002}. Donald R. Kelley, ``Between History and System,''\nocite{kelley_between_2005}, surveys the changing semantic field of the term history from the Renaissance through the Enlightenment. Also useful is Kelley, ``The Theory of History,''\nocite{kelley_theory_1988} which discusses history's emergence from and eventual overshadowing of poetry as a discourse of truth and utility.

``His style, like that of most of his contemporaries, abounds with poetical terms and allusions bordering a little on conceit. As far as language is concerned, it has been the translator's intention to make the Britannia an English classic, calculated for every reader''~\cite[vii]{gough_preface_1787}.

The historical culture of early modern England combined the vibrancy of a cutting-edge field of inquiry with the intensity of an avant-garde artistic scene. On the one hand, the humanist recovery of the past had evolved, by the early seventeenth century, into a high-stakes debate about the purposes to which the past should be put in the present. The questions asked were many. Does history light the way to correct behavior? Does it provide the key to human motivation? Should knowledge of their origins cause us to fundamentally reevaluate our cultural practices and institutions? Does the past, in fact, have anything at all to say to a present that is circumstantially different? On the other hand, the flourishing of literary activity and the lack of clear disciplinary boundaries in which it occurred meant that writers of a more poetic inclination freely applied to the outpouring of historical scholarship their own imaginative acumen. The didactic examples of history were perfectly at home among the ideal archetypes of poetry. Moreover, they possessed equally the capacity to critique and transform their source materials and enjoyed just as equally an elevated status among the discourses of the period. No definitive claim to superiority could be made by one or the other, though such claims were made. It is a testament to the unique literary formation of the early modern period that an opposition between history and poetry could even be imagined. But historiography could be as multivocal as literature (see Annabel Patterson).\nocite{patterson_reading_1994}
and the critical capacities of both genres are hardly to be distinguished at the high level at which both of them operated.

Edward Hall introduced into English history writing elements of the dramatic, such as invented speeches, heavy-handed moralization, and dichotomous characterizations of important figures, that would go on to influence Holinshed and, by extension, Shakespeare. He was also responsible for extending the Tudor myth beyond what his source and predecessor Polydore Vergil had achieved and introduced into the popular imagination the stereotypical characteristics applied to, e.g., Henry V, Richard III, and Margaret of Anjou.\nocite{hall_union_1550} According to May McKisack, Holinshed exhibits ``not much in the way of literary artistry,'' ``the lack of any selective principle governing the composition of the histories,'' and ``lack of critical discrimination''~\cite[117--118]{mckisack_medieval_1971}. Selden, a follower and correspondent of the great continental philologist and chronology Joseph Scaliger, friend of Ben Jonson, did more to advance the study of history in England to something approximating its modern form than anyone hitherto.\footnote{See Woolf, \emph{Idea of History}, 200--42.}

\emph{The first part of the life and raigne of King Henrie the IIII} by John Hayward (1599) based partly on Shakespeare's Richard II according to some perceptive critics.\nocite{hayward_first_1599}

A burgeoning interest in British history registered not only an invigorated appreciation of historical difference, as significant a paradigm shift as that was in the intellectual life of Europe, or the formative stages of nationalistic sentiment, but also the incipient availability of history as a resource ripe for all sorts of ``poetic'' exploitation: lawyers mythologized a pre-Norman source for English common law; heralds dreamt up lineages demonstrating the descent from Adam of their aristocratic patrons; scholars and learned editors mixed authentic with spurious sources in their struggle to piece together the early accounts of antiquity; ``chorographers'' yoked localized history and folklore to geographical description; and poets heaped up grand narratives in verse onto an already vertiginous pile of rehashed, mytho-historical romances.

A whole literature discussing legal theory and practice in early modern England exists. The most useful entrée for the literary scholar is probably Richard Helgerson, \emph{Forms of Nationhood: The Elizabethan Writing of England}.\nocite{helgerson_forms_1992} The classic study of the subject after Maitland remains, however, J. G. A. Pocock, \emph{The Ancient Constitution and the Feudal Law: A Study of English Historical Thought in the Seventeenth Century}.\nocite{pocock_ancient_1987} See especially, for the present discussion, 30--55. Also useful are Donald R. Kelley, ``History, English Law and the Renaissance,''\nocite{kelley_history_1974} a follow-up, with an alternative perspective, by Christopher Brooks and Kevin Sharpe, also entitled ``History, English Law and the Renaissance,'' \nocite{brooks_history_1976} and Kelley's rejoinder in the same issue. For detailed analyses of witting and unwitting heraldic forgeries and the genealogical sciences that eventually emerged to critique them, see J. Horace Round, \emph{Studies in Peerage and Family History}.\nocite{round_studies_1901} Of especial interest to Shakespeare scholars is his chapter on the origin of the Stuarts, 115ff. An excellent overview of the heraldic literature that came out of Renaissance England's fascination with symbols of honor is available in J. F. R. Day, ``Primers of Honor: Heraldry, Heraldry Books, and English Renaissance Literature.''\nocite{day_primers_1990} As Day points out, the fabrication of lineages was commonplace enough for Sir Thomas Smith to state matter-of-factly that any self-appointed gentleman could purchase a coat of arms from a herald, ``the title whereof shall pretende to haue beene found by the said Herauld in perusing and viewing of olde registers''~\cite[28]{smith_republica_1583}. Kevin Sharpe argues that the Elizabethan obsession with heraldry was part of a broader engagement at all levels of society with with visual signification and representation, the best-selling heraldic literature of the period playing a part in the fashioning of ``an educated interpretive community, fully able to read visual symbols as encodings of values and virtues and of privilege and power''~\cite[359]{sharpe_selling_2009}. Surely this uptick in the elaboration of symbolic imagery, though perhaps by Sharpe overstated, mingled with the verbal significations so vital to the early English theater, another prominent venue for fictional historicizing, as well. The best account of the ``creative forgers'' who brandished the tools of textual criticism in support of fraudulent documents of long-standing authority is Anthony Grafton, \emph{Forgers and Critics: Creativity and Duplicity in Western Scholarship}.\nocite{grafton_forgers_1990} For discussions of chorography and local history, see Helgerson 105ff; W. G. Hoskins, \emph{Local History in England,}\nocite{hoskins_local_1972} and more recently Jan Broadway, \emph{'No Historie so Meete': Gentry Culture and the Development of Local History in Elizabethan and Early Stuart England}.\nocite{broadway_no_2006}

And these were often the same people. Insofar as their makers applied their imaginative faculties to existing historical knowledge in order to produce their own particular creative interpretations---embellishing some accounts while diminishing others, adding to or reducing their stock of facts and confirming, refuting, expanding, or rearranging them as they saw fit---these interpretations were, I would argue, the products of rhetorical and stylistic conscientiousness about the disposition of discourse---of, that is to say, poetic processes. I am following here those early modern writers who regard the poet as a ``maker'' of fictions, the truth value of which resided more in their usefulness or verisimilitude than in their factual accuracy, standards of evidence being not totally absent in the period but certainly much looser in strictness, vaguer in definition, and lacking applicability to a well-defined and institutionally policed genre of historical writing. That history or at least some portion of it tends to be fabricated is allowed, e.g., by the author of \emph{The Arte of English Poesie}:
\begin{bq}
``These historical men neuerthelesse vsed not the matter so precisely to wish that al they wrote should be accounted true, for that was not needefull nor expedient to the purpose, namely to be vsed either for example or for pleasure: considering that many times it is seene a fained matter or altogether fabulous, besides that it maketh more mirth than any other, works no lesse good conclusions for example than the most true and veritable.''~\cite[55]{puttenham_arte_1970}
\end{bq}
The point is not simply that Renaissance histories were mere fictions because of their reliance on principles of rhetoric shared with poetry but that both genres, by means of persuasion and exemplarity, aspired to a variety of truth not necessarily coextensive with the proofs of experience. Until Francis Bacon, there was no program of falsifiability to distinguish the natural from the human sciences, both of which began their careers as derivatives of history. On the general application of ``history'' to all categories of knowledge prior to the Enlightenment, see Arno Seifert.\nocite{seifert_cognitio_1976}

 Though much early modern historical writing would persist in reputation and was indeed a significant part of the flowering of Renaissance letters, the true ``de-poeticizing'' of history, instigated by the cultural ascendance of a more trenchant intellectual skepticism, would require the intervention of a later, more avowedly empirical age.*4* The historical work of Francis Bacon is a commonly cited terminus a quo, but the real breakthroughs did not take place until the later seventeenth century, when Shakespeare's works themselves became subjects of scholarly inquiry. Until the current shifted in that direction, the difference between history and poetry as kinds of writing was fine enough to advantage reflection on the mutual, cross-genre borrowings that blurred the distinction to begin with.

*4*The process was not entirely without precedent, of course, and we can perhaps recognize its first glimmers in the celebrated Pyrrhonism of Montaigne. The emergence of abstract skeptical attitudes in certain quarters, however, did not yet inspire a widespread divestment or reconsideration of long-standing opinions and prejudices, nor did they save continental Europe from the convulsions of reformation:
\begin{bq}
``Tentation de plus en plus explicable d'ailleurs, que celle du doute, à mesure que s'écoulait le siècle. On s'apercevait que ce merveilleux épanouissement des lettres humaines, salué avec tant d'espoir, n'avait pas rendu les gens plus sages; il semblait qu'on n'apprit du nouveau que pour se défier de ce qu'on savait déjà; l'enthousiaste élan de la première génération humaniste s'arrêtait aux aspérités de la recherche, s'accrochait aux épines de l'erreur, de l'illusion, du fanatisme, se brisait contre la passion politique et la haine religieuse: après avoir magnifiquement présumé de sa puissance, l'homme constatait quelquesunes de ses incurables faiblesses.''~\cite[45]{pintard_libertinage_2000}
\end{bq}
A strong sectarian bias thus remained at the heart of most historical writing of the late sixteenth and early seventeenth centuries, compromised as it often was by, e.g., strident religious sentiment, a reliance on patronage, and state censorship. History, while benefiting immeasurably from the searching critiques of Humanism, was hard to separate from projects of national identity and the concomitant expectations of religious and political allegiance.

History Plays

Drama turned out to be an ideal form for the dilation of the edifying themes and exemplary characters found in early modern historical matter. The histories of the Renaissance were generally concerned with expounding on universal themes, such as the inevitable fall from power of proud and mighty princes or the teleological thrust of all history toward the preeminence of a great nation---Rome in the case of Polybius, the most popular ancient historian, England in the case of his insular imitators. Such themes were hardly absent from the Renaissance stage, but the structural role they played in organizing historiography was complicated in drama by the foregrounding of character, which introduced into the understanding of history the ambiguities of motive as well as the full breadth of human psychology with all its capacities and flaws, its strengths and weaknesses of spirit. Fortune and fate would remain powerful ideas but only in the discourse of actors taking subjective positions, applying these themes to their attempts to understand their world. Indeed, drama and history seemed to discover each other all of a sudden at the end of the sixteenth century. A vogue for the new historical currents sweeping the intellectual culture of the Continent reached England at a time when that country's national identity was, in the wake of decades of religious strife, more unstable than ever. This adaptation of history for the stage was not inevitable, however.

Prior to the 1590s, drama in England consisted almost entirely of moralizations, court entertainments, adaptations of Bible stories, and plodding translations of Seneca. While savvy criticism and historical investigation can find plausible resonances among these plays with all manner of contemporary events, their formal qualities were nothing like the richly characterized, multiply plotted works to come at the end of the century. Moreover, they were never staged for mass public consumption, and there is little evidence for their impact beyond the occasion of their composition. When Marlowe's \emph{Tamburlaine} swaggered onto the scene in the late 1580s, a new era in dramatic writing appears to have been inaugurated. The abstractions and Roman imitations of the preceding decades were suddenly displaced by a bold, humanistic protagonist and the blank verse of ``Marlowe's mighty line.'' There were few precedents for this new style of stagecraft. The change in the topical content of English drama can be seen even from their titles. A survey of the Stationers' Register, which records the publication of these plays, turns up for the early part of the century such works as: \emph{The World and the Child} (1522), \emph{The Nature of the Four Elements} (?1526-7), \emph{Temperance} (?1528), \emph{1 \& 2 Gentleness and Nobility} (?1529)---perhaps indicating an early affinity for two-parters---\emph{Magnificence} (?1530), \emph{The Play of Love} (1534), \emph{1 \& 2 Nature} (?1530-1534), and \emph{Youth} (?1530-1535).\footnote{W. W. Greg Bibliography of English Drama.\nocite{greg_bibliography_1939}} \emph{Gorboduc}, published as \emph{Ferrex and Porrex} in 1565, is an exception and the play usually marked as the first proper tragedy, but it seems to be an early outlier and retains the strong Senecan stylization that later history plays would drop. \emph{Jocasta}, translated from the Greek of Euripides by George Gascoigne and Francis Kinwelmersh and published in 1573 is much the same (the Elizabethans, at least, seemed to prefer the Romans), though also indicative of the exceptional learning that could go into the composition of early modern plays. \emph{Tamburlaine}, published in 1590, was born \emph{ex nihilo} where drama is concerned. Its precedents must be located elsewhere: in the flourishing of non-dramatic poetry at the time, a new fad for both historical writing and exotic travel literature (\emph{Tamburlaine} is, arguably, both), and in Marlowe's peculiar, tendentious genius. But Marlowe's literary achievement, however notable, was also a historiographical one. As Irving Ribner pointed out long ago, ``it is sometimes forgotten... that the first part of \emph{Tamburlaine}, the play whose overwhelming success virtually ushered in a new era in the English drama, is also a history play''\cite[251]{ribner_idea_1953}. The sudden vogue for and incredible popularity of history plays is probably as surprising as their equally sudden fall from fashion. But in their heyday, they did much to fix the parameters of what were previously far more amorphous genres as well as to explore the limits of what history, as a genre, could contain: it could synthesize all the others, and all the others would eventually break free from history. The history play was the site of this first experiment in dramatization by a literary culture that made its energy felt at an extremely history-oriented time. While it is easy to point out that genres meant little to early modern writers, before their dramas explored the limits of history the style of early modern English plays possessed little heterogeneity and did not engage in the sort of intertextuality that could allow a literary culture to generate identifiable categories, whether to imitate or subvert them, and all art tries to do both within its self-selected bounds.

The generic labels of most sixteenth century plays were, in fact, extremely fluid and often have very little descriptive purchase on the works they attach to. In many cases, the titles themselves were entire discursive sentences, perhaps invented by printers rather than authors in order to more securely confirm their copyrights. The titles by which we know most of Shakespeare's plays, for example, are condensed from far lengthier originals. In the earlier plays of the sixteenth century, ``interlude'' comes up often in the Stationer's Register to describe a work. ``Comedy'' is used just as frequently. ``Tragedy'' appears sometimes, and even ``tragical comedy,'' a genre more usually associated with Jacobean plays, occasionally pops up. \emph{Tamburlaine} was originally identified on its title page as ``two tragicall discourses'' though it is entered in the Stationer's Register as ``the two commicall discourses of Tomberlein the Cithian Shepparde'' (Greg 171). Even plays published as histories might only qualify if one took ``history'' merely to mean ``story.'' Shakespeare's play \emph{The Taming of the Shrew} was published in 1594 as ``A Pleasant Conceited Historie.'' The flurry of historical dramas that followed Marlowe's blockbuster are described in similarly elusive terms, but there is no denying the critical mass they represent. If ``history play'' did not achieve the status of a recognized genre until the First Folio---and even there the designation is arbitrary---there are nevertheless clear signs that playwrights were seeking to emulate Marlowe's successful formula, whether they regarded their works as histories or histories with tragic overtones, as plays depicting the falls of kings would have. And there are clear signs that the history play was at the forefront of both dramatic innovation and historical inquiry.

Far more plays were probably performed than published, but even those for which we have publication records indicates a swell in the production of dramatic historiography---performances and publications both---at the end of Elizabeth's reign: \emph{1 \& 2 The Troublesome Reign of King John} (1591); \emph{The Famous Chronicle of King Edward the first, sirnamed Edward Longshankes} (1593); \emph{The Life and Death of Jacke Straw} (1594); \emph{The First part of the Contention betwixt the two famous Houses of Yorke and Lancaster} (1594) (later \emph{The Second part of King Henry the Sixt,} though not known as such until the First Folio and originally appearing before the ``first'' part); \emph{The Wounds of Civill War} (1594), a play by Thomas Lodge about Marius and Sulla---Roman plays, as well as histories set among Turks, Persians, and Greeks were interspersed among the English histories; \emph{The True Tragedie of Richard the third} (1594) (not Shakespeare's); \emph{The troublesome raigne and lamentable death of Edward the second} (1594) by Marlowe; \emph{Locrine} (1595), falsely attributed to Shakespeare; \emph{The true Tragedie of Richard Duke of Yorke} (1595), printed in the First Folio as \emph{Henry the Sixth, Part Three}; \emph{The Raigne of King Edward the third} (1596); \emph{The Tragedie of King Richard the second} (1597); \emph{The Tragedy of King Richard the third} (1597), this one Shakespeare's and also demonstrating, as would the two versions of \emph{Lear}, how familiar ground could be trodden and retrodden again; \emph{The Hystorie of Henry the fourth} (1598), or\emph{ Henry IV, Part One}; \emph{The Famous Victories of Henry the Fifth} (1598); \emph{The Scottish Historie of James the fourth} (1598); \emph{The First and Second partes of King Edward the Fourth} (1599), possibly by Thomas Heywood and reprinted many times; \emph{The Chronicle History of Henry the Fift} (1600), another Shakespearean remake, reduced to a pithy ``Life of'' in the First Folio; \emph{1 Sir John Oldcastle} (1600), also attributed to Shakespeare; \emph{The Second part of Henrie the fourth} (1600); \emph{The True Chronicle History of the whole life and death of Thomas Lord Cromwell} (1602); and, of course, \emph{The Tragicall Historie of Hamlet Prince of Denmarke} (1603). With the accession of King James, the so-called tragicomedies, city plays, satires, romances, and masques began to be more popular. Though the clustering of ``history'' plays evident in the 1590s smooths out, the occasional edition, sometimes more biographical than historical in a broad sense, manages to achieve publication in the years leading up to Shakespeare's death: \emph{The Tragicall History of D. Faustus} (1604); \emph{When you see me you Know me, Or the famous Chronicle Historie of king Henry the eight} (1605); \emph{The True Chronicle History of King Leir} (1605), \emph{1 If you Know not Me you Know Nobody. Or, The troubles of Queene Elizabeth} (1605) and reprinted many times; \emph{The Second Part of Queene Elizabeths troubles} (1606); \emph{Nobody and Somebody. With the true Chronicle History of Elydure, who was fortunately three severall tymes crowned King of England} (1606); \emph{The Famous History of Sir Thomas Wyat} (1607); \emph{M. William Shakespeare: His True Chronicle Historie of the life and death of King Lear and his three Daughters} (1608); and a play called \emph{Troia-Noua Triumphans} (1612) indicates the legend of Britain's Trojan origins still had some currency even at the relatively late date of its publication. In addition, Sir William Alexander was able to publish three times (in 1604, 1607, and 1616) a volume of \emph{Monarchic Tragedies}, a title suggesting the tragic valence of so much of what passed for history.\footnote{William Stirling} In its latest edition, this book contained the plays \emph{Croesus, Darius, The Alexandrean Tragedie,} and Shakespeare's \emph{Julius Caesar}.

The sheer number of history plays produced at the end of the sixteenth century and the proportion of them that found their way into print indicate how much and how rapidly the Elizabethans were developing a robust historical consciousness and how far that consciousness penetrated the populous at large. Though many of Shakespeare's plays were first printed in quarto, the histories seem to have gotten special attention. The entirety of both tetralogies were published, as were numerous other plays with historical pretensions. The bulk of historical plays published at the end of Elizabeth's reign is testament to an interest that was not confined to a tiny, educated vanguard but that permeated society. The dramatic form, appealing as it did to the lettered and unlettered alike, was perhaps more effective than any other form of history at both communicating an interpreted body of facts and making them resonate with audiences. Plays were, in one sense, adaptations of chronologies, but they also opened up a new role for history in the consciousness of society. Brian Walsh describes how ``theatrical performance emerged as a unique locus of historical work. In a sense, history plays were parasitic on written histories, but they simultaneously broke from those sources to enable new modes of historical presentation, conjecture, and interpretation''~\cite[121]{walsh_unkind_2004}. What passed for history, of course, was often itself parasitic on other histories and on the rhetorical heritage that history shared with poetry. One of the inherent strengths of drama for the presentation of history in early modern England was its concurrent participation in the cultures of performance and print. The playhouse in early modern London set a formal constraint on the dramatic writing of the period as surely as rhyme schemes did to sonnet sequences. The conditions of performance fashioned the composition of plays, a transaction at its most basic level between authors and acting companies on the one hand and audiences on the other, much as the no less social and material conditions of manuscript circulation lay behind even the most seemingly abstract of lyrics. The content that was poured into the theatrical edifice, however, could be far more variegated than that generally allowed in the more metrically policed verse of the period. Whereas formal experimentation in poetry strictly speaking---including much of Spenser's work, the quantitative interludes in Sidney's \emph{Arcadia}, and the songs of Thomas Campion---typically adhered to an exacting, sometimes algorithmic standard inspired by classical models, a play was freer, even if not completely liberated, to attempt the condensing into a unity of contents expounded in all the forms of literary writing then available, simultaneously shaping ``the forms of things unknown'' while having the resources of a rich and evolving poetics from which to draw. With neither rigorous theories nor exclusive coteries of taste to restrict it, early modern drama was free to continuously refresh and reinvent itself.

The process of converting history into drama is a poetic process. A history play is not simply a play in the genre of history but belongs to a larger category that articulates in a particular mode what a specific episode in history means. In general, history must have seemed an ideal source for dramatic plots. Chronicle accounts provided an interweaving of dull facts with famous incidents, particularly the wars and usurpations that featured so largely in Shakespeare's history plays. They also provided interesting, celebrated characters and speeches alleged to have been delivered by them. In the Elizabethan period, which saw the emergence of the historical drama, when the chronicles were converted by poetry into a fiction, tragedy was often the result. \emph{Gorboduc}, the play usually held to be the first English tragedy, written and first performed at the very beginning of Elizabeth's reign, was based on the reign and deposition of one of England's mythic early kings. Significantly, one of its authors, Thomas Sackville, also contributed to the other great collection of tragic tales of that time, the \emph{Mirror for Magistrates}. As a kind of a dramatic analogue, the deposed rules of the \emph{Mirror} speak directly to us in a manner resembling a play, if not exactly respecting the fourth wall. Certainly, tragedies were as popular as histories in the late sixteenth century, due in no small part to the newly available works of Seneca, who exerted so much influence on early English tragic form. The two ``genres'' would soon fuse and then go on to appropriate every other style of writing that could be incorporated into a dramatic rendering of history, into something that could bring personality, motive, and social meaning to otherwise stale accounts.

That history is home to a series of ``actors'' conforming to simplified types, either virtuous or villainous, was only a starting point for Shakespeare. In the theater, he could literally flesh out the slight adumbrations of history offered by chronicles and their versified reductions in order to develop, both telescopically with regard to time and expansively with regard to psychology and motive, what was in effect an interpretation. He could not have anticipated the influence his interpretations would have on subsequent generations, but it is not insignificant that he was writing for an increasingly literate audience increasingly interested in the elaboration of England's history. And that history was only then being manufactured, a nation invented from the patrimony of numerous peoples almost always in conflict. The new style of writing of histories that became such a vital influence on the Elizabethan sensibility could at the same time justify the traditional structure of that society, support its political and religious innovations, and question the authenticity of such authoritarian structures merely by making them all subject to a poetic act. What Shakespeare and his fellow dramatists did was bring historiography into collaboration with its natural interpreter: poetry. As so much history had provided matter for poetry in the past, it was only natural that the poetic energies of the early modern theater first apply themselves to the most fitting supplier of the theater of the world.

The conceit of \emph{theatrum mundi}, that the world is a theater that enables mankind to perform a plenitude of roles, was a commonplace in early modern Europe and found its way into philosophical as well as imaginative writing. There is, of course, Giovanni Pico della Mirandola's famous address by God to Adam: \begin{bq}
Thou, like a judge appointed for being honorable, art the molder and maker of thyself; thou mayest sculpt thyself into whatever shape thou dost prefer. Thou canst grow downward into the lower natures which are brutes. Thou canst again grow upward from thy soul's reason into the higher natures which are divine.~\cite[5]{mirandola_dignity_1998}
\end{bq}
A century after Pico wrote his encomium, the idea still had life, though its inherent optimism had darkened in the shadow of a divided and embattled Christendom.\footnote{See Bouwsma, 112ff.\nocite{bouwsma_waning_2000}} By the time of Shakespeare's Renaissance, a Renaissance we identify in retrospect because he lived in it but that was, at the time, a period of political instability and economic hardship hardly identifiable with the late medieval prosperity of Florence and Venice, ``man'' had become a contemptible worm. The French humanist Pierre Boaistuau, writing in his own extremely successful \emph{Theatrum mundi}, saw humanity's place in the theater of the world as fallen from whatever virtue it might once have possessed:
\begin{bq}
If we will consider man in the first estate that God created him, it is the chief and principal of Gods work, to the end that in him he might be glorified as in the most noblest and excellentest of al his creatures. But if we consider him in the estate of the general corruption spred all ouer the posterity of Adam, wee shall see him nooseled in sinne, monstrous, fearefull, deformed, subiecte to a thousande incommodities, voide of beatitude, unable, ignorant, variable, \& an hypocrite. To be short, in steade of being Lorde of all creatures, he is become slaue to sinne in the which he is borne and conceyued.~\cite[285]{boaistuau_theatrum_1574}
\end{bq}
The new, divine human being, ``the molder and maker of thyself'' optimistically announced in Pico's oration, becomes a degraded wretch in Boaistuau's estimation, free only to suffer the indignity of a fallen nature. The terrible price of the freedom of self-invention promised by humanism was for such writers the unleashing of the worst part of the human personality. An unmoralized history, decoupled from a providential vindication, was no guarantor of justice and offered no hope for the future. This attitude, which pervaded the literature of the time, also influenced the writing of plays and no doubt contributed to their enduring popularity, as this melancholia about the human condition would come to be marked as distinctively modern. But it was not necessarily a lesson that everyone cared to see taught. And the theaters, being places of public assembly, had a great deal of cultural power, enough for them to be perceived as threats to social order as well as public morality. The threat that drama would deviate from the traditional didactic role of poetry and show mankind as it really was and not as it could be is perhaps why some of Shakespeare's contemporaries were so aghast at what the theaters represented: a mirror of the worst vices of a depraved humanity, offering no simple lessons on proper Christian conduct and instead secreting in Machiavelli by the back door. This suggests a rather naive view of the theatrical audiences of the time, who were part of, according to Kevin Sharpe, ``an educated interpretive community, fully able to read visual symbols as encodings of values and virtues and of privilege and power''~\cite[359]{sharpe_selling_2009}. The usual, practical complaint was about the idleness of people who spent their days at the theater instead of pursuing more useful activities.
\footnote{Though it is perennially popular to condemn the idleness of the poor, it is likely that the lack of industriousness among the Elizabethans had as much to do with the economic conditions of sixteenth century England as with any sort of culturally ingrained laziness:
\begin{bq}
Wages were so inadequate that productivity was probably impaired by malnutrition. From a quarter to a half of the population lived below the level recognized at the time to constitute poverty. Few of the poor could count on regular meals at home, and in years when the wheat crop failed, they were close to starvation. It is not surprising that men living under these conditions showed no great energy for work and that much of the population was, by modern standards, idle much of the time. See Morgan, 602.\nocite{morgan_labor_1971}
\end{bq}
Plenty of time was therefore available for attending the theater and becoming absorbed in the historical narrative it created.}
Between a chapter on ``Scurrility or Scoffing'' and one on ``Cruelty,'' William Vaughan wonders Plato-like ``Whether Stageplayes ought to be suffred in a Commonwealth?'' Clearly, not everyone regarded the theater as a valuable contribution to a burgeoning new cultural scene. Vaughan thought them a waste of time that could never be recovered:
\begin{bq}
men spend their flourishing time ingloriously and without credit, in contemplating of plaies. All other things being spent may be recouered againe, but time is like vnto the latter wheele of a coach, that followeth after the former, and yet can neuer attayne equally vnto it.~\cite[N. pag. Book 1, Ch. 51]{vaughan_golden-groue_1600}
\end{bq}
Later in the same work, however, he follows convention by referring to poetry as the chief civilizer of heathens whereas stage plays he condemns for having in antiquity been dedicated to Bacchus. Drama was not, to him, poetry since it did not achieve the same ends poetry was meant to. Thomas Nash, on the other hand, found much good in the plays of the time, particularly the history plays so popular at the time he was writing:
\begin{bq}
To this effect, the pollicie of Playes is very necessary, howsoeuer some shallow-braind censures (not the deepest serchers into the secrets of gouernment) mightily oppugne them. For whereas the after-noone beeing idlest time of the day; wherein men that are their owne masters, (as Gentlemen of the Court, the Innes of the Courte, and the number of Captaines and Souldiers about London) do wholy bestow themselues vpon pleasure, and that pleasure they deuide (howe vertuously it skils not) either into gameing, following of harlots, drinking, or seeing a Playe: is it not then better (since of foure extreames all the world cannot keepe them but they will choose one) that they should betake them to the least, which is Playes? Nay, what if I prooue Playes to be no extreame: but a rare exercise of vertue? First, for the subiect of them (for the most part) it is borrowed out of our English Chronicles, wherein our forefathers valiant acts (that haue line long buried in rustie brasse, and worme-eaten bookes) are reuiued, and they themselues raised from the Graue of Obliuion, and brought to pleade their aged Honours in open presence: than which, what can be a sharper reproofe to these degenerate effeminate dayes of ours.~\cite[N. pag.]{nash_pierce_1592}
\end{bq}
Nash's defense of plays is so interesting because his premise is the same as Vaughan's, that poetry should be an inculcator of virtue, but he includes the depiction of ``our forefathers valiant acts'' as having the same worth and at least diverting the unoccupied gadders about town from even more debauched activities. Whether he is naive enough to truly regard all his forefathers' acts as valiant is hard to say, but he does see merit in rescuing them from the obscurity of unread chronicles and at least presenting them for public discrimination. Perhaps he simply has more faith in audiences to make something of their history, valiant or not, and in the dramatic arts to interpret them to their profit.

Readers vs. audiences (making a case for drama)

What made early modern plays so unique and culturally resonant a medium was their simultaneous existence as texts to be read and performances to be witnessed. Though materially ephemeral, many of Shakespeare's play scripts, in particular, enjoyed multiple printings and circulated among a literate and learned readership. And despite Shakespeare's famous lack of interest in publication, ``in his own age more editions of his plays circulated than of any other contemporary playwright''\cite[21]{kastan_shakespeare_2001}\footnote{This case, however, should not be overstated. Peter Blayney's examination of publication records and sales figures has shown that the supply of printed plays generally exceeded the demand and selling them was hardly a road to riches: ``No more than one play in five would have returned the publisher's initial investment inside five years''~\cite[383--422]{blayney_publication_1997}.} This fact alone would seem to make of Shakespeare, if he were not already, a special case. Certainly he may be, or his company was, or else his works simply mark the beginning of an increased interest in published plays that would continue to increase throughout the seventeenth century, culminating in their elevation into subjects worthy of editorial emendation and commentary, themselves. Whether qualities of his writing or his choice of subject matter---and the popularity of history plays---made his works stand out, we can hardly determine. As with much else in the early modern period, it is difficult to do much more than identify certain traces which would later crystallize into identifiable cultural practices. By the time the First Folio was published, however, its editors felt confident enough about both Shakespeare's legacy and the dignity of drama that they could urge its purchasers to ``Reade him, therefore; and againe, and againe: And if then you doe not like him, surely you are in some manifest danger, not to vnderstand him''~\cite[sig. A3r]{heminge_great_1623}. To the literate public of the seventeenth century, this may have seemed an incredible statement, to some even scandalous, that a book of plays should be set alongside other monuments of literary culture. Are Heminge and Condell really suggesting that plays deserve the same respect, to be read not just once but repeatedly and intently, that there is something about them that is worthy to be understood, something that would reward intense concentration and careful study? This is hardly the sort of reading practice associated with ephemera, just as a folio is hardly the form in which one would have expected ephemera to be published. What the status of Shakespeare's plays as texts to be read indicates for us is their manifest participation in the larger literary culture of the period. They were not separate from that culture for being a sometime vulgar entertainment but drew from it and contributed to its elaboration, as well. At the time Shakespeare was active, the national history of England was a foremost concern of the literary vanguard. The fictional, poeticized histories he made, therefore, have some legitimate claim to bearing out Hamlet's otherwise flip statement. But in what ways might Shakespeare's dramatic historiographies have been appreciated by their readers? In what ways were these texts used?

In their oft-cited discussion of the ``goal-directed'' reading practices of the Renaissance, Lisa Jardine and Anthony Grafton argue that early modern readers actively reinterpreted their texts to suit specific circumstances. The methods of such readers involved more than the silent appropriation and subjective scrutiny intrinsic to reading as we understand it. Jardine and Grafton extend the notion of active reading from the field of mental play into the world of practical affairs, finding that the ``activity'' of Renaissance readers included ``not just the energy which must be acknowledged as accompanying the intervention of the scholar/reader with his text, nor the cerebral effort involved in making the text the reader's own, but reading as intended to \emph{give rise to something else}''~\cite[30]{jardine_studied_1990}.\footnote{See also Sherman.\nocite{sherman_used_2008}} The most useful texts for would-be councilors of state were, not surprisingly, the historical accounts of antiquity, which supplied ready bullion for scrupulous readers to stamp into modern currency. Thomas Blundeville, for example, dedicated \emph{The true order and methode of wryting and reading hystories} to Robert Dudley, the Earl of Leicester, praising him for his love of history and his desire ``to gather thereof such iudgement and knowledge as you may therby be the more able, as well to direct your priuate actions, as to giue Counsell lyke a most prudent Counseller in publyke causes, be it matters of ware, or peace''~\cite[sig. A2r]{blundeville_true_1574}. Likewise, William Camden dedicated his \emph{Britannia} first to Elizabeth and then later to James: grand volumes of history were appropriately aimed at history's most visible actors and likeliest students. While Jardine and Grafton restrict their scope to the reading-influenced maneuvering of Elizabethan grandees---using as a case study the efforts of Gabriel Harvey to trade on his own interpretive prowess---the uncanny power of reading they identify, that it can ``give rise to something else,'' also bears crucially on the dramatic texts of the period, out of which so much could be made by so many. Indeed, though their concern is more for the reading practices of scholars, they aver that ``even in the realm of popular culture, a variety of kinds of reading were understood to take place, and such readings were not sealed off from more ``serious'' and ``educated'' encounters with the written word'' (32).

The goal-directed, scholarly reading of history might have been oriented toward some specific policy or outcome, but toward what goal are the ``readings'' of the past found in the history plays oriented? Propaganda? Allegory? Abridgment? Critique? And what were the unforeseen outcomes of exposing such a text to the judgment of the public? What did that give rise to? Ancient works in Latin and Greek could be put to individual uses by individual readers, but the dramatic works of the English Renaissance had larger, more differentiated audiences. On the one hand, they were performances tailored to exhibition at court, in the city theaters, and in the courtyards of provincial inns. The aristocratic patrons of the acting troupes could sponsor performances for reasons of politics or taste, and the troupes themselves might try to capitalize on literary fashions or current events. On the other hand, plays comprised a published genre of literature in their own right, consumed perhaps as amusing ephemera, perhaps as legitimate reading matter complete with paratexts and glosses even before they appeared in prestige, folio editions. They were, at the same time, works that could be pored over and enjoyed as isolated objects and as events to be experienced in the moment and deeply immersed in a social, cultural, and political context. These play texts captured in perdurable form what otherwise would have been transitory cultural moments, transforming highly localized occasions into a body of printed literature that could circulate with and play against the products of higher intellectual echelons.  The frequently-published history plays, in particular, were thus woven into the quilt of the period's historiographical discourse as surely as any chronicle or chorography.

As the inheritors of a literary canon so much influenced by Shakespeare, we take for granted the legitimacy of early modern drama as an art form deserving of serious study. That one might read a play text in lieu of attending a performance might not strike us as particularly exceptional, but it would have been baffling to much of the readership of sixteenth century England. We should consider how dramatic writing transformed itself from the ephemeral substrate of occasional entertainments into a legitimate genre of writing that was not only worth publishing but worth reading intently, independently of staged productions, alongside more ``serious'' sorts of reading matter. The change in fortunes for drama was not entirely a slow process of elite recognition combined with gradual cultural acceptance, though. Even as they were writing and producing their plays, early modern dramatists fought for their art and promoted it vehemently. They were not satisfied with clinging to some tertiary role in the backwaters of cultural production. Using all the resources of wit and learning at their disposal, they positioned themselves at the vanguard of the period's literary flourishing. However much their contemporaries might have smiled at such presumption, hardly a generation passed before their battle was all but won and the very best of their plays were appearing in handsome folio editions complete with the imprimatur of engraved portraits and dedicatory poems. Lukas Erne argues that the printing of plays (and, according to him, Shakespeare's direct hand in it) had as much to do with their rather sudden legitimation as printed lyric poetry had on the ``formation of Elizabethan poetic taste and practice'' (Erne 33). I would only extend his point by adding that Shakespeare's history plays became, at the same time, a legitimate source of historical knowledge. Not everyone, of course, subscribed to this position. As the famous reprimand of Thomas Bodley, founder of the eponymous Oxford library, to his agent Thomas James indicates, plays were not universally considered to belong among the best that had been thought and said:
\begin{bq}
There are many idle books, and riff-raffs among them, which shall never come to the library, and I fear me that little, which you have done already, will raise a scandal upon it, when it shall be given out by such as would disgrace it, that I have made up a number with almanacs, plays, and proclamations: of which I will have none, but such as are singular.~\cite[219--222]{bodley_letters_1926}
\end{bq}

Of course, this might only represent a minority view. As we shall soon see, more than one noteworthy scholar-writer of the time found plays, if not all of them, perfectly acceptable as edifying reading material. Indeed, what would, for Bodley, have made a play ``singular'' enough to gain admission we can only guess at, though certain plays did make it into the collection. In a realm scant of evidence, however, the Bodleian provides a useful index to their broader cachet. When standards finally shifted in drama's favor, they appear to have done so decisively and sooner than we, and Bodley, might have expected. Stephen Orgel finds the time lag, as reflected by the Bodleian's purchases, remarkably swift:
\begin{bq}
The library's accounts for 1623 record the purchase of a copy of the Shakespeare folio in unbound sheets, and an order for a special binding with the university's arms stamped on the cover. The library was, indeed, the first purchaser of whom we have a record---this order constitutes our earliest evidence that the book was actually in existence. It required only eleven years, and more important the publication of the folio, for Shakespeare's plays to become suitable reading for Oxford's scholars.~\cite[5]{orgel_authentic_1988}\footnote{I cannot find any evidence for a 1623 purchase by the Bodleian of a First Folio. Since the library had an arrangement to receive all books registered with the Stationer's Company gratis, it is unclear why an extraneous purchase would have been necessary. Furthermore, there is no listing for Shakespeare in the Bodleian Catalogues of 1603 or 1620. The 1623 Folio does appear in the Supplemental Catalogue of 1635 but was apparently discarded before the publication of ``Hyde's Catalogue'' in 1674 in favor of the Third Folio of 1664. In fact, the Bodleian would not come to possess another copy of the First Folio itself, which originally bore no intrinsic value (quite the opposite with subsequent, ``improved'' editions regularly appearing) until it incorporated the Malone collection in 1821. While many books seem to have suffered the fate of supersession by newer editions, it seems to me unconvincing that the works of dramatists were already beginning to be valued in the first few decades of the seventeenth century as highly as Orgel suggests. See the overlapping accounts in William Dunn Macray, 52\nocite{macray_annals_1890} and Robert C. Barrington Partridge, 21.\nocite{partridge_history_1938} The almost identical language indicates that Barrington was likely working from Macray, though he contributes a useful discussion on what the history of legal deposit says about the valuation of different genres at the time.}
\end{bq}
That may be going too far, as Gerald Bentley believes that ``the increased dignity which the appearance of the Jonson and Shakespeare folios brought to plays and playwrights must be seen only as a rise from an exceedingly low status to a moderately low one''~\cite[57]{bentley_profession_1984}.
The demand for players at court, see note 5 in Peter Thomson, Shakespeare's Theatre, pg. 23? But they rose to the challenge. In Shakespeare's own work, we have to search for indications of assertive apologetics on behalf of drama.

``where thousands spend the moitie of the day, the weeke, the yeere in Play-houses, at least-wise far more houres, then they imploy in holy duties, or in their lawfull callings. If we annex to this, the time that divers waste in reading Play-bookes, which some make their chiefest study, preferring them before the Bible, or all pious Bookes, on which they seldome seriously cast their eyes; together with the mispent time which the discourses of Playes, either seene or read, occasion: and then summe up all this lost, this mispent time together; we shall soone discerne, we must needs acknowledge, that there are no such Helluoes, such Canker-wormes, such theevish Devourers of mens most sacred (yet undervalued) time, as Stage-playes.''~\cite[307]{prynne_histrio-mastix_1633}

%%%%

Given the extent of their publication and how often the theaters were closed, it may even be that the theater of the mind frequently had to serve, for those with the capacity, as an important substitute for the theater in the round. (See Lukas Erne, Leeds Barroll) Could the demand for drama and the scarcity of performances together have encouraged the market for printed plays and thereby their eventual elevation into artifacts of both national pride and careful study?

Shakespeare's plays might even have been read more often than performed, as the theaters were closed due to plague or politics more often than not, particularly during the Jacobean period, and that is not even including the famous plague of 1601-3 (?) when he is believed to have composed his sonnet sequence.\nocite{barroll_politics_1991} See Leeds Barroll, 172ff.

%%%%

Whatever these historical dramas gave rise to lay at the intersection of their adaptation of historical discourse  and their own reception and appropriation. There is, of course, always the potential for semantic struggle, resistance, and misprision between the writing and reading of a text. In the case of drama, ``reading'' includes a communal component not entirely dissimilar to the public recitations of texts, devotional and otherwise, typical of the period. The occasion of the public performance is an opportunity for group identification and a collective shaping of response that is largely absent from the act of reading alone. The silent reflection of the solitary reader can, in a public setting, be either amplified or attenuated. The unreliable and suggestible nature of audiences was enough for Ben Jonson, in the ``induction'' of \emph{Bartholomew Fair,} to record the appearance on stage at the Hope theater in 1614 of a ``Book-holder'' (prompter) and a ``Scrivener'' with ``articles of agreement'' according to which the audience, provided they remained for the duration of the performance, was contracted to agree
\begin{bq}
that euery man heere, exercise his owne Iudgement, and not censure by Contagion, or vpon trust, from anothers voice, or face, that sits by him, be he neuer so first, in the Commission of Wit: As also, that hee bee fixt and settled in his censure, that what hee approues, or not approues to day, hee will doe the same to morrow, and if to morrow, the next day, and so the next weeke (if neede be:) and not to be brought about by any that sits on the Bench with him, though they indite, and arraigne Playes daily.~\cite[sig. A5v]{jonson_bartholmew_1631}
\end{bq}
As a playwright, and notwithstanding the withering criticism he suffered late in his career, Jonson must have been particularly sensitive to the unpredictability of a text's reception. In private, a reader's thoughts have a better chance of passing over a text uncontaminated by the opinions of others. During a public performance, a general opinion is more likely to sway individual sentiment. Even worse, the reaction to a performance might influence a play's reputation once it is printed, affecting its chances to withstand comparison to its literary peers. Moreover, in the case of historical drama per se, the text is received against an existing discourse that is itself already a significant part of public intellectual exchange.

``The emergence of the author did not, as traditional narratives have suggested, coincide with the development of private and passive reading habits; instead, authors established their authority by invoking readers who would participate directly in their texts''~\cite[22]{dobranski_readers_2005}

Heywood, one of the greatest popularizers of historical knowledge, fed into many of his plays the knowledge that informed his chronicles. 

\begin{bq}
Do not the Vniuersities, the fountaines and well springs of all good Arts, Learning and Documents, admit the like in their Colledges? and they (I assure my selfe) are not ignorant of their true vse. In the time of my residence in \emph{Cambridge}, I haue seene Tragedyes, Comedyes, Historyes, Pastorals and Shewes, publickly acted, in which Graduates of good place and reputation, haue bene specially parted: this is held necessary for the emboldening of their \emph{Iunior} schollers, to arme them with audacity, against they come to bee imployed in any publicke exercise, as in the reading of the Dialecticke, Rhetoricke, Ethicke, Mathematicke, the Physicke, or Metaphysicke Lectures, It teacheth audacity to the bashfull Grammarian, beeing newly admitted into the priuate Colledge, and after matriculated and entred as a member of the Vniuersity, and makes him a bold Sophister, to argue pro et contra, to compose his Sillogismes, Cathegoricke, or Hypotheticke (simple or compound) to reason and frame a sufficient argument to proue his questions, or to defend any axioma, to distinguish of any Dilemma, \& be able to moderate in any Argumentation whatsoeuer.~\cite[N. pag.]{heywood_apology_1612}
\end{bq}

History and drama reinforced one another as appropriate reading matter for as long as they were held to possess similar virtues, until the dramas themselves became historical documents, privileged components of a literary-historical culture they themselves helped to promulgate, which finally came itself to reward study and understanding both as antiquities of the nation and still for the value they have in their own right, a value some few saw too at the time of their original devising.

Sallust:
``Historie ought to be nothing but a representation of truth, and as it were a Map of mens actions, sette forth in the publicke view of all commers to bee examined; And therefore the predescanting opinion of the writer cannot but bring much discredite to the Action, in that hee presumeth to prepossesse the minds of Artists with imaginarie assertions, seeming to teach those, who knew better then himselfe what belongeth to such affaires, to the wiser sort, who will not he deceiued (for that he cometh to Counsel before he be called) he seemeth verie suspitious.''~\cite[N. pag.]{heywood_choice_1608}

This is a long way from \emph{A Mayden-Head Well Lost}.

The Iron Age:
``I presume the reading thereof shall not prooue distastfull vnto any: First in regard of the Antiquity and Noblenesse of the History: Next because it includeth the most things of especiall remarke, which haue beene ingeniously Commented, and labouriously Recorded, by the Muses Darlings, the Poets: And Times learned Remembrancers; the Histriographers.''~\cite[N. pag.]{heywood_reader_1632}

% Used books + using history + plays as books + performance also used

Performance

Richard Schechner, who locates the English Renaissance theater at the chronological intersection of the decline of ``efficacious'' performance in the medieval mold and the rise of popular entertainment~\cite[123]{schechner_performance_1988}. describes a communication circuit of his own: ``To some degree the theater is the visible aspect of the script, the exterior topography of an interior map. Performance is the widest possible circle of events condensing around theater. The audience is the dominant element of any performance. Drama, script, theater, and performance need not all exist for any given event. But when they do, they enclose one another, overlap, interpenetrate, simultaneously and redundantly arousing and using every channel of communication''~\cite[91]{schechner_performance_1988}. Diagrammed on pg. 72

The performance text
The varying influences of plays performed versus plays published is difficult to gauge. Historial dramas offered a unique window onto the historical imagination of the period partly because of the contingencies of their production. Less the products of independent, creative geniuses, they were collaborative improvisations that responded to and shaped the expectations and attitudes of their audiences. As publications, we might best consider the plays as we have received them as snapshots of particular moments of their lives. This is one way of reformulating the problem of multiple editions as one of the strengths of drama as a medium: it captures in its own inconsistent form the inconsistencies of a historiography comprised of multiple, competing perspectives. There is no definitive edition of any play, because definitive editions were not the goal of an acting company accustomed to variations in performance as a matter of course; because publication was out of most authors's hands; but also because the dramatic genre is inherently flexible about the presentation of its subject matter. Those who perform and those who observe had more impact on a dramatic text than readers of books would have, because the ``communication circuit'' in the case of play texts was that much more complex.\footnote{On the communication circuit, see Darnton.\nocite{darnton_what_1990}} It does us little good to argue that one version of a play or another represented an author's original intent or ultimate revision. Even if we could, the conditions of their publication, dissemination, and performance render authorial intention the product of critical transvaluation.

Adrian Johns ``in its modern sense the very concept of an ``edition'' is entirely anachronistic. For books such as the first folio of Shakespeare, not only is there no pair of identical copies in existence, but there is no straightforward way of positing a ``typical'' printed copy against which ``variants'' might be calibrated''~\cite[91]{johns_nature_1998}.

``Shakespeare habitually began with more than he needed, that his scripts offered the company a range of possibilities, and that the process of production was a collaborative one of selection as well as of realization and interpretation''~\cite[7]{orgel_authentic_1988}

Those snapshots, however, 
as they could sit beside more conventional historiographies and offer to readers their own 
The publication of plays created a permanent readership for the cultural contribution of the stage. 
Drama was one of the principal transmitters of historical culture in early modern England: it served as an interface between scholars, playwrights and middle class readers and audiences. Drama created a body of national histories everyone could potentially witness and also participate in, as the playgoing experience was far more participatory than it is now. It was, in fact, just one more form of reading in a culture that embraced diverse reading styles: aloud and communal; silent and solitary; deep and introspective; cursory and selective.

``each show is ``a palimpsest collecting, or stacking, and displaying whatever is, as Brecht says, ``the least rejected of all the things tried.'' The performance process is a continuous rejecting and replacing. Long-running shows---and certainly rituals are these---are not dead repetitions but continuous erasings and superimposings. The overall shape of the show stays the same, but pieces of business are always coming and going. This process of collecting and discarding, of selecting, organizing, and showing is what rehearsals are all about. And it's not such a rational, logical-linear process as writing about it makes it seem. It's not so much a thought-out system of trial and error as it is a playing around with themes, actions, gestures, fantasies, words: whatever's being worked on. From all the doing, some things are done again and again; they are perceived in retrospect as ``working,'' and they ``kept.'' They are, as it were, thrown forward in time to be used in the ``finished performance.'' The performance ``takes shape'' little bit by little bit, building from the fragments of ``kept business,'' so that often the final scene of a show will be clear before its first scene---or specific bits will be perfected before a sense of the overall production is known. That is why the text of a play will tell you so little about how a production might look. The production doesn't ``come out'' of the text; it is generated in rehearsal in an effort to ``meet'' the text. And when you see a play and recognize it as familiar you are referring back to earlier productions, not to the playscript. An unproduced play is not a homunculus but a shard of an as yet unassembled whole.''~\cite[120]{schechner_between_2011}

On multiple perspectives: Annabel Patterson, intro p. 43

A. P. ``Its motives were to make available to the reading public enough of the complex texture of the national history that the middle-class reader could indeed become his own historian---that is to say, a thoughtful, critical, and wary individual''~\cite[8]{patterson_reading_1994}. This is an explanation of all the random information contained in the chronicles, for which see also Patterson on anecdotes. In the chronicle narratives that inspired drama, we get a signature feature of drama: ``the presence of random information and unintegrated source narratives, the chaotic multivocality resulting from the diverse authorial pens'' ``the qualities of this text impose a set of exegetical rules on the readers and compel them to read it with a 'literary' eye that is attuned to rhetorical nuance. Whatever meaning a reader can glean is inevitably derived from the interpretation of those nuances''~\cite[22]{djordjevic_holinsheds_2010}. Shakespeare doesn't use history to propound lessons as much as he propounds a version of history inflected by poetry and leaving thereby the sense up to the critical auditor to determine. Perhaps it was this polyvocal aspect of history that inspired Shakespeare's own quite polyvocal historical plays, that inspired the very polyvocality of drama itself, creating situations it is up to audiences to decipher. And wasn't the revival of drama during the Renaissance part and parcel of the general classical revival happening in other areas? The men of the universities, after all, were the same ones who wrote and attended plays and compiled the histories that both preceded and followed them.

Concluding paragraph:

What Shakespeare inherited from Marlowe was a notion of history that not only questions its providential quality but exposes its relentlessness, its illimitable, interminable inertia in spite of all human struggle or ingenuity. Tamburlaine (considered the ideal virtuous prince by Bracciolini and others) looking at his map, to see how much of the world is left for him to conquer is a pre-figure of Lear, who also asks for a map of his nebulous domain, to see how he might also work his own fiction upon it: Tamburlaine to possess as much as possible, Lear to control its destiny. Or both to control destiny against the implacable logic of history which throws down every conquerer. In the history plays themselves, we see iterations of the cycle. In King Lear, we see the cycle itself, as a fiction, broken and burnt on a wheel of fire. Tamburlaine wants his sons to continue his conquests as Lear desires his daughters to execute his living will. We can only achieve such a reading by reading across the plays instead of only inside them. This note refers to Marlowe's Tamburlaine and how it exhibits the ascendency of Machiavellian virtu instead of Christian virtue.\footnote{See Ribner, 251--266\nocite{ribner_idea_1953}} Is there some transition from the wheel of fortune to the mirror of insight?

As a principle of inquiry, I will not confine myself to the ``history plays'' proper or even read those history plays I do examine as history plays strictly speaking. My interest, instead, is to look closely at a selection of plays incorporating historical matter and identify in each how Shakespeare's invention, his poetic method, organizes its historical meaning. Across his entire body of work, Shakespeare transforms history into poetic fiction. That is, he performs a kind of research by which he assembles dramatic arguments out of the diverse materials available to him. Some of these materials belong to a more explicit category of historiography than others, but it is doubtful he cared to distinguish very much between chronicles strictly speaking and other sorts of ``histories'' such as contained equally unverifiable ``true tales'' from far-removed times and places. Even his more putatively factual sources were not absent of fictitious gestures, whether legendary accounts reported for the sake of completeness or clearly contrived but no less illuminating speeches allegedly delivered by important historical actors. The fictions Shakespeare derived from these sources are a kind of dramatic historiography that is historical insofar as it recapitulates material from a traditional corpus of narratives about the past and poetic for incorporating into its design overt markers of its fictionality. These markers are integral to the construction of these fictions and vital for an understanding of them as species of historical interpretation. Shakespeare uses poetry to propose to his audiences (and readers) alternative meanings in history. He discovers through his process, and discovers to us, a significance beyond anything on the order of causation, the proper sequence of events, or even fidelity to the truth---a significance that could best be articulated in a poetic register. I will consider how Shakespeare used the form of drama to translate history into poetry and test the ways in which his poetic interpretations of history, in dramaturgically diverse circumstances, participated in and worked to enrich the historical imagination of early modern England. Rather than simply identifying and commenting anew on the sources Shakespeare is traditionally supposed to have relied on in composing his plays, however, I propose instead to initiate a reevaluation of the evidence within specific plays of his engagement with the historical culture of which his own historical writing was a significant component.

I want to call attention to three traits, in particular, shared by early modern history with Shakespeare's historical drama: its self-conscious constructedness as literary performance (Richard II), its publicness as the common property of its audience (history is literally staged in a public arena, the counterpoint to public-ation) and as the only available window to expose otherwise private matters, of which history is full (and which Coriolanus himself is reluctant to expose) (Coriolanus), and its anxiety about continuity (King Lear), where we see history plays occurring in cycles that look back as well as forward and no conclusion without resolution (King Lear disrupts this). See Spenser's history with its telling gap during the reign of Arthur. Poetry is well suited to an epic presentation of history, too, as its repetitive forms carry the reader from foot to foot, line to line, or stanza to stanza as fluently as the narrative proceeds from reign to reign. There is a relationship between the unbroken line of descent to the present and the monarch as ``the mirrour of succeeding ages''~\cite[3]{camden_remaines_1605} as William Camden refers to Queen Elizabeth using a metaphor that appears also in Macbeth. Poetical was used as often to mean false or even seditious as poets themselves were enlisted as sources of truth. Edward Hall, the author of that thing, laments that so much of his country's past, for want of writing, is lost to oblivion and dedicates his own work to preventing that from happening in the future: a link from the past to the present, quite in keeping with the Ciceronian idea of history that he happens to quote. 

Literary drama might seem to be located at least one remove from the surer ground of sober historical narrative, but in a culture less interested in outright discriminations between fiction and nonfiction than, to paraphrase Francis Bacon, in judging the suitability of texts to be variously tasted, swallowed, chewed, or digested, a play had as much claim to be thoughtfully ruminated on as any handsomely bound chronicle. This is not to say that the distinction between the factual and the fabulous was either insignificant or ambiguous but that the stakes were different. Moreover, plays---the successful ones---arguably enjoyed a more substantial exposure, being dual-form literary products that were consumed both as printed texts and as live-action performances. As such, they occupied a unique position in the imaginative economy of the period: at the nexus between the visually stimulating, rhetorically provocative stage and the intellectually engaging, theoretically practicable text. Indeed, the play in early modern England perhaps represented in its time an artistic singularity whereby the theater of the senses came closest to correspondence with the theater of the mind. History in the Renaissance merely provided a convenient crucible for this to occur.

In a famous letter dated October 8, 1594 from Henry Lord Hunsdon, Lord Chamberlain of the Household, to Sir Richard Martin, then Lord Mayor of London, Hunsdon requests that his ``nowe companie of Players'' be permitted ``to plaie this winter time wthin the Citye at the Crosse kayes in Gracious street.'' His request comes with the assurance that, now that the plague had passed, his company had ``vndertaken to me that where heretofore they began not their Plaies till towards fower a clock, they will now begin at two, \& haue don betwene fower and fiue and will nott vse anie Drumes or trumpetts att all for the callinge of peopell together, and shalbe contributories to the poore of the parishe where they plaie accordinge to their habilities''~\cite[73--74]{malone_society_collections:_1907}.
This letter has hitherto comprised the entirety of our evidence for the duration of dramatic performances in the early modern English theater.\footnote{See Peter Thomson, 21.\nocite{thomson_shakespeares_1992}} That a Peer of the Realm, cousin of the monarch, and one of the most powerful men in the country had to make formal supplications to a Lord Mayor in order to get his players permission to perform is surprising. More surprising are his assurances that the players will conform to earlier start times, refrain from making noise, and, least plausibly, give to the poor. The performance of plays was at least an important enough issue to spark a power struggle between the Privy Council and a series of Lords Mayor of London that was finally put to rest when playing companies came under the incontrovertible protection of royal patronage.\footnote{For an extensive analysis of this letter and its context, see Andrew Gurr.\nocite{gurr_henry_2005}} Plays had cultural power, and they earned that power rapidly, more rapidly than the intellectuals and commentators of the period could keep up with. Condemn them though they might, they were a form of artistic expression officially sanctioned by the government and popular among the public readership.
