\chapter{The question of Cordelia and her father\label{ch:KingLear}}
The final four lines of \emph{King Lear} are a testament to the narcissism of small textual differences. Assigned to Albany in the First Quarto and Edgar in the First Folio, it has been left to Shakespeare's editors and critics to judge which attribution is more authentic, or---if attesting aesthetic consistency rather than authorial intent---more sensible. In the eighteenth century, the editorial consensus granted the lines to Albany; his higher rank recommended him as the more decorous choice against the authority of the Folio, often dismissed in these early deliberations as an unreliable, players' redaction: ``This Speech, from the Authority of the Old 4to is rightly plac'd to Albany: in the Edition by the Players it is given to Edgar, by whom, I doubt not, it was of Custom spoken. And the Case was this: He who play'd Edgar, being a more favourite Actor, than he who personated Albany; in Spight of Decorum, it was thought proper he should have the last Word''~\cite[219]{Theobald}.
Modern editions, however---those that do not explicitly favor the Quarto or else print both the Quarto and Folio texts separately---prefer Edgar, whether for consistency of characterization or out of fidelity to the Folio. The reasons generally cited in support of Edgar rest on a variety of assumptions: for example, that Edgar has been offered the kingdom by Albany and therefore ought to speak last; that Edgar was the name of a real English king; that Edgar overall is a more important character in the play or develops in such a way as to make this culmination logical. William H. Matchett, in considering this textual variant, even goes so far as to suggest that ``there would be a motive, however mistaken, for someone to change the speech heading, unable to believe that Shakespeare would so far abandon convention as to let anyone other than the highest ranking remnant have the last word''~\cite[208n]{Matchett}.

The problems of speech assignment in the play are part of a much larger debate about the status of the \emph{King Lear} text. For the past few decades, considerable controversy has attended the ``two text'' problem of \emph{Lear}. The critical question is whether the Folio text represents an authentic final version of the play as revised by Shakespeare himself or if the two extant texts, Quarto and Folio, ought to be considered two, separate plays. As with many other early modern dramatic texts existing in multiple versions, conflations that partake of the ``best'' of all available material are the norm. The scholars who produce them are forced to issue verdicts where editorial finesse is insufficient or a choice among irreconcilable alternatives unavoidable. Of course, difficult choices are unnecessary if we treat the Quarto and Folio plays as distinct and complete unto themselves, in which case Albany and Edgar may each conclude their own respective plays uncontroversially. Indeed, there are advocates for this position who argue that the differing ways in which Edgar and Albany are developed as characters in the Quarto and Folio texts support their differing conclusions.\footnote{For the exemplary statement of this position, see Warren.\nocite{Warren}} Opponents of the two texts approach acknowledge the problem presented by the differences between Quarto and Folio (and point out that it is not a new problem) but remain unconvinced that we must treat the two editions as two, distinct plays. There is, after all, no unassailable evidence that Shakespeare participated in a determined process of revision and that, therefore, we should not try to reach an editorial compromise. Attempts to ``teach the debate,'' as it were, have recourse to some impressive and impassioned scholarship,\footnote{See Urkowitz: 1980, Blayney: 1982, Taylor and Warren: 1983, Clare: 1995, Knowles:1999, and Knowles:2008} but the argument that \emph{King Lear} exists in two, unique versions may ultimately seem of small consequence to students of the play not specifically engaged with textual problems. Most of the differences, it could be argued, are not so great as to bedevil attempts at literary analysis. But they are not altogether inconsequential, either. Although the preponderance of scholarly opinion appears to have marshaled against the two texts theory in recent years, the differences between Quarto and Folio \emph{Lear} do pose interesting challenges to close readers of certain moments of the play where the discrepancy---in this case, whether the play's closing remarks properly belong to Albany or Edgar---can seem fraught with interpretive consequences. Those consequences remain hidden if the textual problem is ignored. Philip C. McGuire explores the interpretive possibilities raised by both versions: ``\emph{King Lear} ends differently in each playtext, and what is lost by presenting one ending or the other as definitive is the awareness that each way of concluding the play poses alternatives that, although different, have their own coherence and integrity''~\cite[108]{PhilipMcGuire}. Arriving at a ``correct'' reading is not necessarily required, then, as part of a critical analysis. The very existence of irreconcilable alternatives ought to signal the uncertainty about conclusions---and, as we will see, historical continuity---that \emph{King Lear} brings to the fore. Whether Shakespeare himself revised his playtext (and whether that would put to rest any controversy about meaning), we are left with a work that requires our intervention as readers and spectators on multiple levels, a work that reflects its decidedly inconclusive content in its own, textually-indecisive form.

Shakespeare's King Lear, after the fashion of its titular monarch, has undergone a tortuous progress through the halls of criticism. More than any other play, Lear has been a sounding board for the dramaturgical sensibilities of its subsequent adapters. Though many critics have endorsed it as Shakespeare's definitive tragic masterpiece, they have also found in it much to stir their incredulity, indignation, and horror. Lear, in many estimations, is a nearly perfect piece of dramatic poetry that is nevertheless almost impossible to perform. As with most early modern plays, we know next to nothing about King Lear's contemporary reception (it seems not, however, to have been a popular component of the theatrical repertoire). Only later, after the Restoration, does the opinion of posterity begin its steady compilation. A complete rewriting of the play on behalf of the morally fastidious temper of the neoclassical age was followed by shifts in taste that approved in turn its emotional vitality, its depiction of the journey toward self-awareness and redemption, and even its irrational cruelty. For nearly half its performance history, it was seen only in the revised version of Nahum Tate, who modernized Shakespeare's archaic language, cut out the part of the Fool, and swept from the stage the pile of corpses that "makes many Tragedies conclude with unseasonable Jests,"*7* an alteration that even won Samuel Johnson's grudging approval. In 1788, a yet more restrained interpretation was produced by the actor John Philip Kemble, who further cleansed the language and, in keeping with his puritanical tendencies, found even some of Tate's interpolations too saucy for the stage. It was not until 1838 that Shakespeare's Lear, at the urging of William Charles Macready, who objected to "all the disgusting trash of Tate," was restored to the theater in its original form.*8* If eighteenth-century audiences preferred Tate's decorous denouement of virtue rewarded and the Romantics, revisiting the original text, emphasized the reflection in nature of the characters' passionate self-creation, the postwar sensibility in the twentieth century sympathized with the play's apocalyptic and even absurdist undertones: the idiotic crudeness of human life, the self-assured order of a society weighed on the scales of experience and found wanting. Peter Brook's 1962 production, later turned into a film, is the most often cited example. Brook's spare and devastating staging was inspired by the work of his collaborator, Jan Kott, who declared that "King Lear is a play about the disintegration of the world," a far cry from A. C. Bradley's more optimistic supposition, earlier in the century, that "the business of 'the gods' with [Lear] was neither to torment him, nor to teach him a 'noble anger,' but to lead him to attain through apparently hopeless failure the very end and aim of life."*9* Perhaps only the most sophisticated literary works of the most perdurable relevance can convincingly evoke, for different readers at different times, significations as contrary as nihilistic submission to "hopeless failure" and our indomitable ability to overcome it.
	Diverse reactions to a single text are diversified further when the single text is shown to be manifold. As noted above, renewed emphasis has been laid in recent years on the significant textual differences between the play's two principal versions and the extent to which they provide for irreconcilable readings. Several important characters, from the First Quarto to the First Folio, have their lines augmented, reduced, or even reassigned. In all, there are about 300 lines in  the Quarto that do not appear in the Folio and about 110 in the Folio that are likewise not in the Quarto—a disparity unmatched by any other of Shakespeare's plays.*10* In more than one instance, the alterations bear significantly on the tone of specific scenes and the salient qualities of certain characterizations. A strong case has been made more than once that the two versions of King Lear should be treated as two distinct plays.*11* Beginning with Alexander Pope's edition of 1725, however, editors have endeavored to preserve the most "Shakespearean" material from both Quarto and Folio in conflated editions that have, in turn, supplied raw material to critics. There were good reasons for this practice. Pope himself, in the preface to his edition of Shakespeare's complete works, provides the most elegant defense of the (even then) much-maligned interventions of editors. Since we lack an authoritative edition produced under Shakespeare's own supervision, Pope argued for the necessity of composing the closest possible alternative through a discreet comparison of all the surviving versions. In so doing, he identified many of the issues with which textual scholars and historians of the book still contend:
If we give into this opinion, how many low and vicious parts and passages might no longer reflect upon this great Genius, but appear unworthily charged upon him? And even in those which are really his, how many faults may have been unjustly laid to his account from arbitrary Additions, Expunctions, Transpositions of scenes and lines, confusion of Characters and Persons, wrong application of Speeches, corruptions of innumerable Passages by the Ignorance, and wrong Corrections of 'em again by the Impertinence, of his first Editors? From one or other of these considerations, I am verily persuaded, that the greatest and the grossest part of what are thought his errors would vanish, and leave his character in a light very different from that disadvantageous one, in which it now appears to us.*12*
If Pope is right, and it is incumbent upon a scrupulous editor to correct wrong applications of speeches, there must be some difference between a Lear that ends with Albany and a Lear that ends with Edgar. One or the other must be correct and what is correct must have been Shakespeare's intention—a bit of editorial logic that always seems both circular and irrefutable at the same time. Having to make such a choice, however, inevitably distracts us from what it means to have the choice at all. Rather than choosing the best reading, that is, we might instead reflect on the significance of the problem itself: why it exists and whether it bears significantly on the play and our reception of it beyond possibly being a printing error.	
	How much of a difference, then, does it make whether Albany or Edgar is given the closing speech? A brace of sententious couplets can hardly supply an adequate peroration to a play as abject and tragically profound as Lear. And there is arguably very little a formal summation could even accomplish given what it must follow. In other Shakespearean tragedies, the concluding speech is variously an occasion for the restoration of order and the vindication of authority, a memorialization that effects the reintegration of a society, or the dispensing of rewards and punishments in a tidy transaction of poetic justice. In all these cases, the play ends but gestures toward an imaginary future, a temporality which guarantees the didactic purchase or historical relevance of the moral and political conclusions it reaches. In King Lear, by contrast, none of these situations obtain. Albany/Edgar may portentously adjure the remaining characters to "Speake what we feele, not what we ought to say", but we can imagine nothing that might actually be said in response to what has just happened and no further action, given the play's disruption of historical continuity, that could take place. And this disruption of historical continuity is what makes King Lear so unique among Shakespeare's tragedies and so devastating in its dramatic consequences: any guarantee that Lear and Cordelia will not have died in vain is undercut by the total negation of all subsequent history. There is, quite literally, nowhere for the narrative to go, no "afterward" into which we might project our imaginations and the very existence of which—the promise that human life will continue despite tragedy—ordinarily serves to mollify the shock and horror that tragedy can elicit. When Lear and Cordelia die, we are forced to confront and imagine for ourselves not simply the poetic consequences "internal" to the play, the necessity that Albany or Edgar somehow carry on with the rulership of the kingdom, but the extra-narrative consequences of rewriting what was believed at the time by many to be history. King Lear is not merely a revision or dramatization of a popular story, after all, but a complete, counterfactual revitalization. Among Shakespeare's historical and historically-oriented plays, it is perhaps the most potent example of a work in which the historical is thrown into relief against predominantly literary concerns, rather than the other way around. In Lear, the plot does not operate against a determinate, historical background, that is, but serves to challenge the determining aspect of history itself. The play effects this, ultimately, by modeling for its audience what it means to come to the realization that history might not be an adequate guide to conduct in the present or contain the promise of human salvation in the future. The means by which this modeling is accomplished happens to be, in its own way, quite modest. In prior versions of the Lear story, the king repents his folly, is reunited with his daughter, and dies a happy death. Whatever disasters later befall her, they are none of his concern. In Shakespeare's, however, the king confronted with the bare fact of his own humanity is still forced to watch his daughter die before we, and the rump of characters left alive at the play's end, are forced to watch him die. Lear protests his daughter's execution by slaying her executioner—too late. Albany, Edgar, and Kent protest Lear's passing by fecklessly imagining that all might still be well if Lear resumes the kingship and then, when he is gone, by searching for some moral to the tale. We might protest, in our turn, that this is no way to end a story, with an abuse of history. And so what is essentially a domestic tragedy dilates to become consequential, in the play, for an entire kingdom and then, outside the play, for the conception of history itself. In a way, King Lear redefines tragedy. Though part of a literary tradition of tales of fallen kings, what is tragic about Lear is how the falls of kings can become inconsequential even to kings themselves when confronted with personal, domestic loss and how that loss can be as catastrophic as any world-shaking calamity. Shakespeare, in this play, goes some distance toward equalizing these two levels, polar opposites, by presenting them simultaneously, as impinging on one another—or even elevates the personal at the expense of the historical and political. With the dead Cordelia in his arms and the fate of the kingdom at stake, Lear can barely acknowledge the "more important" issues pressing upon him. If there is some credence to the notion that early modern subjects began to think of themselves as self-motivated individuals, we should certainly expect to find in their literature such a new definition of tragedy as this: tragedy that cuts as unkindly in the private as in the public sphere. By operating in both at once, Shakespeare finds a way to make national history serve an intimate, literary project. This is quite a radical position for a play to have taken: chronologies of events both religious and patriotic loomed large in the early modern mentality. To call into question their legitimacy was practically a challenge to the foundation of the dominant culture. But it is also to acknowledge the significance of daily life, its greater significance, in fact, to issues that can come across as fabulous or abstract, particularly to an increasingly diversified, decreasingly aristocratic theater-going public.
	Of course, our own lives go on after reading or watching a performance of King Lear, but that does not mean Shakespeare did not suppose in his writing the possibility that it might not and offer that supposition as an intellectual exercise, as a challenge for his readers and audiences. The challenge, actually, is in rising to meet such a demand, which requires a reading of the play that is less concerned with ultimate meanings—is it hopeless or redemptive?—than with how Shakespeare triangulates character, plot, and historical substrate in order to communicate specific ideas to his own specific audience. It would be far easier to subsume the tragic in Lear into a tidy and familiar ideological structure. In remarks positing tragedy as a more human rendering of apocalypse, Frank Kermode recognizes the human responsibility that lies at the heart of Shakespeare's conception of tragedy in King Lear but but ignores the historical problem the play presents in order to elevate its ending—and ending itself—into a representative abstraction. In King Lear, according to Kermode:
Tragedy assumes the figurations of apocalypse, of death and judgment, heaven and hell; but the world goes forward in the hands of exhausted survivors. Edgar haplessly assumes the dignity; only the king's natural body is at rest. This is the tragedy of sempiternity; apocalypse is translated out of time into the aevum. The world may, as Gloucester supposes, exhibit all the symptoms of decay and change, all the terrors of an approaching end, but when the end comes it is not an end, and both suffering and the need for patience are perpetual. We discover a new aspect of our quasi-immortality; without the notion of aevum, and the doctrine of kingship as a duality, existing in it and in time, such tragedy would not be possible.*13*
Kermode is perceptive about the tenor of tragedy for a Revelations-minded audience. Eternity, however, and "the doctrine of kingship as a duality" are among the familiar concepts the play challenges. Rather than exhausted survivors going forward into any kind of future whatsoever, what if there were no future except what we supply for ourselves? If the storyline of the Lear episode is assumed to follow what was, in early modern England, a well known chain of events ultimately derived from Geoffrey of Monmouth's Historia Regum Britanniae, Shakespeare's version upends it entirely. Rather than justify such a salient authorial choice or seek an aesthetic consistency within the narrative as it is given, I would argue that Shakespeare's radical alteration of the Lear story, his evisceration of historical continuity, bears considerably on not only how we should read the play but also on Shakespeare's attitude toward history itself. That attitude is certainly hard to pin down given his varied takes on historical material, and we must be cautious about ascribing ideological intent to poetic choices. But the choice to end King Lear as it does evinces a skepticism about history as either the source of useful lessons or an invisible force directing worldly events to some calculated end. The play seems to give us the decision about what to do when human action results in catastrophe. The text cannot even decide who should speak its closing lines, a fitting circumstance of indecision, intentional or not, for a play that concludes so indecisively.*14*
	The choice between Albany and Edgar would appear then to be arbitrary at best, a matter of personal interpretive discretion, if not completely irrelevant. But it is the fact that there is a choice that is important: how to go forward into a non-existent future is a question posed alongside who is to usher us into it. And what about those final lines, so brief a coda to so lengthy a play? Franco Moretti, in contemplating the nature of tragedy itself, finds the "chilling stupidity" of the closing lines of Lear apposite to the circumstances in which they are delivered:
The speech of Edgar is the most extraordinary—and appropriate—of anticlimaxes. Its blind mediocrity indicates the chasm that has opened up between facts and words, or more accurately, between referents and signifieds. The close of King Lear makes clear that no one is any longer capable of giving meaning to the tragic process; no speech is equal to it, and there precisely lies the tragedy.*15*
In a footnote to his own discussion of King Lear, on the other hand, Terry Eagleton refers to this assessment as one of Moretti's "rare blindspots" and regards the lines as indicative of an inexorable reality that lies at the core of human experience. To Eagleton, the closing speech is "no trite tag, denoting as it does that organic unity of body and language, that shaping of signs by the senses, of which Cordelia is representative." He goes on to point out that the play dramatizes an abstract truth about the relationship of language to experience:
The play has also demonstrated that to speak what one feels is no easy business. For if it is structural to human nature to surpass itself, and if language is the very index and medium of this, then there would seem a contradiction at the very core of the linguistic animal which makes it 'natural' for signs to come adrift from things, consciousness to overstep physical bonds, values to get out of hand and norms to be destructively overridden. It is not, after all, simply a matter of reconciling fixed opposites: it is a matter of regulating what would seem an ineradicable contradiction in the material structure of the human creature. King Lear is a tragedy because it stares this contradiction full in the face, aware that no poetic symbolism is adequate to resolve it.*16*
The difference between these two positions is subtle. Either the tragic heart of King Lear is in its identification as tragic the inadequacy of speech to rationalize the chaos of existence; or, it is the function of tragedy as a genre to confront this inadequacy, attempt to resolve it, and inevitably, spectacularly fail. Both Moretti and Eagleton have described, in the abstract, crucial aspects of "the tragic" as a concept and their relevance to what goes on in King Lear. And both come near the mark in identifying tragedy as arising from flawed, human action, from the contradictions in human nature. But such arguments tend to ignore the narrative details of the play itself, as if poetic texture were irrelevant to the contours of a smooth idea: Lear is tragic because tragedy consists in our inability to face up to tragedy, as the characters at the end of this play cannot find an adequate way in which to express the profundity of their trauma. But is such a critical conclusion adequate to the text? The peroratory bromides served up by Albany/Edgar are easy to read as rhetorically empty analogs to Lear's inarticulate howls and exclamatory O's. Language does appear to break down. This failure of language to harmonize with experience, however, is a problem not inherent to the text of King Lear but a problem dramatized by King Lear. Shakespeare, hardly ever at a loss for words to represent anything, has represented here what the struggle to fit words to events looks like—a struggle we might be able to identify with but not a struggle the text of the play fails to live up to itself. Any sort of discourse is surely inadequate to any kind of truth or reality, but it is not the inadequacy of language that is the problem in this play; it is only, perhaps, the inadequacy of present language. Where words fail, our only recourse is more words: a perpetual filling of the void that thrusts toward a future, never to come, in which we might finally have enough of them. Speech does not fail. It simply goes on, searching, modifying, never falling silent simply because it is no better than silence. The silence at the end of King Lear is not so much representative of the failure of language than the start of a lacuna we must fill ourselves, that the play requires us to fill ourselves.
	Quite apart from this sort of linguistic/philosophical analysis, however, is the rhetorical texture of the speech—the fact that it is delivered as a pair of couplets in the fashion of a sententious epigram. In her extended study of such moments of poetic closure, Barbara Herrnstein Smith refers to the affective qualities of epigrammatic conclusions: "In speaking of an utterance, poem, or couplet as 'epigrammatic,' we refer not only to a kind of verbal structure but to an attitude toward experience, a kind of moral temper suggested by that very structure. The epigram seems to offer itself as a last word, an ultimately appropriate comment, a definitive statement."*17*  An effective conclusion along these lines, according to Smith, will come across as if it were inevitable, as if an entire poem, seen retroactively, were heading in its direction. Smith labels this predetermined quality of epigrammatic conclusions "hyperdetermination." In the case of Lear, however, the closing epigram is not "hyperdetermined" in the way Smith describes—it is not as if everything in the play logically leads up to it—and its sententiousness is at least somewhat deflated by its context. But Smith's insights can help us come to terms with its peculiar effectiveness. Rather than providing a satisfactory, moral summing up of the play, Albany/Edgar can only gesture toward doing so. The final four lines do not achieve a hyperdetermination; they only possess the form of it. They are spoken as if the personally devastating, historically disjunctive climax of the play could be overcome using traditional, rhetorical means. In the process, those means themselves are exploded—but not in any way to expose the insufficiency of language or the artificiality of rhetoric (these would be givens as much then as now). Some sort of epigrammatic closure is exactly what we expect, and is what we get, but Shakespeare so overburdens the attempt at closure itself that the couplets have the opposite of their intended, or we might say conventional, effect. That is, rather than neatly bridging the gap between the irrational tragic situation of the play and the moralizing sensibility of the early modern reader or spectator, they open it even wider. In this way, Shakespeare uses the qualities of the epigrammatic conclusion that Smith describes against it, creating a poetic short circuit that stuns our complacency at just the moment when it would normally be assuaged. One might say the lines create a figure of a figure: the rhetorical appearance of a rhetorical device that not only says more than it says but points in one direction only to reach out further in the other. The gentle music of rhyming couplets, that might otherwise lull us, instead stirs our alertness even more. The dead march that follows is not so much the sonic echo of a defeated world resigned to carry on, as it appears, but a signal to the audience that, in the breech Shakespeare creates between the legendary past and the present that had long used it as a foundation, it will have to find closure on its own.
	In fact, the play obviates adequate response as one of its principal dramatic strategies. Its envoi hardly evokes the grand, ritualistic speeches that wrap up so many other plays, except insofar as it undercuts them, and strikes instead a tone of embarrassed, strained solemnity. If there is little indication that Albany/Edgar's speech reinstates a shattered social order, it may be useful to instead consider the audience as the object of address. The boundary between players and spectators in the early modern English theater—like that between texts and readers—was a  notably permeable one, after all.*18* Dramatic illustration of this always-potential permeability is not necessarily limited to soliloquy, metatheatrics, and the comedic aside. In this case, the effect achieved by an elegiac appeal to nobody echoes in sentiment the relentlessly repeated "nothing" that so explicitly distributes nullifications of identity and linguistic efficacy throughout the play. It is Cordelia, however, who initially—or, perhaps, finally—teaches us that speaking what we feel means literally saying nothing when what we are enjoined to speak is framed by a superincumbent rhetoric. In such contexts, and if feeling is unutterable, such contexts may be all there are: a mute affirmation of our self-possession is all that remains of emotional authenticity and is the only way to communicate the ineffable content of "what we feele" outside rhetorical constraint when our "love's more ponderous" than our tongues. Albany/Edgar's empty exhortation is no less constraining, though made in perhaps no less good faith, than Lear's injunction for his daughters to protest their love for him at the beginning of the play. The only difference, and it is the crucial one, is that this final appeal, made as if it is the only available course of action, is spoken into the darkness at the limits of the stage, the edges of recorded history. And response is suspended between the ideal, bracketed world of theatrical time and the mundane world of the audience that demands, with its own form of everyday rhetorical constraint, that we only ever speak what we ought to say. There are no justifications to be made for actions taken or not taken. No one could make any sense, moral or otherwise, of what has happened. And, despite the inclusiveness of the speech's "we," it is not clear who, beyond a playgoing audience, is meant to witness it. Even the fate of Kent, the closest thing to a surrogate for the audience in the play at this point, is uncertain.*19* Among Shakespeare's tragedies, King Lear perhaps depends the most on its audience—a contemporary audience, in particular—to make sense for itself of what has happened and why.*20* This is not least a challenge for readers and editors of the play today, who have textual choices to make that further complicate how we consider its reception.
	The discrepancy between the two texts of King Lear, between one ending and the other, is a kind of textual-material analogue for the evacuating of historical meaning that occurs at the end of the play. We have no ideal Lear text just as we have no ideal ending. King Lear, in its conclusion, breaks free from the constraints of historical inevitability by interrupting the established chronology of England's history. With both texts before us, we are presented with a choice as to who should speak the final lines. With either one, performed or not, we are given the responsibility of somehow resolving the historical impossibility of the ending with the very history out of which the play is excerpted. The choice between Albany and Edgar would appear to be something of a red herring, not because it does not matter in ways internal to the text, but because it distracts critical thought, so often focused on editorial idealizations, from the way the play circumvents the very issue that choosing between them seems to resolve and which the play itself, in its interrogation of historical contingency, leaves open. As readers of early modern plays, we might prefer a single text to interpret, but a critical practice focused on best texts will always be thwarted by the circumstances impinging on theatrical production. Alterations were the norm, not the exception, whether for reasons of casting, venue, censorship, or even the creative whimsy of the author. All plays exist in multiple versions, even if we do not have them. Entire histories of revision and performance are, after all, obscure to us. Increasing the number of "texts" of King Lear either from one conflated or preferred edition between the two available still does not capture the multiplicity of possible variants. If the smallest discrepancies between copies of a single edition cannot satisfactorily be rectified, even the most rigorous textual theorizing cannot hope to resolve larger ambiguities such as interpolations, differently-assigned lines, omissions, and the like. The suppositions built on suppositions that would be required to arrive at a convincingly ideal edition unfortunately result in a greater, not a lesser, distance between our experience of the play and its original form—that because the original form was part of a fluid process of composition, performance, revision, and both authorized and unauthorized publication. At no point does a single best text emerge nor is it clear a playwright of the period operated according to modern principals of textual editing. No doubt, they operated more in the world of occasion and contingency that still characterizes the theater. Rather than looking for Shakespeare's genius at work behind every line, it might be more fruitful to consider the "text" of King Lear first of all as an open-ended theatrical mash-up involving an entire theatrical community and including audiences instead of a singular product of a singular creative mind and second of all as an element of a broader continuum of texts dealing with the Lear story.
	The textual ambiguities of the multiple versions of King Lear are reflective of the play's historical dislocation. As is well known, Shakespeare's telling of the Lear story is its only instance of significant departure from the traditional account, in which Cordelia prevails and Lear is restored to his throne. While he clearly drew specific details from multiple sources—including Geoffrey of Monmouth, Holinshed, and possibly Spenser and the authors of the Mirror for Magistrates—his radically different ending is unique. Though many excellent studies exist that enumerate Shakespeare's debts to and divergences from his sources, the reason he altered the facts of history, as they were then believed to be, has not been sufficiently explored. Critics may surmise whether it should be Edgar or Albany who more properly assumes the mantle of rulership, but Shakespeare gives no clear indication—nor any clear indication in the text that either they or anyone else does, theatrical tradition that the final lines go to the new authority figure notwithstanding. Since the deaths of Lear and Cordelia, not to mention Regan and Goneril, foreclose the presumed succession, it may be viable to suppose that nobody succeeds to the throne, that Shakespeare's Lear is cut off from history. What would be the consequences of this reading? And why would it have even mattered to Shakespeare's contemporaries that his fairy-tale from ancient Britain diverges so much from the standard historical account?
	In an extended essay that elaborates on "the plurality of the modalities of belief," the French historian Paul Veyne attempted some years ago to explain how highly learned people from antiquity to the birth of modern historical science could dismiss the more fanciful elements of the Greek myths while still crediting their ultimate historicity. The appointment of King Minos to the judiciary of the underworld might not have been taken as literally true by either the ancients or their Renaissance successors, but there was at least a general belief that Minos himself existed. Why, Veyne asks, did intelligent people, though they might be fully prepared to discredit any number of spurious texts or fabulous legends, nevertheless continue to pass on as authoritative so many traditions that could be vouchsafed by neither hard evidence nor common sense? He ascribes the perdurance of such credulity to an epistemological distinction—between things known from experience and things known only from books—lacking in the premodern consciousness. In studying an old text, he argues, the scholars of the past assumed it to possess "the depth and consistency of reality itself" and operated as if "deepening one's understanding of the text will be the same as deepening one's understanding of reality."*21* Until the moderns won their quarrel with the ancients, that is, and the deferential translatio of traditional authorities was displaced by a more skeptical historical method, readers lived in their books, or rather through them: they could identify their contradictions, solecisms, and infelicities adeptly enough and even express doubt about the veracity of what lay before them, but they regarded old texts as necessarily bearing some element of truth, however obscure. Sometime in the seventeenth century, the ground shifted. The respect accorded ancient authorities dissipated and with it the unquestioning belief in the historical basis of myths. But the historical problem posed by myths remained: "People no longer asked, 'What truth does myth have? For it contains some truth, since nothing cannot speak of nothing.' Now they asked, 'What meaning or function does myth have? For one cannot speak or imagine for nothing.'"*22* The question of myth's standing as testimony to history, that is, changed into an anthropological dilemma about the role myth played within a given culture's worldview, a worldview inferior to the more enlightened recognition of both historical difference and empirical analysis—or so the modern perspective would have it. By writing about the modalities of belief, however, Veyne did not intend to emphasize how blind past societies were to the fundamental facts of reality we now take for granted but to suggest how normal it is for an intellectual culture to operate according to contradictory beliefs, to believe something in one respect but also not believe it in another. It was possible, in other words, for premodern subjects to believe in the truth-value of a myth without necessarily believing in its literal facticity just as it is possible today to "believe in" the provisional conclusions of scientific inquiry while fully expecting them to be swept away by new evidence and better theories in the future. Multiple modalities of belief are not only characteristic of intellectual life in general but particularly pertinent to our understanding of some of the more curious convictions of premodern subjects.
	Although occasioned by the myths of ancient Greece, Veyne's proposal can help explain the unusual persistence of a similar mythopoetic structure in early modern England: the Brutus legend. According to this account, first promulgated by the twelfth century Welsh cleric Geoffrey of Monmouth in his Historia Regum Britanniae, the British people are descended from the great-grandson of Aeneas, a Trojan named Brutus (hence the name of Britain). From Brutus issued all of the British kings down through Arthur, for whom Geoffrey is a principle source, and Cadwallader, the last of the kings to predate more dependable historical sources. Despite skepticism about the reliability of Geoffrey's text stretching back all the way to his contemporaries, the myths he devised (and which he claimed to have himself translated from a lost Welsh original) retained both a widespread popularity and historical credibility well into the seventeenth century. Indeed, his chronicle fit neatly into a Renaissance tradition of forged documents and invented ancestries, the Antiquitatum Variarum of Giovanni Nanni (or Annius of Viterbo), which also stood up against criticism for quite a long time, being the most famous. It was surely flattering for the English monarchs, particularly the Welsh Tudors, to pretend they derived from the stock of Trojan princes and for the people of London to style themselves citizens of "Troynovant." But it is also curious that even the very learned, many of whom expressed reservations about it, were reluctant to dismiss the Brutus myth outright. Too popular to challenge directly, they granted it the same general pass given then by all historiographers to accounts of the past that might be useful for the indoctrination of virtuous conduct or the demonstration of efficacious examples even if they were factually dubious. The story clearly had a value that transcended what attached to mere fables or romance. The idea that Brutus and his successors were a part of history and not merely wishful antiquarianism perhaps connected the English people to an otherwise obscure and irrecoverable past. Even if all the details of these royal lives as recounted in so many sources in so many different ways were not necessarily plausible to them, they could nevertheless believe they had to have some basis in fact. The remoteness and inscrutability of their origins might else have cut them off from the common historical narrative of the Western world, in which fact blended freely with fancy and every city and nation wanted a part. That early modern polities took seriously their derivation from the exploits of Homeric myth is certainly an indication of people living in their books, but it also exemplifies the plasticity at the time of the idea of "history" and the uses to which history could be put when it was unbounded by a subservience to mere accuracy. Like the Greeks with their myths, the early modern English could simultaneously discount or accept as fabulous much of the narrative content of their own foundation tales while nevertheless believing that they contained some truth, some value, some bearing on their own lives and times. They were, after all, recorded in books that few had any reason, motive, or competency to doubt.
	This double-mindedness the English had about their heroic past opened up a unique space for the imaginative writers of the period to work in. Since so many of the details were lacking for the distant centuries during which the legendary monarchs reigned, a historically-minded author could freely invent within the quite accommodating outlines of the legendary frame. A gap in a myth always furnishes ready supply for further myth making. In addition, the affective attachment to national history that was especially strong in the Elizabethan age became a resource to be exploited. The stakes for storytelling are entirely different, their impact far more personal for readers and audiences, when the story that is being told dovetails in some way, however remotely, with "real life." As with so much fantasy and science fiction writing in more recent times, additional purchase on readers's interest is gained when an imaginary past, future, or alternative present adopts an etiological function. In the case of England and its relationship to its antiquity, the kings of bygone ages became fodder for many sorts of literary improvisation: speculative chronologies, advice books for princes, moralized fables, dramatic reimaginings, commentaries on contemporary political situations, and more. The possibility that these poetic interpretations  bore some kernel of truth made them all the more compelling—and convincing.
	Among these imaginary monarchs, there are several whose fictional reigns became matter for the playwrights of the early modern theater, including Shakespeare. Veyne's insights about different modalities of belief may provide an approach to one of the most bedeviling questions about Shakespeare's King Lear: why did he change the ending? We have already examined some of the effects the ending had and the consequences for the historical continuity the play presupposes and then disrupts but not yet why he might have changed it at all. Although he was not the first dramatist to lay claim to the tale, Shakespeare is responsible for making the story of King Lear (spelled "Leir" by most of his predecessors) nearly as well-known as that of his more illustrious descendant King Arthur. In every era, what has confronted readers and audiences of his version of the play most acutely is that ending, an ending so improbable, unexpected, and painful that it demands attention from any attempt to resolve the play, as a text, into an aesthetic unity. What is troubling about the ending of King Lear is not simply the distress of an old man who is stripped of his dignity, driven to insanity, and forced to watch his own daughter die. What also dies with Lear and Cordelia is, chronologically speaking, England's future. According to the chronicle accounts of the story, Cordelia should, in fact, have recovered Lear's throne and outlived him. Her nephews would ultimately rise against her, but Shakespeare obviates even that possibility by killing off Goneril and Regan, too. This was no minor legend he had altered, either. By the turn of the seventeenth century, the Lear story, which was part of Geoffrey's Historia, had appeared in more than 50 versions,*23* including an earlier play that kept the happier ending intact. Later in the century, Nahum Tate's adaptation of Shakespeare's play to Restoration sensibilities reintroduced these concluding felicities and even contrived a romance between Cordelia and Edgar—for a century and a half, this was the only Lear that audiences ever saw performed. Shakespeare's anomalous exception to this tradition changes the course of the legendary events which many in early modern England still held sacrosanct. The Trojan origins of the British people and the idea of London as "Troynovant" were powerful symbols for both a nobility that had become obsessed with symbols,*24* especially those evoking the grandeur of ancient heredity, and an increasingly more confident nation that was ever seeking to distinguish itself from Europe and indeed from the rest of the world.*25*
	We do not, however, usually count King Lear among Shakespeare's history plays properly speaking: it was written later than the others (when history plays in general were out of vogue), its setting is the distant rather than the relatively recent past, and it is not even categorized as a "history" in the First Folio. Nevertheless, Whereas in Richard II, in many ways a precursor to Lear, Shakespeare demonstrates the consequences of speaking from the position of history, in King Lear, he challenges the usefulness of history as an explanatory mechanism and suggests the futility of depending on history as a source of right outcomes or useful examples. But the way he uses history in this play would not be possible, would not have had quite the same resonance, if the Lear story were not credibly embedded in the historical consciousness of the English people at the time he was writing. In fact, the distant antiquity of Britain was perhaps of greater interest at no other time than when Shakespeare was writing, but also under the greatest threat from factually punctilious quarters. But it took a long time indeed for the legends of which King Lear formed a part to be extricated from both the popular imagination and learned discourse. One can hardly identify with certainty the deathblow, despite withering assaults against them, as they kept being revived. For a century or more, they inhabited a sort of grey area of credibility, neither completely believed nor entirely abandoned.
	In fact, among the historically-minded of the time, a debate raged between proponents and detractors of the ancient history, each side with some claim to scholarly authoritativeness. While no definitive distinction existed before the seventeenth century between more or less historically-accurate sources, the textual critics of the time had already begun to cast a jaundiced eye on the more dubious documents. As in other areas of culture, however, there was a great deal of overlap between the high and the low in early modern letters. Popular legends mingled freely with learned treatises in Latin, the latter often adducing evidence to support the historicity of the former. As far as most people were concerned, Geoffrey of Monmouth was a reliable transcriber of authentic accounts, and the British were long-established, noble descendants of the Trojans. Richard Harvey wrote an entire treatise defending the Trojan inheritance,*26* and even King James referred to "the heritage of the succession and Monarchie, which hath bene a kingdome, to which I am in descent, 300. yeeres before Christ."*27* Those who needed James's ancient lineage spelled out for them in detail could consult the exhaustively-titled treatise prepared by the parson George Owen Harry, The genealogy of the high and mighty monarch, James, by the grace of God, king of great Brittayne, \&c. with his lineall descent from Noah, by diuers direct lynes to Brutus, first inhabiter of this ile of Brittayne; and from him to Cadwalader; the last king of the Brittish bloud; and from thence, sundry wayes to his maiesty: wherein is playnly shewed his rightfull title, by lawfull descent from the said Cadwalader, as well to the kingdome of Brittayne, as to the principalities of Northwales and Southwales: together with a briefe cronologie of the memorable acts of the famous men touched in this genealogy, and what time they were. Where also is handled the worthy descent of his maiesties ancestour Owen Tudyr, and his affinity with most of the greatest princes of Christendome: with many other matters worthy of note. Polydore Vergil, surveying the historical records which mention Britain as well as taking recourse to his own reason, concludes that Britain and its surrounding isles, being in constant communication with mainland Europe, must reasonably be assumed to have been inhabited since the earliest times, like all other lands after the Flood. Nevertheless, he rehearses the Brutus story concocted by Geoffrey only for the sake of completeness. He himself, finding no corroboration for the legend elsewhere, is skeptical: "But yet nether Livie, nether Dionisius Halicarnaseus, who writt diligentlie of the Romane antiquities, nor divers other writers, did ever once make rehersall of this Brutus, neither could that bee notified bie the cronicles of the Brittons, sithe that longe agoe thei loste all the bookes of their monuments."*28* William Camden, the greatest of the Elizabethan chroniclers, is similarly reluctant to dismiss the legend—perhaps protesting too much that he does not wish to do so—and ultimately gives it over to the discretion of his readers to determine its veracity:
For mine owne part, it is not my intent, I assure you, to discredit and confute that story which goes of him [Brutus], for the upholding wherof, (I call Truth to record) I have from time to time streined to the heighth, all that little wit of mine. For that were, to strive with the streame and currant of time; and to struggle against an opinion commonly and long since received. How then may I, a man of so meane parts, and small reckoning, be so bold, as to sit in examination of a matter so important, and thereof definitively to determine? Well, I referre the matter full and whole to the Senate of Antiquarians, for to be decided. Let every man, for me, judge as it pleaseth him; and of what opinion soever the Reader shall be of, verily I will not make it a point much material.*29*
The Scottish historian George Buchanan, however, has no qualms whatsoever about eviscerating the whole fad for imagined histories (and uses a metaphor familiar to Shakespeareans to do so):
Among all the tribes of the British, there was such a lack of writers that before the arrival of the Romans there, everything laid completely engulfed by the desolate gloom of silence, and we are not able to learn the deeds performed in that place even by the Romans other than from the Greek and Latin histories: and about these things which preceded their arrival, rather their conjectures should be believed than our inventions. For truly the things our writers have revealed, each one concerning the origin of his own clan, are so absurd I would not have supposed they needed to be disproven more thoroughly, unless there were people who are delighted by those fictions as though serious matters and magnificently pleased with themselves to be embellished with the feathers of others.*30*
Even John Milton found cause enough to include in The History of Britain, one of his latest works, a complete recounting of the legendary kings first reported by Geoffrey of Monmouth and the exploits conventionally attributed to them. He was certainly writing against the current. By the late seventeenth century, academic historians had made great strides in textual criticism and philology. Many of the fictions and forgeries that had for so long stood authoritative they discredited, and through the application of their accumulated learning combined with fresh skepticism about the validity of received authority, they had begun to bring order to the confusion of fabulous legends and authentic records that chronicled Britain's past. In particular, a consensus was forming about the unreliability of all accounts of British history prior to the coming of the Romans that were unsubstantiated by the Romans's own historians. Milton was not altogether ignorant of these developments, but his doubts were mollified by the poetic inclination that would ensure the literary survival of these legends even as graver scholars stripped from the factitious all facticity. His excuse for the further perpetuation of this apocryphal material, though he was writing when it was reaching the limit of its shelf-life as history, shares an ecumenical flavor with the quasi-humanist compilers of the previous century, the first stumblers through the thicket of conflicting sources:
What ever might be the reason, this we find, that of British affairs, from the first peopling of the Iland to the coming of Julius Caesar, nothing certain, either by Tradition, History, or Ancient Fame hath hitherto bin left us. That which we have of oldest seeming, hath by the greater part of judicious Antiquaries bin long rejected for a modern Fable. Nevertheless there being others besides the first suppos'd Author, men not unread, nor unlerned in Antiquitie, who admitt that for approved story, which the former explode for fiction, and seeing that oft-times relations heertofore accounted fabulous have bin after found to contain in them many footsteps, and reliques of somthing true, as what we read in Poets of the Flood, and Giants little beleev'd, till undoubted witnesses taught us, that all was not fain'd; I have therfore determin'd to bestow the telling over ev'n of these reputed Tales; be it for nothing else but in favour of our English Poets, and Rhetoricians, who by thir Art will know, how to use them judiciously.*31*
Milton's humble gesture to the authority of his learned predecessors is customary enough but gains in interest by the subsequent qualification that, even if the earlier accounts are fabrications, they might still be of some use to poets—traffickers in another kind of truth. Here, Milton's attitude to history is congruent with the older humanist one: a story may be useful whether or not it is true, its value laying more in its utility than its accuracy. A generation of political convulsions  that led to doubt about the relevance of history to contemporary concerns and scholarly development in the discovery, criticism, and dissemination of new texts and methods,*32* however, had made the humanist aspiration to the advisement of princely conduct obsolete*33* and refashioned the myth-spinning of Geoffrey and his ilk into a source of national identity and the sentimental medievalism of strictly imaginative writing—a trend that would reach its full-flowering in the historical romances of a later generation of authors. In offering to supply matter for poets, Milton might have been thinking of his own erstwhile plan to write an Arthurian epic or else to the "sage and serious Poet Spencer" who did do so and managed to generate exempla of moral truth more palatable in his estimation than the bitter pills of moral philosophy. He might also have been assigning to the poet's exclusive privilege a moral agency in the representation of history above and beyond service to a prince, continuing his project of, according to Richard Helgerson, moving the poet "into the place formerly reserved for the ruler, appropriating as he does so the ancient forms that once stood for imperial power."*34* Shakespeare begins this work himself by making out of Elizabethan historiography, a plank of imperial power in many respects, a dramatic mythology comprehensible to a broader audience, an audience we might tentatively regard as a public. Discarded as matter for serious historiography or plausible genealogy, the ancient legends could still find employment in the articulation and aggrandizement of the English people's self-consciousness—as, by dint of their ancient veneer, one of its preeminent expressions. To this end, the ponderous chronicles overlapped with the more poetic appropriations of historical material, though both would provide source material to Shakespeare for King Lear.*35*
	The most intriguing, pre-Shakespearean version of the Lear story specifically must be the one appended by John Higgins to his 1587 redaction of The Mirour for Magistrates.*36* In previous editions, Cordelia's fate was more or less a footnote to Lear's. She has a part to play in his story—the dutiful daughter who returns from an undeserved exile to restore her father to his throne—but is quickly dispatched once Lear has died. Higgins, however, inverts this arrangement. His poem folds Lear's story into Cordelia's, which she tells in the first-person from her own point of view. This is in keeping with the narrative style of the Mirror for Magistrates series: downfallen princes admonish readers directly to, as the author's induction puts, "marke the causes why those Princes fell." But it is a remarkable innovation to have made Cordelia the central, tragic figure in the well-known legend of King Lear and his three daughters. There is some sense to this, of course. Since the Mirror is a collection of "tragedies" about the falls of princes, Lear's triumph and redemption do not fit the wheel of fortune profile, but Cordelia's unfortunate fate does. This foregrounding of Cordelia in a text meant to serve as a vade mecum of self-reflection for the ruling class points toward the expanded dramatic purpose her character acquires in Shakespeare's play.
	As in Higgins, the tragic linchpin of King Lear is not, per se, the untimely death of the king himself or even, I would argue, his protracted suffering. Lear does not survive to rule his kingdom once again, but he is nevertheless offered the throne by Albany. This is no small difference, but in historical terms, it could be argued that it does not entirely matter what happens to Lear at this point. As long as Cordelia succeeds and the established line of kings may continue, we might not marvel at this, the sort of liberty Shakespeare is known to take in all of his history plays, legendary or not. While Lear's end is certainly pitiful and his decline painful to observe, the play is tragic, properly speaking, only for the sake of Cordelia. This is a contentious point to make, but it is Cordelia's precipitous downfall, not Lear's, that drives the closing scene of the play, her inglorious death that mortally breaks Lear's heart, and her failure to succeed that interrupts the continuity of British history. In her own, final words, she demonstrates her awareness of this fact, that she knows exactly where she is in the tragic tradition and exactly the role she has inherited from the Mirror for Magistrates:
We are not the first
Who with best meaning have incurr'd the worst.
For thee, oppressed king, am I cast down,
Myself could else out-frown false Fortune's frown. (King Lear, 5.3.3-6)
This is remarkably self-conscious language—meta-tragical, it might be called—and a remarkable moment of sympathetic exchange between Cordelia and Lear. "For thee," Cordelia says to Lear, she has done everything, has shown to what extent her love is more ponderous than her tongue: it is more ponderous, because her fall is predicated on his. Were it not for Lear, she could "out-frown" Fortune and overcome with stoic resolve her defeat. Were it not for Lear, she would not even consider herself cast down. She is defeated, cast down in that respect, but her meaning of "cast down" has here an added affectiveness. These lines bear even greater epigrammatic potency than the last lines of Albany/Edgar. And they are similarly conclusive, as Cordelia is about to fall silent for the remainder of the play. Her triple rhyme within a double couplet accented by a quadruple alliteration verges dangerously close to an excess of sonic effects, but there is also the faintest echo in this overly poetic sequence of the Mirror for Magistrates's rhyme royal. In that text, as has been observed, Lear's story is folded into Cordelia's, who had prior to that been the loyal, youngest daughter in his. Shakespeare has managed to deploy his materials in such a way as to virtually have it both ways. The succession depends on Cordelia's survival, and Lear places all his hopes in her, hopes, like the succession itself, that are shattered by Edmund's cynical maneuvering. But Cordelia claims that her defeat is not so much dynastic as personal, familial, as the defeat of someone who wished not to advance herself but another. That the other is aged and infirm suggests that Cordelia is not simply, in this admission, professing the dependence of her identity and status on his. She demonstrates the dignity of self-possession in explaining what she perceives to be the true nature of her defeat, not a military but an emotional one, and, more importantly, in taking possession of the narrative itself. She recognizes the tragic mold of their situation—they were not the first and will not presumably be the last to suffer the whims of fortune—but undercuts, in her pity for Lear, the interpolating of her fall into the series of admonitory examples that informed an increasingly moribund style of historical praxis. The force of her final lines carries through to the end of the play, and we cannot forget them. In fact, Cordelia's silence after Lear's exhortation that they go off to prison like birds in a cage is as mysterious as Isabella's at the end of Measure for Measure or Coriolanus's at his final meeting with Volumnia. She does not respond to Lear again—perhaps out of indulgence, perhaps assent—and that she does not speak again at all gives her final lines a reinforced sense of finality that makes of the remainder of the play a rather awkward, inconclusive epilogue. It is as if the play ends at this moment but the other characters do not yet know it.
	Had both Lear and Cordelia lived, as in the hitherto traditional course of events, the play might have had more in common with the tragicomedies then beginning to appear on the Jacobean stage. What it does have in common with the tragicomedies is its deeply cynical attitude toward the governing values of a formerly feudal society that was becoming ever more individualistic. What Shakespeare adds to this problematizing trend in playwriting—what is his unique contribution with King Lear—is the blending in, to an abstract and urbane meditation on interpersonal ethics, of a historical perspective. The publication of the chronicle King Leir, which was nearly contemporaneous with the First Quarto of King Lear,*37* might have encouraged him to attempt a rewriting, and a revisiting of the historical genre, in the spirit of a changed theatrical atmosphere. The Mirror for Magistrates had, from its first edition, demonstrated how the chronicle histories could be adapted to poetic ends, the figures of history given voices to speak on their own behalf directly to readers. Gorboduc, the first English tragedy, also based on a historical legend of the same ilk as King Lear, and partly attributed to one of the Mirror's original contributing authors, extended that idea to the stage. The chronicle King Leir continued the tradition, adding elements of romance that further indicated the potential for shaping historical accounts in the mold of broader narrative visions. History, in the hands of these poets and dramatists, became a moldable material, resulting in works that fell somewhere between historical fiction and the biopic. Some fantastical element inherent to the legendary past, or perhaps because it is so ill-defined, made it especially compelling material. That Lear's story did come from a legendary period, the historicity of which was beginning to be questioned toward the end of the sixteenth century, might have sanctioned Shakespeare to use it for such a thoroughgoing dramatic experiment. And so he has his Cordelia die, and we are left wondering for what reason.
	Cordelia is the key element in the schema that coordinates that experiment, the character who was always central to the Lear legend as a plot device but who gains from Shakespeare a symbolic significance beyond simple protagonism, even as she has her "part" substantially curtailed. She is the victim of her father's folly, a folly that is personified first by an actual Fool and eventually, when he comes to identify with the "poor bare, forked animal" that his hapless illogic has made him, by Lear himself. Around Cordelia circulates Lear's nostalgic longings for an imagined past of paternal authority and filial obedience: she represents the country from which the play so suddenly bursts forth, at the very moment when Lear bursts it apart, and the country to which he would return. She is also the favored child who, in her absence from the center of the play—even as she directs from behind the scenes all the action of the play so hopefully toward Dover—evokes the absence of Queen Lear, who is absent even from Shakespeare's text, though her spirit haunts the division of the kingdom in the chronicle play. Janet Adelman regards this omission as structural to the gender dynamics of the play:
For the idealized mother Lear seeks in Cordelia and the horrific mother he finds first in her sisters and then in himself are psychically one, merely flip sides of one another; they have a common origin in the developmental history of male identity as it is tenuously separated out from its originary matrix, the mother that it—like this text—would occlude.*38*
Adelman makes of Cordelia perhaps too symbolic a figure—a figure, that is, that only has meaning insofar as she means something to Lear—but she is right to emphasize the considerable dramatic efficacy of Cordelia as a character, who exerts her force despite being out of sight for most of the play. But Shakespeare's idiosyncratic positioning of Cordelia is, in many ways, a more interesting phenomenon than what happens to Lear, especially where history is concerned. Cordelia's death, by the very fact that it interrupts history, serves like the most conclusive of dramatic bookends to Lear's discovery of himself as out of joint with a world that can just carry on: hence his total disconnection from the ineffective protests of Albany, Edgar, and Kent, who try to grant him a crown he no longer recognizes, who try, though it is "bootless," to draw him back into the drama he has forsaken. Instead, he dies as he must in order to experience the final, fatal fact of bare humanity, and we, as readers and audience, experience it with him. Cordelia dies as she must in order to affect this transformation but also to grant her character a strange dignity, the dignity of failure, of going beyond the service of the progression of history, a history of one fall after another, and into a role of heroic service that sacrifices everything for a hopeless, futile cause. But it is the intersubjective quality of Cordelia's sacrifice that makes her a fascinating figure, not her attempt to play the military savior. Her elevation of her personal relationship to Lear above the significance of historical contingency is what makes the play historically revolutionary. She turns history into a personal matter that supersedes its usual narrative thrust of one event after another and turns the exemplary mirror of history inward.
	Perhaps there is something of the magistrate's mirror at work in Shakespeare. Higgins's Cordelia is a mirror for princes who narrates her own lamentable fate as an admonition to—what exactly? There is no clear lesson. But it is nonetheless a tragic fate, the fall of a prince, and it is not for the compiler of these tales to say which might actually serve in the capacity of a mirror for any given reader. Such tales could also be relevant and effective in the city playhouse, which offered to theatergoers its own varieties of moralistic mirrors. One example is a Biblical play by Thomas Lodge and Robert Greene, A Looking Glasse for London and England, the very title of which is explicit about the didactic purpose of plays. In its conclusion, the authors extend the analogy presented by the play's action into a fulmination against the excesses of city life:
O London, mayden of the mistresse Ile,
Wrapt in the foldes and swathing cloutes of shame:
In thée more sinnes then Niniuie containes,
Contempt of God, dispight of reuerend age.
Neglect of law, desire to wrong the poore:
Corruption, whordome, drunkennesse, and pride.
Swolne are thy browes with impudence and shame.
O proud adulterous glorie of the West,
Thy neighbors burns, yet doest thou feare no fire.
Thy Preachers crie, yet doest thou stop thine eares.
The larum rings, yet sléepest thou secure.
London awake, for feare the Lord do frowne,
I set a looking Glasse before thine eyes.*39*
In this passage, the mirror, or looking glass, shows how remarkably versatile a basis for poetic figuration it is. When it appears as a physical object, it is hardly ever innocent, alluding by its very nature to an intersubjective encounter with oneself. Analogically, a mirror can be anything that reflects back the true nature of things, an ironic function for something that really only traffics in appearances and often distorted ones at that. But this reflecting function was crucial to early modern poetics. The idea that writing was revelatory, that it allowed you to see yourself as you really are, was fundamental to all species of discourse. Lodge and Greene are so insistent on it that they force the issue. King Lear, however, develops the intersubjective encounter afforded by the mirror to a much higher level of abstraction, whereby the play itself becomes at its conclusion a reflecting surface for the association between the experience of the characters in the world of the play and the experience of the audience as witnesses to their own history. The play like a mirror reflects for the audience the personal costs of history which requires a disruption from history's smooth flow. In the chronicle Leir, Cordella is a more central character and a romance heroine. In Lear, she functions rather as an audience surrogate, witnessing like an audience member the proceedings of the opening scene and bearing witness to the degraded, transformed Lear at the end. In language that echoes across the dramatic literature of the period, Cordelia herself experiences the shock of recognition of the barest state of humanity when she sees her father again for the first time:
Was this a face
To be oppos'd against the [warring] winds?
[To stand against the deep dread-bolted thunder?
In the most terrible and nimble stroke
Of quick cross lightning? to watch—poor perdu!—
With this thin helm?] Mine enemy's dog,
Though he had bit me, should have stood that night
Against my fire, and wast thou fain, poor father,
To hovel thee with swine and rogues forlorn
In short and musty straw? Alack, alack,
'Tis wonder that thy life and wits at once
Had not concluded all. (King Lear, 4.7.30-41)
The most obvious precursor to this rhetorical image of disbelief, this recognition of the distance between form and substance, idea and reality, is in Marlowe's Doctor Faustus. The resemblance to Faustus's encounter with Helen of Troy is uncanny:
Was this the face that launched a thousand ships
And burnt the topless towers of Ilium?
Sweete Helen, make me immortal with a kiss.*40*
And from Shakespeare's own Richard II is an equally famous moment of mirrors and misprision:
Was this face the face
That every day under his household roof
Did keep ten thousand men? Was this the face
That like the sun, did make beholders wink?
Is this the face which fac'd so many follies,
That was at last out-fac'd by Bullingbrook?
A brittle glory shineth in this face,
As brittle as the glory is the face,
For there it is, crack'd in an hundred shivers.
Mark, silent king, the moral of this sport,
How soon my sorrow hath destroy'd my face.*41*
If that were not enough, even one of Shakespeare's comedies, All's Well That End's Well, cites similar lines from a now lost ballad: 
"Was this fair face the cause," quoth she,
"Why the Grecians sacked Troy?
Fond done, done fond,
Was this King Priam's joy?"
With that she sighed as she stood,
With that she sighed as she stood,
And gave this sentence then:
"Among nine bad if one be good,
Among nine bad if one be good,
There's yet one good in ten."*42*
An earlier example of such a moment occurs in a mid-sixteenth century translation of Cicero's Tusculan Disputations:
Is this that Telamon, whom late frayle glory raysde on hye.
On whom the Greekes of late dyd gase with many an enuious eye.*43*
The pattern and variations on it also show up in several places later in the seventeenth century, too. One example is Thomas May's The tragedy of Julia Agrippina:
Was this the face, that once in Caesar's love
Was Agrippinaes rivall, and durst hope
As much 'gainst mee, as my unquestion'd power
Has wrought on her?*44*
Cordelia looking on Lear's face is the most touching version of the "was this the face" tradition because it's the only one that involves true intersubjectivity. In the case of Faust, he is still living through his texts, recalling a past glory. In the case of Richard, he is wrapped up in himself and the infinite regression of his own self-indulgent symbolism. But Cordelia remarks on the true gap between the image of authority, which is only image as we have learned from Lear himself, and the bare humanity before her. History attempts to provide rationales for human behavior, and the history play to depict the waxing and waning of fortune. In King Lear, however, Lear and Cordelia step out of history and climb off the wheel of fortune, leaving history no longer as a mirror for human action but human action as something that cannot be comprehended by history. 
	We must wait for the epigrammatic finality of Lear's closing lines to make sense of the play as an aesthetic whole. All the perturbations in meaning, the equivocations, the horrors, the generic displacements get smoothed out by "the weight of this sad time" and the event, Cordelia's death, that configures Shakespeare's radical reinterpretation of the Lear story. Our purpose in reading the play, or the audience's in watching it performed, is, in fact, to witness Cordelia's death, to witness the moment of release from history and from the paternal/familial order of society that structures it. The play is less about leitmotifs of seeing than a constant recurrence of mirror images, mirrors by which we see but do not witness ourselves: the play, as a second cousin of the Mirror for Magistrates, holds a mirror up to the audience. The Fool, like a mirror, reflects back to the unhearing Lear the substance of his folly. The Gloucester subplot mirrors the main plot; the rival sisters Regan and Goneril mirror each other in motive and behavior, and their mirror-image husbands, the good Albany and the evil Cornwall, like the virtuous King of France and the conniving Burgundy, reflect the dutiful Edgar and the scheming Edmund. Cordelia, in this configuration, is an absent element. Who does she mirror? The answer is Lear. In two scenes we see mirror images of a sleeping Lear regarded by a pitying Cordelia, and later,  as if in counterpoise, the more celebrated tableau of the dead Cordelia in the dying Lear's arms. Lear, who "hath ever but slenderly known himself," asks for a "looking-glass" by which to gauge Cordelia's life. Is this the end itself or only an image of it? Shakespeare characteristically leaves us with an equivocation. In historical terms, it is surely the end. Debates about the implied succession do not give the play its proper due as a rejection of historical time itself, of the guarantee of continuity. As a work of poetry, it is surely an image to be reflected back to the reader or audience. The play is a reflection of history as it is: without point or justice, without sense but the sense we bring to it. The audience is left to fill in the ending, as critics have attempted to do for centuries, and it may be this that often makes productions of Lear so deeply unsatisfying, and  that convinced audiences for more than a hundred years to prefer a happier ending. The play requires too much of us, the "we" of the final lines too great a demand if it is extended to include the audience. It evinces a deep skepticism about historical progress, about the potential for any historical discourse, whether teleology or exemplum, to have any reliable, practical bearing on modern life. Albany/Edgar's "smug sententiousness"*45* is what is dismissed by the play's conclusion, just as the stoic philosophy as a brace against the depredations of history is dismissed. The outcome of Edgar's education of Gloucester is his father's death, and we should expect the same for Lear and not be surprised by the dual outcome in the main plot. With all these characters dead for the sake of their educative journeys, only the audience is left to bear witness to Albany/Edgar's unrelenting piousness even in the face of utter destruction, of the absolute end of history. And we are meant to perceive that and decide for ourselves what to do with the sad time, its weight, and the necessity of carrying on without a convenient discourse to guide us, and that is what makes King Lear such a challenging play.