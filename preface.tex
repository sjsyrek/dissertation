In writing this dissertation, my intention from the beginning has been to find some way to bring literature and history together. I did not think of devising an approach that would combine the methods of literary criticism with those of historical research, per se, but of attempting to project myself back to a time when disciplinary distinctions were as little defined as they were defended. I wanted to discover something about the early modern understanding of what history is, what it is for, and what its relevance is to human life. Specifically, I felt it worth evaluating some of Shakespeare's historically-oriented plays for evidence of a mentality about history that could not be specified in direct prose by an arbitrary selection of writers as well as it might be by a cultural form, the theater, that represented to the greatest extent possible the imaginative world of an entire society. I assumed this mentality could not have much to do with twenty-first century feelings about history. I also worried that any sort of methodology built on anachronistic category distinctions would only end up making this distant worldview seem like a reflection or, at best, a harbinger of our own. Moreover, I did not want to simply ``do history'' in order to determine the broad ideological characteristics of early modern historical thinking, with plays serving as superficial tokens, or ``do literary analysis'' as a way of demonstrating their internal, linguistic elusiveness. It seems to me just as reductive to cite early modern literature as representative of cultural characteristics and dominant discourses, abstractly identified in some plurality of texts, as to propose that literature was a special genre that possessed the higher calling of interrogating or slyly subverting those characteristics and discourses. Historical readings of literary texts, in other words, often come across as facile in terms of their critical sophistication while critical readings that attempted to justify their validity through historical contextualizing often come across as naïve in terms of their historical sophistication. My goal in this endeavor has been to avoid privileging one over the other but to nevertheless unsettle certain assumptions about the relationship between literature and history, particularly as that relationship was understood in late-sixteenth and early-seventeenth century England.

One problem with this endeavor, brash as it sounds, is the difficulty of positioning a methodology between a field that embraces them---history---and a field---literary studies---that has long tended to eschew them. In the case of the former, it is impractical to do more than consider some of the relevant findings of certain areas of inquiry; it is not easy to evade the appearance of cherry-picking, given the diversity of perspectives and variety of approaches in the historiographical field. The usual course in the latter, however, is to ascribe methodological definitions to others while maintaining for oneself a privileged position of open-minded reflection, one with the ambiguity. While historians cannot even begin to work until they have a methodology in mind, because their goal is explanation across texts and not the preservation of their aesthetic inviolability, the methods of literary analysis seem genetically predisposed to undermine their own fundamental assumptions as soon as they can be elaborated. Cleanth Brooks, when asked to write an essay about the New Criticism, demurred that there might be any such thing: ``The New Critic, like the Snark, is a very elusive beast. Everybody talks about him: there is now rather general agreement as to his bestial character; but few could give an accurate anatomical description of him. Even when one believes that the Snark has actually been netted, he usually turns out to be not a Snark at all but a Boojum''~\cite[592]{Brooks}. A few decades later, when Stephen Greenblatt and Catherine Gallagher wrote a similar account of the New Historicism, and its emergence from ``an impatience with American New Criticism,'' they were equally credulous that other people thought that what amounted to an incoherent variety of critical practices that ``resisted systematization'' was an identifiable enterprise: ``How could something that didn't really exist, that was only a few words gesturing toward a new interpretive practice, have become a `field'?''~\cite[1--2]{Greenblatt/Gallagher}. These are distinguished representatives of two of the most dominant paradigms of literary study in the twentieth century---the paradigms that inform most of my own work, as well---and yet they are standoffish about belonging to any club that would have them as members. Brooks, with whom ``close reading'' itself is closely identified, thinks that even this shared legacy of the New Critics, possibly the only one according to him, does not exactly have the best name: ``it might be more accurate to substitute `adequate reading.' `Adequate' is, to be sure, a relative term; but so is `close.' (How close is close enough?)''~\cite[600]{Brooks}. Hardly any practicing New Historicists, if they self-identify as such, would reject entirely the set of practices employed by the New Critics any more than they would disagree with Brooks's advocacy of a critical method that is adequate to the text under consideration.

Nevertheless, it is hard to stake a new position without striking down an old straw man, and it is on that basis that Gallagher and Greenblatt contend, against all appearances and voluminous protestations to the contrary,\footnote{A search of the relevant online databases for ``new historicism'' will turn up hundreds of articles articulating any number of positions on this school of thought that claims not to be one. Greenblatt and Gallagher, in their disavowal of New Historicism's methodological integrity, go so far as to reject the usual conventions of capitalizing ``new historicism'' in their text, an orthographical reification they do observe when referring to New Criticism, and qualifying it with the definite article.} that ``new historicism is not a repeatable methodology or a literary critical program''~\cite[19]{Greenblatt/Gallagher}. In contrast to the verities offered by either a historically agnostic New Criticism---``so-called,'' as even Brooks puts it---or the positivist orientation of traditional (``old'') historicism, literary New Historicism defines itself instead according to what it is not. It thus offers a flexibility about the scope of the sort of evidence that might be brought to bear on a literary-critical argument. New Historicist arguments are predicated on the interpenetration of text and context and thus the availability of the whole field of cultural activity informing the composition of any object of writing, even if individually they grasp only a fragment of that field. New Historicism is, in that sense, unlimited as a critical discourse, but it is also unconstrained by standards of evaluation for what constitutes the cultural field, the ground, it must tacitly presuppose. As a substitute for making direct statements on the matter, New Historicism outsources the job to history itself while, at the same time, nervously keeping history at arm's length. Alan Liu, an admitted New Historicist himself, regards this treating of history as an unacknowledged intellectual refuge the latest episode of the ongoing embarrassment in literary studies about the relationship of literature to history. Blurring the lines between the two, New Historicism, according to Liu, abjures the older paradigm in which the gap was mediated by the history of ideas only to nostalgically recall that paradigm in its inability to separate the subjectivity of the critic from the subject the critic seeks in literary history. Without a firm historical ground to stand on, that is, the New Historicist is left proposing an alternative version of the same thing but that reflects in its critical conclusions the structural determinant of the relationship it maintains with history itself: surversion. Speaking on behalf of the New Historicist interpreter, Liu admits, ``\emph{I am embarrassed of my marginality as interpreter---specifically, of the whole enterprise of literary history, the academy, and the intellect in which I am implicated}'' \cite[747]{AlanLiu}.

In my own readings of Shakespeare's history plays, I have attempted to remain sensitive to the dangers of foregrounding subversive qualities for the sake of a transgressive critical practice and of disavowing my presumptions about history so as to leave them ``an unthought and unregulated manner'' \cite[743]{Liu}

As evasive of its critical principlesEvasions though these might be, it is arguably a strength of the literary critical establishment that it has resisted codification of specific ways of meaning---and I have not even mentioned deconstruction.

Although many literary critics have taken pains to adopt some of the tools of historical analysis (the lack of reciprocity notwithstanding), there remains in even the most ``historical'' literary criticism a recognition of the surplus in texts that resists being historicized: an ahistorical, non-theoretical indulgence. There is pleasure in the literary text but also the historical one.

``Historicism has largely and somewhat unconsciously moved in the last century from an enterprise that used history to interpret the text to an enterprise that uses the text as a means to explain history'' \cite[7]{Fulton/Coiro}. They discuss methodological weariness among early modernists, who have become less speculative and less self-conscious about their own methods as an inherent part of their practice. They worry that a literary method that becomes too similar to history, in a reductive fact-oriented form, runs ``the danger of simply doing history, with the potential of doing it badly'' (6).

Paulina Kewes discusses the paradox that among early modern writers, ``the authority of the past declined. And yet the belief in the usefulness of history for the present persisted'' \cite[1]{Kewes, History and Its Uses, 1-30}. ``With the notion of the `historical revolution' dominating late-twentieth century scholarship, historians prized accounts of the past that seemed to them to aspire to impartiality, factual accuracy, and a secular outlook---in short, that pointed the way toward modernity. Students of literature, for their part, moved from mining early modern historical writings for allusions to and allegories of public figures and events, to denying the existence of any such correspondence, to seeking to recover the broad ideological and political engagement of historiography'' (Kewes, 1-2). ``the time seems right to ask how the past was exploited to meet the concerns of the present in early modern England'' (2).

Literary critics have been arguably too beholden to historians.

I do not want to reinforce old or establish new taxonomical categories but join Benjamin Griffin in what we calls ``a new investigation of what Tudor-Stuart people meant when they talked about `history plays' ~\cite[8]{Griffin}. One way in which I deviate from Griffin's method is by considering less the generic differentiation of the term history and more the uses to which a historical-dramatic matrix could be put, when history was combined with poetry.

Brian Walsh examines the tension between providence and human agency that is itself examined in the history play. History plays, according to Walsh, issue their skepticism about a providential role in the unfolding of human events by demonstrating, by the very fact of their performance, ``that history is not a naturally occurring form of knowledge. It cannot exist autonomously. It must be produced.'' ``history as always a species of imagination'' (Walsh, book, 11). But this makes history seem almost too identical with fiction---very close to the arguments of Hayden White, who disclaims the pretensions to veracity implicit in any historical account. It seems to me, rather, that early modern subjects were capable of holding two positions about history in their minds simultaneously. On the one hand, that true history did exist, but that its operations were ultimately obscure and attempts to direct its flow inevitably vain. On the other hand, that history as a \emph{form of writing} was indeed a poetic product that could only ever imperfectly capture the truth toward which it thrusted. They were both too fatalistic (?) to believe history is all entirely invented and too canny about the rhetorical qualities of all writing to believe that any particular account had a better claim than others to accuracy.

Fredric Jameson famously described history as ``the experience of Necessity,'' a narrative category that does not consist of a content of its own but that gives to events the form of inevitability, as if there were some unseen motive force at work behind them---an absent cause---that has structured in advance our experience of the past and our expectations for the future. ``History is what hurts,'' Jameson asserts, ``it is what refuses desire and sets inexorable limits to individual as well as collective praxis, which its `ruses' turn into grisly and ironic reversals of their overt intention. But this History can be apprehended only through its effects, and never directly as some reified force'' \cite[87--88]{Jameson}.

``When an *Actour* presents himself upon the Stage, until he speak, he is but a *picture,* and when he speaks, he is but a Storie; (and therefore perhaps a Player is called *Histrio,* quasi *Historio*) for as one says well, that a Judg is *lex loquens, a speaking law*: so we may say as truly, that a Player, is a speaking Picture: or a Historie in person; and seeing we know no hurt, by a Picture; and cannot but commend Historie: why should Plays be condemned, which are but a composition made of these two? A Historie is not condemned, if recording the life of *Julian*; it set down, his cruelty against *Christ*. And if an Historian may lawfully write it, may not we as lawfully read it? and if we may lawfully read it; may not a Player as lawfully pronounce it?''~\cite[42--43]{RichardBaker}

omne tulit punctum, qui miscuit utile dulci 
lectorem delectando pariterque monendo.~\cite[ll. 343--344]{Horace}