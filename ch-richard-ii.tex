Who is the author of history?

\chapter{Is not the king's name twenty thousand names?\label{ch:RichardII}}
In the very middle of Shakespeare’s \emph{Richard II}, the doomed king returns from Ireland to discover that he has been been outmaneuvered by his upstart rival, Bolingbroke.
While Richard has been absent from England, he learns, his aggrieved cousin has managed to conquer, on top of the kingdom itself, the hearts and minds of its people.
Stephen Scroop, one of the few lords still loyal to the king, describes Bolingbroke’s rapid progress as if he were a force of nature: 	
\begin{vq}
Like an unseasonable stormy day, \\
Which makes the silver rivers drown their shores, \\
As if the world were all dissolv’d to tears, \\
So high above his limits swells the rage \\
Of Bullingbrook, covering your fearful land \\
With hard bright steel, and hearts harder than steel. (3.2.106--111)
\footnote{All Shakespeare quotations are cited from William Shakespeare. \emph{The Riverside Shakespeare}. 2nd ed. Ed. G. Blakemore Evans and J. J. M. Tobin. Boston: Houghton Mifflin, 1997.}
\end{vq}
An accumulation of synonymous details worthy of Richard himself, this epic simile---more an epic mixed metaphor---depicts Bolingbroke’s invasion as a freak occurrence.
He likens it to a hundred-year storm that, in causing the rivers to flood, has instigated the land to rise up in the chaotic fury of unnatural, revolutionary fervor and overthrow itself.
This highly affected language is characteristic of the play.
Written entirely in verse and with frequent use of such higher level poetic devices, it transforms history into an artifact of language.
The excesses of that language bespeak the excesses of the poetically formalized world in which the drama of \emph{Richard II} unfolds.
Poetry in this play, in other words, is the reflection of nature.
Scroop’s comparison of the king’s presumptuous cousin to a natural catastrophe discloses an ideal of nature as ordinarily balanced, harmonious, and placid: there is no ``unseasonable’’ without seasonable.
He came unexpected, has been irresistible, and in his excess of ambition is like a surfeit of nature itself, an over-going of ``his limits’’ as monstrous in its unseasonable appearance as the normally predictable flow of rivers suddenly rising up to ``drown their shores’’ (and, perhaps, as monstrous rhetorically as the comparison of flooding rivers with swords and stubbornness).
The image of nature gone awry speaks powerfully to the theme of legitimate kingship that runs so prominently throughout \emph{Richard II}.
While the king volubly identifies his royal mandate with the divinely ordered natural world, Bolingbroke’s seizure of the throne occasions more than one prophesy of the bloodshed it will precipitate.
Most vivid is that of Carlisle, who warns that ``tumultuous wars /  Shall kin with kin and kind with kind confound.
/ Disorder, horror, fear, and mutiny / Shall here inhabit, and this land be call’d / The field of Golgotha and dead men’s skulls’’ (4.1.140--144).
Such prophecies carry all the more force, because they are not merely idle threats.
History has borne them out, as has Shakespeare’s own prior series of historical plays.
``Real’’ histories, however, were also prone to report supernatural affiliations with political events.
Echoing Scroop in Shakespeare’s play, the Jacobean historian John Speed credits s similar watery catastrophe:
\begin{bq}
Fearefull were the tragedies which ensued these times; and heare now what is written of some Portents or wonders, presaging the same.
The Bay or Laurell trees withered over all England, and afterward reflourished, contrary to many mens opinion; and upon the first of January, neere Bedford towne, the river between the villages of Swelston and Harleswood where it was deepest, did upon the sodaine stand still, and so divided it selfe, that the bottome remained drie for about three miles space, which seemed (saith Walsingham) to portend that revolt from the King, and the division which ensued.
\footnote{John Speed. \emph{The History of Great Britaine}. London, 1611. 608.
Also reported by John Stow, the division of the river signifying the division of the kingdom.}
\end{bq} 
Natural calamity combines with tragedy to presage, retroactively, history.
What Scroop’s elevated language and Speed’s fanciful account together illustrate is the mentality that both history and poetry shared in early modern England.
The ``historical imagination’’ of the period combined the desire to narrativize and thus experience the past with the poetic license required to paint of that past a useful picture.
History was, at the time, something like the natural philosophy of politics: part description, part interpretation, and always seeking to admonish the complacencies of the present in relating the faults and deviancies of the past.
To accomplish these ends, documentary evidence had to be supplemented with flourishes of creativity: lacking first hand accounts, almost anything might be admitted, and what was missing was often contrived.
On the other hand, such was the indeterminacy of the past that passionate debates about the reliability of sources and the conclusions that ought to be drawn from them made history seem inchoate and malleable.
History, it was beginning to be realized, is not the product of mere fortune or an unfathomable, providential intention.
History emerges from human imagination as much as the human actions of which it consists.
The early modern stage, then, offers an ideal lens for examining how the dramatists of the period reflected on history’s inherent poetic qualities.
Among Shakespeare’s ``history plays’’, \emph{Richard II} is the most conspicuously involved in opening history up to a poetic sensibility.

\emph{Richard II} has an ironic relationship to the historical imagination of early modern England.
On the surface, the play appears well in keeping with the moralistic strain of traditional historiography.
A corrupt king is punished for stereotypical excesses, meeting a deserved end so that the social order can be reintegrated.
Richard’s tragic demise could easily be read as this sort of cautionary tale for rulers, as Holinshed explicitly recommends\footnote{Richard's deposition is glossed by Holinshed as a special case worthy of attention (and heavy-handed moralization): ``Thys ſurelye is a very notable example, and not vnworthye of all Princes to bee well wayed, and diligently marked, that this Henry Duke of Lancaſter ſhoulde be thus called to the kingdome, and haue the helpe and aſſiſtance (almoſt) of all the whole realme, which perchaunce neuer thereof thought or yet dreamed, and that king Richard ſhould thus be left deſolate, voide, and in deſpaire of all hope and comfort, in whom if there were anye offence, it ought rather to bee imputed to the frayletie of wanton youth, than to the malice of his hart: but ſuch is ye deceiuable iudgement of man, whiche not regarding thyngs preſent with due cõſideration, thinketh euer that things to come, ſhall haue good ſucceſſe, and a pleaſante delectable ende.’’ Raphael Holinshed. The chronicles of England, Scotlande, and Irelande. Vol. 2. London, 1577. 1107--108.}, or, more generally, as a rendering of the allegory of fortune that is so marked in medieval writing.
In this view, the play that Shakespeare wrote does not participate in the fervid historicism transforming early modern intellectual life but merely disposes historiographical material to serve rather conventional literary ends: the commentary on events provided by the historiographers, in its own context dissociated from its subject, becomes in his dramatic writing fully realized in the condensed plots and overwrought characterizations required by its far less discursive composition.
That is, the role of the literary text is to take what must be interpolated into strictly historical accounts---motives, psychological complexity, sayings and speeches---and seamlessly integrate (or narrativize) it into their literary derivatives.
While the arrangement of historical content into literary form might be lauded for giving rise to interesting novelties in terms of plot and rhetoric, novelties for which Shakespeare is particularly well known, such a perspective neverthless relegates such appropriations to an isolated domain of literary effects.

This is, however, to separate histories too much from the literary energies enlisted in their composition and the generally ``poetic’’ quality of much historiographical writing in the early modern period.\footnote{My intention here and following is not to argue that history is poetry and poetry is history, as if distinctions do not matter and as if early modern writers had no sense of or opinions about their differences, but rather that their practical uses of the terms, as well as the overlaps in their perceived utility, means that neither is prior to the other.
The very fact of their intellectual adjecency---unthinkable now---surely made any attempt to define one in terms of the other (even if in opposing terms) predicated on the overarching rhetorical system, the quintessential Renaissance archetext, in which they both partook.
I bear in mind throughout Sir Philip Sidney’s perhaps offhanded remark that ``many times [the historian] must tell of events whereof he can yield no cause; or, if he do, it must be poetically’’ (Sir Philip Sidney. A Defence of Poetry. Oxford: Oxford University Press, 1966. 36). This is more often than not the case.} While the utility of categorizing Shakespeare’s play as history, a version as good as any other, is questionable, it is certainly the case that history in general performed different kinds of work at the time he was writing than we expect of it now.
The usefulness of a particular account might depend more on contemporary applicability than evidentiary accuracy.
And if ``truth’’ was purportedly the goal of history writing, then as now, a moral truth might have been just as good as a ``factual’’ one.
Indeed, historical texts had to carefully negotiate the contradiction between their potentially subversive discoveries and the conservative justifications of the present that were often required of them.
This typically led them to enlist the tools of poetry: in the invention of intellectual genealogies, for unsubstantiated narrative exposition, or even to provide cover for dangerous ideas.
Genre boundaries were less carefully looked after, as well, and drama especially was comprised of borrowings from chronologies, romantic ``histories,’’ verse, and non-dramatic \emph{de casibus} tragedies.
In particular, history and tragedy, as varieties of literature, share many correspondences through the early modern period, before history embarked on its distinctively non-literary course.
Simply put, \emph{Richard II} can be categorized as a history play, because it happens to dramatize historical material, but it can also be categorized as a tragedy, because that material entails the fall from grace of a heroic figure.
In its time of initial production, it was labeled a tragedy despite containing history, because it was then most readily compared to popular stories about the falls of princes.
Later, when it appeared in the First Folio, it was grouped with the histories, because it was only in retrospect that it could be identified as part of a once-fashionable dramatic trend.
A question we are forced to ask is how glib these labels were and, to the contrary, how useful they are in helping us to parse their intellectual status in the early modern literary field.

Worth keeping in mind, as well, is the fluid state in which history found itself at the turn of the seventeenth century.
Texts of long standing authority and the institutions and organizing principles of society based on them were exposed to so much correction, contextualization, and overall scrutiny that an Elizabethan subject could be forgiven for concluding that the world is made entirely of self-authorizing fictions.\footnote{A well-known and representative example is the ``discovery’’ of feudalism in the early modern period and the corresponding insight that human institutions are neither static nor natural but the product of ongoing contingency and negotiation.
Such inquiries encouraged an intellectual climate that was moving from the rote transmission of knowledge toward curiosity about the underlying principles governing human conduct, an attitude that would have primed well-informed theatricial audiences for the critical attitudes toward cultural shibboleths that characterizes early modern historical drama: ``Study of the feudal law was peculiarly calculated to cause men’s minds to pass from reflexions on the forms of the law to inquiry into the social and economic realities which underlay them.’’ (J. G. A. Pocock. The Ancient Constitution and the Feudal Law: A Study of English Historical Thought in the Seventeenth Century. Cambridge: Cambridge University Press, 1987. 81)} Most relevant to \emph{Richard II} is the fiction of the absolute power of monarchs, a Renaissance idea spuriously projected back onto a period during which the king’s actions were far more circumscribed.
Indeed, it was treated by the Parliamentarians of the Civil War period as ``an innovative phenomenon to be resisted by an appeal to deep-rooted constitutionalist traditions.’’\footnote{David Norbrook. ``‘A Liberal Tongue’: Language and Rebellion in Richard II’’ in Kirby Farrell, Ed. Critical Essays on Shakespeare’s Richard II. New York: G. K. Hall \& Co., 1999. 121--34. 122--23.} Also, as a historical figure, there are few kings as archetypal as Richard II.
Even in his own time, he was regarded (at least by his detractors) as a fantastical figure: effeminate, unmanly, disgracefully unwarlike, stubbornly adolescent in his behavior, addicted to pleasure, overly concerned with clothing and appearance, and prone to favoritism.
He is not entirely dissimilar to Bolingbroke’s ``unthrifty son’’ and possibly, as we will see, just as conscientious about his self-presentation.
This attitude toward Richard, as discussed by Christopher Fletcher, continues to the present day, inasmuch as we assume Shakespeare’s attitude toward the king must have somehow been unilaterally informed by the common opinion about him circulating at the time, which was itself comprised of numerous, contradictory sources.
The complex array of rhetorical strategies in which documents from Richard’s reign and its aftermath are engaged, however, makes dubious any attempt to authenticate his traditional reputation as childish and dissolute: ``it seems entirely possible to argue that this picture of the king, far from representing an exaggerated version of reality, represents almost the inverse of the truth.’’\footnote{Christopher Fletcher. ``Manhood and Politics in the Reign of Richard II.’’ Past and Present 189 (2005): 3--39. 39.} The question here is whether or not Shakespeare’s writing could have risen above Richard’s proverbial status as a figure of incompetence and tragedy.

\emph{Richard II}, as a special case, seems designed with these very questions in mind.
More explicitly than any other so-called ``history’’ play, it plays on the unstable boundary between history and poetry.
History, of course, could mean ``story’’ rather than ``inquiry’’ strictly speaking.
And tragedy, though a poetic category itself, could easily migrate into the subtext of rhetorically ``factual’’ accounts of the past in its more general meanings.
These categories, however, tended to be as interchangeable as they were malleable.
Enough contexts existed for differing uses of history and poetry that it is only by anachronistically inserting ourselves into what was even then a debate\footnote{If the terms were not exactly used interchangeably, the debate over the relative merits of history and poetry, which goes back to Aristotle and was picked up by Renaissance scholars, indicates their close semantic relationship.
The fact that it even was a debate, that history and poetry could be opposed, registers the profound distinction between intellectual life then and now.
Nobody today would seriously argue that poetry is a superior discipline to history much less that it is a discipline, in any way equivalent, at all. This, I think, is remarkable.
See Donald R. Kelley. ``The Theory of History’’ in Quentin Skinner, Eckhard Kessler, et al. Eds. The Cambridge History of Renaissance Philosophy. Cambridge: Cambridge University Press, 1988. 746--61.} that we can discover stable meanings for either of them.
Rather than determining whether or not a play like \emph{Richard II} would have ``counted’’ as history at the time is less interesting than accepting that it did, in fact, participate more than casually in historical discourse and investigating what might have been its contribution.
As far as this goes, it is admittedly natural to see the play as a rather commonplace figuration of the tragic in history, a figuration typical of so many history plays at the time.
This chapter will argue, however, that another reading is possible, one that takes these entirely legitimate---but more obvious---interpretations into account but moves one step beyond them to reveal an even more interesting text than previously considered.
\emph{Richard II}, I contend, is a play that takes the ferment of early modern historical thinking for granted and as the basis for a fascinating theatrical game, a game in which Richard and Bolingbroke’s contention to dominate language exposes the poetic furnace at the heart of historical discourse.
At its most subtle, the play functions at a level of abstraction layered above the assumptions we might normally make about the properties of early modern historiography and thereby recasts them in new and profound ways.
Most of all, it ironizes the act of thinking historically by asserting the constructedness of historical representations through a relentlessly poetic form; by dramatically undermining its own didactic subtext; and by deploying in Richard a character who seems to be self-consciously aware of his historiographically contrived, and thus hopeless, situation.

When Richard finally takes the full measure of Bolingbroke’s triumph, he oscillates back and forth between postures of sweet despair and royal afflatus.
``Mock not my senseless conjuration, lords,’’ he says at first, ``This earth shall have a feeling, and these stones / Prove armed soldiers, ere her native king / Shall falter under foul rebellion’s arms’’ (3.2.23--26).
He is in his conception both Cadmus, the nation-founding sower of dragon’s teeth, and Orpheus, whose music charmed flung stones to fall harmlessly at his feet.
He envisions himself, in his royal capacity, as coextensive with nature and speaking a king’s language of command, command over nature, a language that sets the limits of language itself.
This language may seem the issue of a deranged, self-absorbed mind, but it also produces the necessary fictions that make kingship possible.
But the assertion of divine authority---indeed, its overexertion---always entails a denial of the truth of actual human frailty, and this contradiction produces in Richard, in the form of his utterances, a divided mind.
The artificiality of his performance is at loggerheads with the suppressed truth of his divinely-sanctioned authority, which is only divinely-sanctioned for as long as people believe it is.

The function of this play is to make that performance stick out.
In a fictional world delineated by poetic meter, Richard’s verbiage comes across simultaneously as the native idiom of this world and a cancerous excrescence that overdevelops it into something monstrously excessive.
Like the Player Queen in \emph{Hamlet}, Richard protests too much.
From his boasts that stones will fight on his side and that God has angels employed on his behalf (3.2.58--62), he blanches at the first bad news, suffering a synecdochic anemia as ``the blood of twenty thousand men’’ rushes out of his face.
Reassured by Aumerle, he picks himself up again: ``I had forgot myself, am I not king? / Awake, thou coward majesty, thou sleepest.
/ Is not the king’s name twenty thousand names? / Arm, arm, my name!’’ (3.2.83--86).
His confrontation with historical necessity, the implacable inertia of human events that defies all attempts to contain it with the idols of dogma, provokes his sublime confrontation with his split self.
He is a mortal king facing a usurper (a crisis more historically banal than dramatically apposite), but he is also the ``majesty’’ that his mortality imperfectly embodies (a crisis in the poetic construction of his self, which must face the historical-dramatic crisis).
Under the stress of enacting that majesty, his imagination explodes.
He talks to it.
He calls it to arms.
From actual soldiers, to their hyperbolic quantification, to animated rocks, he finally arrives at the quilting point of them all, of all his imaginative constructs.
As if reaching the pinnacle of subjective crisis, he perceives his very identity shattered, like the mirror in his deposition, into twenty thousand pieces, only to be reassembled extemporaneously into a new ideal form, like some many-limbed Eastern deity: Richard as Avalokitésvara.
This moment of apotheosis, however, does not last.
More bad news follows---the return of historical reality---and along with it the culmination of Richard’s solipsistic philosophizing on the pattern of history he is trapped in.
Sentient in this moment of clarity to the extra-dramatic circumstances of his inscription into history, Richard delivers his most famous ode to the despair of kings: 
\begin{vq}
Let’s talk of graves, of worms and epitaphs, \\
Make dust our paper, and with rainy eyes \\
Write sorrow on the bosom of the earth. \\
Let’s choose executors and talk of wills; \\
And yet not so, for what can we bequeath \\
Save our deposed bodies to the ground? \\
Our lands, our lives, and all are Bullingbrook’s, \\
And nothing can we call our own but death \\
And that small model of the barren earth \\
Which serves as paste and cover to our bones. \\
For God’s sake let us sit upon the ground \\
And tell sad stories of the death of kings: \\
How some have been depos’d, some slain in war, \\
Some haunted by the ghosts they have deposed, \\
Some poisoned by their wives, some sleeping kill’d, \\
All murthered. (3.2.145--160)
\end{vq}
Characteristic of the world-making language exemplified by Richard throughout the play is his repetition of grammatical structures.
Here, they are the hortatory phrases that describe this moment as they look at the same time toward their own abstractions as general truths.
``Let’s talk of graves\dots’’ refers to the foreboding he feels in the present for an end that is inevitable no matter who holds the throne.
``Let’s choose executors and talk of wills\dots’’ gloomily forecasts a future in which Richard imagines himself stripped of his trans-human qualities, of the twenty thousand names that make him both more than human but also reaffirm that he is human (much like the superfluity insisted on by Lear)---of everything that makes him more than a forgotten corpse.
His is a voice from history imploring on his audiences generations hence to rescue him from history’s power of turning fiction into truth, a power that we see slipped from his grasp.
In both of these phrases, he refers to writing.
He speaks of writing in the dust with tears and writing out wills for the next generation, suggesting the writing of an account of present sorrow for the sake of future edification, of writing a history not out of words but out of affect.
This briefly considered possibility is undercut, however, but his realization that Bolingbroke---more a symbolic function than anything else, at this point, of history’s uncontainability---sets the terms of history now, or at least Bolingbroke thinks he does.
This is, in a way, his first lesson to Bolingbroke.
Later, he will have the chance to school his rival directly in what history actually makes out of the human desire to control it.
As we come to the last of the ``Let’s’’, we also approach Richard’s most succinct encapsulation of history: kings deposed, slain, haunted, poisoned, killed in their sleep, ``all murthered.’’ Here, we see even more repetition, compressed so as to give point to that which is a king’s only destiny in history: death.
It is, perhaps, not a coincidence that so many of Shakespeare’s own kings suffered such fates.

There is also in this speech an interesting play on the word ``deposed.’’ Richard repeats it three times (and once more earlier in the scene).
While, as many critics have pointed out, it does indicate the king ``self-indulgently anticipating total defeat’’\footnote{William Shakespeare. King Richard II. Ed. Charles R. Forker. Arden Shakespeare (Third Series). London: Arden Shakespeare, 2002. 30.}, it also, as another instance of repetition woven into the text, figures the language that restrains him.
Richard’s fate, in chronologic and dramatic history, is to be both overthrown king and dead body, both senses of ``deposed’’ facts of history that will continue to haunt Bolingbroke, continue to define Richard’s posterity, and together comprise the archetype of what he describes as the general fortune of all those who presume ``to monarchize, be fear’d, and kill with looks’’ (3.2.165).
Like a deponent verb in Latin, which has a related meaning, Richard’s deposition is passive in form but active in meaning.
It is part of the historical irony according to which Shakespeare characterizes \emph{Richard II} that he elucidates this historical affectivity himself.

One might usefully compare Richard’s highly nuanced subjectivity in this scene with Gaunt’s comparatively wooden sensibility.
Gaunt, who has his own anthology-ready, set piece oration earlier in the play, seems to exist only to die.
As the upholder of traditional values, he is a foil to Richard’s impetuosity.
And since Bolingbroke is his heir, his death sets the rest of the plot in motion.
We should otherwise wonder, given his early demise, why he is in the play at all.
Gaunt’s famous lamentation, however, for an idealized England that never was manages to graft a genuinely nationalistic sentimentality to a wistful, perhaps even cynical, despondency that his effusive encomium is itself, partly because he dies so early on, a failed historical fiction.
As the rest of the play will bear out, Gaunt’s conservative picture of his country’s exceptional nature is exactly the sort of beautifully bland fantasy that canny rulers would like their subjects to believe without truly believing it themselves.
The promise of renewal brought by Bolingbroke, who is hardly the sort to indulge in his father’s brand of sententious mythologizing, is predicated on exactly this image of the golden past, of a history that demands restoration.
Bolingbroke, however, will come to have his own reckoning with the inevitable contradictions inherent in such attempts at language control.
The most ironic deployment of this rhetorical strategy will, of course, be demonstrated by his son, Henry V.
The newly-crowned Prince Hal, who is well-practiced in his own, personal style of restoration based on an imagined past (one that he partly borrows from Hotspur), raises the fiction-spinning of historical revision to its most spectacular pitch.
Much like \emph{Richard II} if through different means, \emph{Henry V} formalizes the dramatic, and therefore suspect, nature of historical discourse.
And as we are with Richard II’s gorgeously-wrought despair, we are easily seduced by Henry V’s grandiose bravado---until he too ends up as a corpse on a stage, the basis for yet more historical nostalgia, yet another ``deposed’’ king.\footnote{Historical nostalgia of the variety I am describing was no trivial matter in early modern England.
In fact, much of the cultural character of what we call ``early modern England’’ in order to set it apart from ``medieval England’’ is the result of debates about history---institutional, legal, monarchical---that were happening at the time.
The Protestant Reformation, on the one hand, and the abolishment of the monarchy on the other, were both predicated on the ``revival’’ of what were perceived to be historically-supported truths that had somehow been lost or suppressed, as if historicity were a necessary prerequisite for challenging power and instantiating social change.
Ironically, both of these revolutionary cultural shifts, each at either end of what is often studied as a coherent period, had to do with historically-informed interpretations of the royal office: practical, in the former case; philosophical, in the latter, though each dependent on opposite conclusions about the contradiction inherent in the nature of the king.
The extent to which early modern political theology, especially as described by Ernst Kantorowicz, is informed by literary (and other) fictions is discussed by Victoria Kahn.
``Political Theology and Fiction in The King’s Two Bodies.’’ Representations 106.
(2009): 77--101.
John Milton cites Holinshed in The Tenure of Kings and Magistrates as providing a precedent for the trial of Charles I.
Milton, a not unsympathetic reader of Shakespeare, may himself have read history through the filter of literature.
When the deaths of kings are involved, we are longer speaking merely of ``influence,’’ though, as the historical enterprise in early modern England received additional stimulus from the printing and performance of dramatic literature (as testified by the strange career of John Haywood’s  history, perhaps influenced by Shakespeare’s play, The first part of the life and raigne of King Henrie the IIII).
Shakespeare’s play clearly impinged on the historical thinking of the time, though the history it treated could end up used in a variety of ways.
King James could find justification for divine right and constitutionalists that right’s limitation.
Both notions were historical fictions improvised in this period on the basis of the same patchwork of inconsistent, fragmentary, unreliable texts we have available to us now.}

By looking at the play as a commentary in form and action of the nature, limitations, and frustrations of history, we have the opportunity to consider some concrete ways in which literature itself shaped, sublimated, and often short-circuited historical thinking.
The alternative, which I propose to reject, would be to regard the play as operating in a closed, literary system that merely reflects the ``real’’ symbolic structure of its context without any feedback returning the other way.
Even the historicist arguments which are most sympathetic to literature tend to presume upon a ``culture’’ that informs it, even if this culture is only resolvable as an intellectual back-formation.
This is, admittedly, a difficult bind to think oneself out of, since all historical interpretation, including the interpretation of literary texts from a discrete ``past’’, is an unavoidably subjective enterprise.
Even in the case of the New Historicism, the set of reading strategies most interested in situating the speech acts of literature within an ideologically neutral and genre agnostic matrix of cultural practices, there is no escaping the modeling of literature as a projection of some other, more fundamental, background.
Alan Liu goes even further, suggesting that a New Historicist is really ``a subject looking into the past for some other subject able to define what he himself, or she herself, is; but all the search shows in its uncanny historical mirror is the same subject he/she already knows: a simulacrum of the poststructuralist self insecure in its identity.’’\footnote{Alan Liu. ``The Power of Formalism: The New Historicism.’’ ELH 56.4 (1989): 721--71. 733. For another early critique of the New Historicism, see also Jean E. Howard. ``The New Historicism in Renaissance Studies.’’ English Literary Renaissance 16 (1986): 13--43.} Liu’s critique of the New Historicist desire to encounter the past both in its unmediated plenitude and with full cognizance of the inescapability of an analytical subjectivity grounded in the present offers a useful perspective on early modern historicism.
Then, as now, the effort of the imagination to bring the past closer is a species of intellectual wish fulfillment, but it is also an integral component of modern subjectivity, in general.
Shakespeare’s plays very much seek to answer this innate desire to commune with our own history while at the same time foregrounding what that means on an affective level.
The slippage between historical objectivity and poetic subjectiveness is marked throughout the ``history of the history’’ of Richard II, the king.
The narrativizing of his reign by contemporary and near-contemporary commentators, their Renaissance redactors, and all of this material’s modern inheritors has turned his life into paper: there is little about this king, as Fletcher shows, that is not the product of one fiction or another.
And it is undeniable that the ``fiction’’ of Richard II served as a model for other monarchs seeking to articulate their own representative functions.
Literature and history were equals in this regard.
If Elizabeth I could exclaim, ``I am Richard.
know ye not that?’’ while looking over a history with the antiquarian William Lambarde,\footnote{John Nichols, Ed. Bibliotheca Topographica Britannica. Vol. 1. London, 1780. 525.
For a thorough overview and defense of the oft-questioned authenticity of Lambarde’s conversation with Elizabeth, see Jason Scott-Warren.
``Was Elizabeth I Richard II?: The Authenticity of Lambarde’s ‘Conversation.’’’ The Review of English Studies 64.264 (2013): 208--230.
Some critics, however, are reluctant to accept that theater was as important a mode of discourse as many of their colleagues, especially New Historicists, would like to believe.
Touching on Elizabeth’s famous statement, in particular, see Leeds Barroll.
``A New History for Shakespeare and His Time’’ in Critical Essays on Shakespeare’s Richard II. Kirby Farrell, Ed. New York: G. K. Hall \& Co., 1999. 93--120.} it was no more surprising to find Charles I, on the eve of his execution, alluding to Shakespeare’s play directly.\footnote{Sean Kelsey. ``The Death of Charles I.’’ The Historical Journal. 45.4 (2002): 727--54.}

Indeed, as a dramatic meditation on the late Elizabethan fascination with vernacular history, \emph{Richard II} is perhaps Shakespeare’s strongest and most artistically consistent statement.
Having completed a kind of dramaturgical apprenticeship in the history play genre with the first tetralogy, he seems to consolidate his art with the first installment of the second.
With \emph{Richard II}, that is, Shakespeare moves beyond the chronicle-driven plots and hackneyed moralisms characteristic of earlier history plays and toward a more subtle style of dramatic representation, a style that saturates history’s raw materials with something approaching a historiographic philosophy.
\emph{Richard II}, in particular, demonstrates how history is a poetic artifact created by human will.
This is reinforced by the form of the play as well as its treatment of its historical content.
On the one hand, it is written entirely in verse.
There is no varying of registers, as we find in the later plays of the second tetralogy, and no prose interludes as we find in most of the plays Shakespeare wrote.
Much of the language is affected, as well, with frequent usage of couplets, embedded stanzaic forms, rhymes, epic similes, and set piece speeches.
On the other hand, the dramatic action is consistently formal in nature as well as structure, with the two ritualistic (and mutually parodic) judgment scenes that bookend the play pivoting on the climactic deposition scene at the center.
They even meet having arrived at the center of their contention, England, from geographically opposed directions: Richard from Ireland, Bolingbroke from France.
An intriguingly post-colonial dimension to the play hides here and possibly alludes to the two most prominent sites of English imperialist ambition before the early modern period.
It is certainly a subtext of the play that, amidst all the highly wrought speculation about the symbolic nature of the monarch, whoever happens to be occupying the throne inevitably leads the kingdom to war.
After four plays in which the crown is passed around like a prize in an aristocratic game, \emph{Richard II} ruminates almost comically on the emptiness that it encircles.
Once again, \emph{King Lear} will bring these ideas to their ultimate conclusions.
Bolingbroke coming from France as a savior figure, but also as a usurper in a military action, almost looks like Cordelia.
In Lear, however, there are no saviors and there is no historically informed moral to be drawn.
The idea that royal authority means anything at all is completely annihilated.
\emph{Richard II}, however, has not yet gotten there, though there are enough interesting correpondences that a case can be made for comparing them.

What Shakespeare may have realized---or decided---with this play is the potential of having characters perceive themselves as actors in history: a kind of irony of anachronism that allows for richer poetic possibilities than a mere versified rehearsal of nationalistic prejudices and popular myths.
What makes \emph{Richard II} particularly successful as a history play is the extent to which the often opposed genres of which it is comprised, history and poetry, are seamlessly integrated.
The entirety of the play’s ``historical’’ content is delivered in verse, and the action of the play is pointedly focused on the events leading up to, catalyzing, and falling out as a consequence of Richard’s abdication.
There are no subplots, comic interludes, or scenes written in prose to offset the poetic ceremoniousness in which everything that happens in this play is couched.
By these means, history is presented in the play, in fact, as an effect of poetry, a rhetorical remainder that indicates how mistaken we are to privilege the historicity of poetry over the ``poeticity’’ of history.
If it is indeed the office of poetry to body forth the forms of things unknown, what Shakespeare produces with his history plays are real bodies purporting to stand for the unknown dramatis personae of the past.
If there is a certain built-in irony to the retrospective representation of historical events, Shakespeare exploits that irony fully.
Insofar as his characters trope quite happily on their configurations as theatrical spectacles, they are able commentators, if obliquely, on the ideological underpinnings of their modes of representation, too.
Richard II may be the ultimate meta-theatrical monarch, reveling in the performance of his power, but he also serves a function that is distinctly meta-historical.

Much has been written about the meta-theatrical qualities of Shakespeare’s dramatic writings.
Inasmuch as his plays were meant in a literal sense to be staged, in their ironic gesturing toward the histrionics of daily life and, a fortiori, the performative natures of power and politics, they also stage us---as witting or unwitting actors in the drama of human existence.
The analogy of the world with the stage that extends across his work, however, is not the sole means by which his characters signal the constructedness of their (and thus our) reality.
Historical self-consciousness, an awareness that societies are not static but grow out of their own pasts, also contributes to the intellectual landscape of his plays.
In the learned society of early modern England, as elsewhere in Europe at the same time, the concept of ``history’’ was reaching a turning point.
If the relationship of the present to the past was the subject of keen interest during the Renaissance, the literature of the time reflected the emerging awareness of the significance of historical difference.
While the recognition of historical difference by itself might mean little, writers and thinkers were beginning to consider history more as a series of problematics than as a basis for reductive moralizing or revisionist nationalism.
They began looking at history critically, seeking both to authenticate their sources and derive arguments from them about the nature and origins of human institutions.
New kinds of texts appeared to challenge the chronicle form and the reduction of history to moral lessons useful to the present.
In many ways, however, history was still regarded as a cousin to poetry, and historical accounts provided raw material for texts engaged in more imaginative pursuits.
This literary work, however, was no less interested in a critical analysis of both history itself and the historicist sensibility then in ascendance.
History as a mere chronicling of events or stockpiling of exempla is confronted in Shakespeare by its more affective dimensions.\footnote{For an excellent account of the ``theatrical affects’’ deployed in Richard II, particularly with regard to the bodily experience of hearing in performance its highly versified language, see Brian Walsh.
``The Dramaturgy of Discomfort in Richard II’’ in Richard II: New Critical Essays. Ed. Jeremy Lopez. Shakespeare Criticism Volume 25. New York: Routledge, 2012. 181--201.} Or it might be more accurately stated, at least in the case of \emph{Richard II}, that Shakespeare critiques the status of history as an affective attachment: how we perceive it, how we instrumentalize it, and what utility it has for better understanding our own subjectivity.
Just as allusions to theatricality in the plays demystify the structure of human relations, dramatic representations of historicity interrogate the necessity and origins of those relations.
In different ways, Shakespeare’s dramatic works communicate his searching response to the persistent question, ``What is history?’’ In \emph{Richard II}, he shows the uselessness of trying to master it, of answering that question in hegemonic terms.
In protesting too much for triumphalist teleologies or golden ages to come, the poetic histories of early modern England encoded within themselves the markers of their own instability.

From very early on, usually in the idiosyncratic depiction of certain figures (Henry VI, Margaret of Anjou, Richard III) Shakespeare has his characters engage in historical-literary reflection that is both contained within historical action but also rises above it.
This is partly a function of the subjectivity his characters so persuasively enact: their extra-dramatic perspective on themselves and the events occuring around them dovetails with ours as we become ``readers’’ of the dramatic action along with them.
The nascent (and never absent) form of this reflection typically manifests as foreshadowing in curses, prophecies, and proleptic summaries of future events disguised as homiletic predictions or political disquisitions.\footnote{The significance of prophecy in historiography---and the skepticism to which Shakespeare subjects it---is explored at length by Line Cottegnies.
With particular reference to the three Henry VI plays, Cottegnies discusses how Shakespeare’s dramatic historiography ``reflects contemporary anxiety towards prophecy, but also manifests a sceptical view of historical transcendence, foregrounding the conflict between individual agency and providential pattern.
It also poses the question of interpretation as authorial irony can be perceived through the use of polysemous riddles.’’ Line Cottegnies. ``Lies Like Truth: Oracles and the Question of Interpretation in Shakespeare’s Henry VI, Part 2’’ in Cottegnies, et al, eds. Les Voix de Dieu: Littérature et prophétie en Angleterre et en France à l'âge baroque. Paris, France: Sorbonne Nouvelle, 2008. 21--34. 24.
As I have been arguing, it is this ironic tenor that Shakespeare seems to bring to the entire question of making sense of human events, perhaps on the basis of prophecy’s fallen edifice and at the expense of a historiographical didacticism that would deliver as the guide to moral or political action a history devoid of affective consequences.} In later plays, however, reckoning with history becomes a more subtle affair: characters struggle against the historical inevitability already written into their futures---they are, literally, always at the growing edge of what is about to be, what they are about to become---as if confronted by a terrible, unavoidable destiny.
The character of Henry VI is one of Shakespeare’s earliest forays into this form of prophetic anachronism.
The king in Henry VI, Part Three dies with a prophecy on his lips:
\begin{vq}
That many a thousand, \\
Which now mistrust no parcel of my fear, \\
And many an old man’s sigh and many a widow’s \\
And many an orphan’s water-standing eye--- \\
Men for their sons, wives for their husbands, \\
And orphans for their parents’ timeless death--- \\
Shall rue the hour that ever thou wast born. (5.6.37--43) \\
\end{vq}
He is, of course, speaking to the Duke of Gloucester, the future Richard III, the character whose historically predetermined villainy is so pronounced, he becomes a caricature.
But we see in Henry’s dying doom the outline of Richard II’s own imprecations against Bolingbroke.
Like Richard, Henry speaks in heaps of paratactic language.
As if to reinforce his soothsaying through the rhetorically ritualistic repetition of parallel elements, he piles on every possible contingency, followed up by an interminable accounting of the evil portents surrounding his murderer’s birth.
His review of bad omens goes on for so long, in fact, that the impatient and impious Duke stabs the king mid-sentence, going so far as to admit the truth of these signs and oracles for the times yet to come.
Interestingly, given Richard II’s own last words, in which he calls on his soul to mount to heaven, Gloucester observes sardonically how Henry’s blood does not mount but sinks into the ground, ``down, down to hell’’ (5.6.67).

It is, perhaps, only in remembrance, only with the aid that history brings to memory, that kings mount to heaven or descend to hell.
Richard III, whose own deeds had long since been ``chronicled in hell,’’ is hardly meant to evoke much sympathy, but it is a question what role Shakespeare wanted us to believe that we have in determining the course of history in a world of dark omens and curses that seem always to bear their poisonous fruits.
Only the most simplistic reading of Henry VI’s prophecy would reduce it to the level of irony alone, of a recounting of things to come that are already known to have been.
At this stage of his career, however, it may be that Shakespeare had not yet matured enough as a writer to entirely avoid cheap plot devices, as he certainly has by the time of \emph{King Lear}, in which he narrativizes signs and wonders only for the sake of rejecting them as the products of witless imaginations.
In Richard III’s case, his inability to rise above his entirely scripted nature is partly due to his embodied form and partly because the historical inevitability in which he participates, a current that runs through the entire first tetralogy, is fragmented among many characters.
Even his body is a historically determined artifact.
He ``is not only deformed, his deformity is itself a deformation.
His twisted and misshapen body encodes the whole strategy of history as a necessary deforming and unforming---with the object of reforming---the past.’’\footnote{Marjorie Garber. ``Descanting on Deformity: Richard III and the Shape of History’’ in The Historical Renaissance: New Essays on Tudor and Stuart Literature and Culture. Ed. Heather Dubrow and Richard Strier. Chicago: University of Chicago Press, 1988. 86.} History, that is, is to be molded and shaped, not simply received.
Put another way, it cannot simply be received since it is necessarily made by our own molding and shaping of what we learn about the past.
Richard III’s body is the body we gave him, in a sense, and for the sake of explaining why history has the shape that it does.

The strategy of deforming, unforming and reforming history, I would add, is fully realized in the character who lurks behind him, almost like an alter ego, dogging his every move: Queen Margaret.
Margaret is memory in the first tetralogy.
Certainly, by the time of \emph{Richard III} the play, she has risen beyond her role as a supporting character to serve as a harbinger of historical necessity, if a simplistic variety of it.
As she tells Richard outright, she continues to exist only to make ``repetition of what thou hast marr’d’’ (1.3.164).
And repetition, as we see with Henry VI and Richard II both, is the form that world-mastering language frequently takes---always occurring in epic catalogues of deeds done, qualities ascribed, symbolic objects, names, etc.
Repetition also structures the history plays as a group, if we consider them parts of a greater historiographical project and not only as discrete units.
\emph{Richard III}, the last in the sequence chronologically but only halfway through in order of composition, ends quite pointedly with Henry, Earl of Richmond and the future Henry VII, promising an end to conflict:
\begin{vq}
England hath long been mad and scarr’d herself: \\
The brother blindly shed the brother’s blood, \\
The father rashly slaughter’d his own son, \\
The son, compell’d, been butcher to the sire, \\
All this divided York and Lancaster, \\
Divided in their dire division, \\
O now let Richmond and Elizabeth, \\
The true succeeders of each royal house, \\
By God’s fair ordinance conjoin together! \\
And let their heirs (God, if thy will be so) \\
Enrich the time to come with smooth-fac’d peace, \\
With smiling plenty, and fair prosperous days! \\
Abate the edge of traitors, gracious Lord, \\
That would reduce these bloody days again, \\
And make poor England weep in streams of blood! \\
Let them not live to taste this land’s increase \\
That would with treason wound this fair land’s peace! \\
Now civil wounds are stopp’d, peace lives again; \\
That she may long live here, God say amen! (5.5.23--41)
\end{vq}
As easy as Richmond’s speech is to read as Tudor triumphalism, it should not be glossed over that \emph{Richard III} is followed immediately by its namesake \emph{Richard II} and the dispute between Bolingbroke and Mowbray that set brother against brother, father against son to begin with.
Retroactively, it becomes ironic: the bloody days were only beginning.
Richmond’s language, here, is also characteristic of kingly speech.
The multiple appeals to God, the piling on of parallel phrases, and the illocutionary assumption that his words effect their intended actions---the reconciliation of the Yorks and Lancasters, the marriage of their heirs, peace in the nation, the fertility of the fields, are emblematic of royal language that endeavors to work its will on history.
The most unkindest cut of all is that following up Richmond’s victory with Richard II’s deposition undercuts what looks otherwise like the unambiguous procession of history into the hands of the Tudors and Elizabeth.
Marjorie Garber, in reflecting on the unstable boundary between the tetralogies, articulates the important question with which this strange dilemma confronts us: ``Which time takes priority here, ‘historical time’ or the time of theatrical history? The fact that the first comes second and the second comes first instructively problematizes the whole question of double time as it relates to the genre of the history play.
The first tetralogy predicts the second; the second also predicts the first.’’\footnote{Marjorie Garber. ``‘What’s Past Is Prologue’: Temporality and Prophecy in Shakespeare’s History Plays’’ in Renaissance Genres: Essays on Theory, History, and Interpretation. Ed. Barbara Kiefer Lewalski. Harvard English Studies 14. Cambridge, MA: Harvard University Press, 1986. 301--31. 323.}

Richard II, who arguably suffers a psychological deformity analogous to, if more imaginatively febrile than, his descendant, elevates this style.
As a character, he is a kind of masterpiece of exactly this sort of linguistic grandiosity.
While he shares with Richard III a terrible fate and infamous reputation, he comes much closer than his successor in history and predecessor in drama, and precisely because of the excessiveness of his language, to articulating the schizophrenic nature of his own circumstances.
For a period during which popular assumptions about free will might have wobbled uncertainly between the old dispensation and the new\footnote{For an account of how traces of the ``old faith’’ persisted in Protestant England, see Arthur Marotti.
``In Defence of Idolatry: Residual Catholic Culture and the Protestant Assault on the Sensuous in Early Modern England’’ in Redrawing the Map of Early Modern English Catholicism. Ed. Lowell Gallagher. Toronto: University of Toronto Press, 2012. 27--51.}, such deterministic plotting could have seemed simultaneously appropriate, in keeping with the fatalistic flavor of Protestantism, and deeply troubling, as it typically dabbles in the supernatural and robs from these historical actors part of their humanity, and perhaps plausibility, too: belief in predestination does not come packaged with any certainty as to what one’s destination is.
And this is the rub for Richard.
He behaves as if being the master of poetic idiom, of the form of his literary reality, makes him the master of that reality, too, like an Orpheus whose music could inspire stones to build cities and elicit the favor of hell itself.

The poetic idiom of \emph{Richard II}, however, does seem to reflect the imaginative diktats of its master poet.
Richard’s characteristic self-inflation through numbers is echoed by other characters who let large numbers colorfully substitute for sober accounting.
Gaunt admonishes Richard that ``a thousand flatterers sit within thy crown’’ (2.1.100).
York, exasperated by Richard’s disinheritance of Bolingbroke, sagely informs him, ``You pluck a thousand dangers on your head, / You lose a thousand well-disposed hearts’’ (2.1.205--6).
Then there are the twelve thousand fighting men reported by Salisbury inflated to twenty thousand by the king.
There is Bagot’s challenge to Aumerle, that ``I heard you say that you had rather refuse / The offer of an hundred thousand crowns / Than Bullingbrook’s return to England’’ (4.1.15--17) and Aumerle’s response that echoes Richard’s own speech: ``I have a thousand spirits in one breast, / To answer twenty thousand such as you’’ (4.1.58--59).
Aumerle seems almost to be parodying royal speech, though none can outdo even in that the royal tongue itself, as we see in Richard’s deposition, where he recollects how far he has fallen from when he ``under his household roof / Did keep ten thousand men.’’ (4.1.282--83) and compares himself to Jesus betrayed by Judas: ``So Judas did to Christ; but He, in twelve, / Found truth in all but one; I, in twelve thousand, none’’ (4.1.170--71).
Richard is also quite enamored of lengthy cataloging.
We have already looked at his rundown of the death of kings.
He is elsewhere even more additive in his attempt to capture language in its totality.

The most vivid examples occur before and after the deposition, when Richard reduces the mystery of his office to a series of fungible quantities.
He does this first at Flint Castle:
\begin{vq}
I’ll give my jewels for a set of beads, \\
My gorgeous palace for a hermitage, \\
My gay apparel for an almsman’s gown, \\
My figur’d goblets for a dish of wood, \\
My sceptre for a palmer’s walking-staff, \\
My subjects for a pair of carved saints, \\
And my large kingdom for a little grave. (3.3.147--153)
\end{vq}
He looks to be far from his typically magisterial attitude in this speech.
Behaving as if the ``name of King’’ were a piece of transferable property, or the kingship a job for which he is being made redundant, he lists the royal appurtenances and their more mundane equivalents for which he will exchange them on his way out the door.
There is something ironic about his little presentation, however, something that is almost like a joke, one made at Bolingbroke’s expense.
For Richard, like a subversive satirist or the madman who speaks truth (once again, a stance brought to its height in \emph{King Lear}), does not truly believe that royalty is merely a show of property.
Or, to be more exact, he knows that a king must act as if it is not.
The objects signified by the words mean nothing compared to the metonymic function of the words themselves: they all stand for ``king’’ even though none of them can make a king.
As with the small possessions of the religious ascetic, their representational affinities count for less than the numinous realm they point to, the sanctioning discourse that language can never directly touch but only ever approximate.
Bolingbroke, Richard implies, thinks the kingship is, in fact, property, like the property passed down from Gaunt, which he also claims as his.
The danger resides in too cynically giving the lie to the rhetorically-constructed nature of the royal persona, too easily forgetting that the ``king’’ is made up of words, not objects, and words do not enter people’s hearts quite as straightforwardly as goods can exchange hands.
Richard’s self-debasement is a revealing debasement of the royal office itself, a debasement that, from his perspective, echoes Bolingbroke’s transactional tactics.
On such a level, there is little separating a large kingdom from a little grave.

Later, in the deposition scene itself, Richard disclaims the throne in similar terms, making the same sort of metaphysical mockery of the legalistic proceeding that Bolingbroke has orchestrated and presumes sufficient for trading places with the king:
\begin{vq}
Now mark me how I will undo myself: \\
I give this heavy weight from off my head, \\
And this unwieldy sceptre from my hand, \\
The pride of kingly sway from out my heart; \\
With mine own tears I wash away my balm, \\
With mine own hands I give away my crown, \\
With mine own tongue deny my sacred state, \\
With mine own breath release all duteous oaths; \\
All pomp and majesty I do forswear; \\
My manors, rents, revenues I forgo; \\
My acts, decrees, and statutes I deny; \\
God pardon all oaths that are broke to me! \\
God keep all vows unbroke are made to thee! \\
Make me, that nothing have, with nothing griev’d, \\
And thou with all pleas’d, that hast all achiev’d! (4.1.203--17)
\end{vq}
There is again emphasized in this speech---with its repetitions, balanced clauses, and rhyming couplets---the constructedness of royal language, language that Bolingbroke mistakes for exchangeable objects and rewritable facts.
In addition to giving up his crown and sceptre, Richard expands on his prior catalog of kingly attributes by adding such things as can hardly have been demanded by a legalistic proceeding.
To his crown and sceptre he adds kingly sway, pomp and majesty.
To his manors, rents, and revenues he piles on his balm and sacred state.
Richard almost seems to be challenging Bolingbroke to minimize their function, to actually think himself superior to the world of words that Richard had fashioned as insulation for the royal prerogative.
What he claimed before could not be done on Earth, Richard now does himself.
But does he simply hand all of this over to Bolingbroke? It seems rather that he consigns these monarchical qualities to oblivion.
``Make me nothing,’’ he almost says, and in that ambiguous act of asking God to invert his relationship to the world, he also reveals to Bolingbroke the nothingness he has won for himself: the crown that is both heavy and empty.
Indeed, the performative qualities of Richard’s self-abnegation are exaggerated entirely for Bolingbroke’s benefit.
``Mark me,’’ he says to his presumptuous usurper, and then practically pantomimes his abdication.

Questions are raised by this: do these words have the real force of what they express? Is the legal proceeding engineered by Bolingbroke simply another fiction that Richard exposes, an inferior fiction for being so obviously a fabrication meant for history, so that its readers will accept as unimpeachable fact a deposition devoid of ideological doubt or emotional resonance? What does it do to the fiction of the king’s speech if the king uses it to negate his kingship? In this moment, a rupture is created in history, a rupture that makes all subsequent royal speech unstable and practically parodic of itself.
We get some sense of this parody in Act 5, when the Duchess of York calls King Henry ``a god on earth’’ (5.3.136).
A darker tone still permeates \emph{Henry IV, Part One} and \emph{Henry IV, Part Two}, in which the guilty king cannot sleep and Prince Hal ``usurps’’ him, too.
By the time we get, chronologically speaking, to the \emph{Henry VI} plays, the crown is nothing more than a token---until Richmond claims it for himself, with his own final (but not quite final) plea for historical rejuvenation.\footnote{It is worth pointing out that Shakespeare, in depicting these events and translating them into dramatic verse, would have encountered discrepancies in his sources without having any clear metric by which to gauge their relative accuracy, though he may well have only looked to their theatrical transferability.
The deposition scene itself, as Annabel Patterson points out, is inconsistently related by Holinshed and Hall.
Whereas Holinshed ascribes Richard’s forced abdication to an act of Parliament---which is actually typical of his historiographical practice---it is in Hall where we find ``a highly emotional and metaphorical speech by Richard complaining that he has not been permitted time to grow up and mend his ways.’’ Annabel Patterson.
Reading Holinshed’s Chronicles. Chicago: University of Chicago Press, 1994. 114--15.
The choices Shakespeare made when depicting history were no by means dictated by his sources, but it is significant on its own that he would have encountered history as something requiring representational and interpretive choices to begin with.}

As a king, Richard certainly comes across as a feckless and orotund ruler who abuses his power, thus losing the support of his people, and is outmaneuvered by a more capable strategist, thus losing his kingdom.
As a fully realized, Shakespearean character, however, he demonstrates an acute awareness of the sort of predetermined, historically moralistic narrative in which he acts.
He understands that the nature of his power is poetic, that it relies entirely on his ability to impose upon the world the designs of his own imagination.
In this regard, he is a poetic character---always trying to ``make’’ his world---trapped within a poem, a made thing.
That he realizes the artificiality of historical discourses gives him much occasion for the solipsistic reflecting on them, and makes him often seem alienated from all the other characters around him.
But like many prophetic figures, his knowledge does not, and cannot, save him.
Bolingbroke, in contrast, thinks he can master history from within.

Whereas Richard brazenly verbalizes his reality into being, we are left in the dark as to Bolingbroke’s motives and often his actions, too.
He does, indeed, master Richard, but it is Richard who, surprisingly, turns out to be the most prescient character in the play and the one from whom Bolingbroke comes to learn the most.
Like us along with him, Bolingbroke comes to realize the schizophrenic nature of being a king, of the necessity of trying to write what has already been written, of existing in a world of language that requires but thwarts ever getting a firm hold on it.
I agree with Ronald R.
MacDonald, who notes ``Richard’s distinct and unenviable role to enact and clarify certain paradoxes that do not stem from his idiosyncrasies, but are latent in the political order.
He must live those paradoxes for all to see, and not the least of these is the fact that the more he insists on his power as the anointed king, the less real power he wields.’’\footnote{Ronald R. MacDonald. ``Uneasy Lies: Language and History in Shakespeare’s Lancastrian Tetralogy’’ in Shakespeare and History. Ed. Stephen Orgel and Sean Keilen. New York: Garland, 1999. 58.} My only qualification would be that ``real power’’ is always out of reach of language and that Bolingbroke will discover this, too: it is not only Richard who tries to close the gap between words and world.
What I perceive in such aspects of the play is a layer of rhetorical abstraction once removed from the level of historical allegory.
There is the action of the plot, derived from history, and then there is the tenor of its treatment.
Perhaps the greatest irony unspoken by the play is the irony that proleptically reaches back from \emph{Richard III}, the history play Shakespeare wrote just prior to but also at the greatest remove from this one.
Henry Richmond’s call for an end to ``bloody days’’ looks differently, as we have seen, when immediately followed by all the bloody days that preceded them.
In such ways does Shakespeare ironize our faith in historical progress and put into doubt that anyone can take hold of history’s rudder and steer its course.
What he dramatizes in \emph{Richard II} is the king’s inescapable need to pose as the master of his own narrative.

The king, who at first thinks himself history’s master is instead mastered by it.
He is certainly the master of the play’s uniquely poetic idiom, and he delivers his mellifluous and often equivocating pentameters as if the poesis of his speech were tantamount to the poesis of his world---as if, that is, he were himself the progenitor of history itself, the source of its unfolding.
As the denouement of the banishment scene in Act I, he enjoins Mowbray and Bolingbroke to swear never ``to plot, contrive, or complot any ill / ‘Gainst us, our state, our subjects, or our land’’ (1.3.189--90).
His piling of words upon words is characteristic here as elsewhere: plot, contrive, and complot are as synonymous with one another as are state, subjects, and land.
And this injunction follows a lengthy sequence of other qualifications of their exiles, as if Richard could own their actions even beyond his jurisdiction, and an even lengthier scene during which the trial by combat is announced, called off, and then replaced with Richard’s improvised sentences.
One gets the feeling this was a pantomime planned in advance.
At a stroke, Richard eliminates both potential enemies, sets himself up to arrogate the wealth of the more dangerous one, and establishes some claim to be merciful.
Is he indecisive and fearful of confrontation or evincing a royal monopoly on the ontological power of speech? Words, in any case, are not enough, even in Richard’s world created by words, since the doom he places on Bolingbroke portends his own.
To the usual argument that Richard believes his destiny is foreordained by his birth, it might be suggested that his rhetoric is more strategic than naïve.
He certainly acts as if he has a linguistic power beyond mortal capacity, as if his every speech were like a soliloquy addressed directly to God, but perhaps he is only performing this rhetorical posture deliberately.

It is not without good reason that criticism about Richard emphasizes the over-abundant, nearly delusional quality of his language.
His rhetorical posturing and self-aggrandizing manner---qualities he manages to maintain even in abjection---are easy to blame for his downfall.
The bad timing of his expedition to Ireland coupled with Bolingbroke’s deft and narratively foreshortened manipulation of events are just as culpable, however.
If his \emph{hamartia} is his language, then it is not necessarily the style of that language per se that is his undoing but the irreconcilability of a ``king’s speech’’---which seeks to impose on and integrate reality---with the multiplicity of speech acts that resist even Richard’s attempts to contain by overwhelming them.
Royal speech, Shakespeare is telling us, is, if not doomed to fail, one of history’s illusory by-products.
A king who seems to speak history, that is, who speaks from history, and on history’s behalf, is engaged in fiction making, in crafting a poetry of power.
This particular kind of fiction merely has the appearance of fact, of rising above all other language to serve as language’s master discourse.
Only in such circumstances can history appear to be authoritative.
We are beguiled into believing it by the self-sanctioning authority of the ``source,’’ and also insofar as it structures our own reality.
It is for this latter reason that the play \emph{Richard II} is in its form an entirely poetic artifact, but one that makes a pretense of hiding its artificiality, like the Lacanian truth that takes the form of a fiction.
If, as Ronald Levao suggests, Richard’s error ``has been to press his manipulation of reality too far,’’ exposing his divine right as the ``subjective willfulness’’ of a petulant, indecisive improviser,\footnote{Ronald Levao. Renaissance Minds and Their Fictions: Cusanus, Sidney, Shakespeare.
Berkeley, CA: University of California Press, 1985. 314.} it is nevertheless inevitable, the play proposes, for kings to adopt this role of poet master of a poetically-constructed world.

Whatever we infer about Richard’s ``true’’ feelings and motives, what we are given on the page is the only representation of himself that is appropriate, a public posture of power over language.
The real argument is over its effectiveness, not whether or not he himself has any confidence in what he says.
And when its effectiveness is challenged, his verbosity is no less fecund in grasping after new schemes of control.
His imaginative activeness proliferates right up until the moment when his sphere of physical activity is reduced to a little world peopled only by thoughts, and even in death, as previously noted, he is full of declamations.
Unfortunately for Richard, he does confront the reality---even if he does not quite or cannot acknowledge it---that one cannot so easily make a world out of words.
His tale, after all, belongs among the \emph{de casibus} and not the \emph{res gestae}.
Rather than bending to his ineffable will, history slips from Richard’s grasp and into Bolingbroke’s, who plays a different (and ultimately just as futile) game with language.
In \emph{King Lear}, Shakespeare will return to the theme of the king as history’s fool and ultimately leave us disabused of any reliance we might have had on history’s promise of either secular or divine redemption.
``Histories’’ happen while ``history’’ itself, in its forward-moving, backward-gazing retrospectiveness, remains a pastness that is always one step ahead, in the future, a totality ever just beyond the horizon of our comprehension.
The particular inquiry, as it were, always stands apart from the unknown truth of the past.

The irony of Richard’s situation is that his history has and had already been written.
Even if his audiences were not all intimately familiar with the details of English history, Shakespeare himself dramatized that events following Henry V’s military triumph, and there are interesting literary effects of his choosing to write these plays, as it were, out of order.
There would have been no surprises in store for Shakespeare’s original audiences in terms of plot, for one thing.
Interest in the play would have focused instead on its particular interpretation of events and characterizations to the extent they were, in fact, well known from chronicle histories.
Had Shakespeare dramatized Richard merely as an inept and untenable ruler, he might have pandered to whatever prejudices about a distant past remained three centuries after his reign, but he would also have produced a far less absorbing play.
What he does do is demonstrate that our ideas about history are indistinguishable from the form that structures them: they are nothing but words.
History, that is, is the product of the same sort of ``making’’ as poetry.
Where the two actually do differ, for which \emph{Richard II} makes a compelling case, is in the degree to which this making of history can be mistaken for the real power of making evoked by poetry: the unbounded creations of the imagination do not, obviously, translate into reality even if the most powerful among us often think they can be made to do so.
No king can write his own history as he goes along.

Crucially, the play’s historical disposition comes across in the relationship between the two characters around which this formal play is formally organized.
The confrontation between Richard and Bolingbroke dramatized by this play is very much a confrontation between incompatible styles of comportment.
The verbose and grandiloquent king cannot but be contrasted to his canny and taciturn cousin.
In the interview between Bolingbroke and King Richard that is at the heart of Shakespeare’s \emph{Richard II}, Bolingbroke’s calculated taciturnity contrasts strikingly, and famously, with the fulsome display of Richard’s subtle and often convoluted rhetoric.
Richard addresses his abdication with language that is as introspective and introverted as it is ambiguous---and it is so ambiguous that the printed text cannot in many places adequately capture its polysemous quality.
Bolingbroke, on the other hand, responds to Richard with silences and perfunctory remarks.
These are ambiguous in their own right, of course, if only for their parsimony of expression.
The difference between these styles of ambiguity betrays a profound difference in the linguistic philosophies of these characters: Richard’s language is grammatically ambiguous.
He fills the air with words and ideas, clever puns and perilous homophones.
It is a positive ambiguity, an ambiguity of presence that Richard revels in, the presence in so much language of multiple, contradictory possibilities all, importantly, emanating from him---he is the king as creator, through language, of the possibility of meaning itself.
Bolingbroke, by distinction, plays off what we might call a negative ambiguity or an ambiguity of absence: by saying little and keeping what he does say so compressed and controlled, he ends up with linguistic alibis against undesirable interpretations of what he does say while at the same time leaving others---and us---to supply meanings and motives for him, just as he so deftly relegates to others the execution of his will.
We might imagine Bolingbroke deliberately positioning himself to retroactively adopt whichever of those meanings or motives he finds most advantageous in whatever his present circumstances, as if they had been the truth of the matter the entire time.
His claim, at the end of \emph{Henry IV, Part Two}, that he had no intention to ascend the throne, ``But that necessity so bow'd the state / That I and greatness were compell'd to kiss’’ we may or may not regard as disingenuous, but it is his tight-lipped bearing in \emph{Richard II}, his withholding of language from us in this play that leaves open this interpretation of events to begin with.

Bolingbroke’s practical, unforthcoming bearing is established in \emph{Richard II} at the outset.
Following his banishment, Gaunt chastises him for failing to respond to the Lord Marshal’s generous offer to accompany him to Dover: ``O, to what purpose dost thou hoard thy words, / That thou returnest no greeting to thy friends?’’ Bolingbroke’s response looks forward to Cordelia’s own reticence in \emph{King Lear}: ``I have too few to take my leave of you, / When the tongue’s office should be prodigal / To breathe the abundant dolor of the heart’’ (1.3.253--57).
There follows a rather Senecan stichomythia, recalling earlier Elizabethan drama, in which he tersely dismisses Gaunt’s obtrusive moralizing.
Where Gaunt tries to cheer him with well-wearable maxims---``there is no virtue like necessity’’ (1.3.278)---and the consolation of philosophy, the word-hoarding Bolingbroke rejects the ``bare imagination’’ (1.3.297) of pretending your situation is not as bad as it is.
In this, he contrasts quite severely with the world-creating power of thought that Richard will use to summon stones to his defense and, after that, invoke in his cell.

The differing styles of speech of Richard and Bolingbroke, as commentators have long noted, seem to befit their differing attitudes to royal power: its inviolable sanctity in the case of Richard, its unmitigated contingency in the case of Bolingbroke.
The naive monarch who narcissistically refuses to disavow his divine right challenges and is defeated by the master tactician of modern, political reality: this is one of the most common ways to read the play\footnote{For Bolingbroke as Machiavellian, see Irving Ribner. ``Bolingbroke, a True Machiavellian’’ in Richard II: Critical Essays. Ed. Jeanne T. Newlin. New York: Garland, 1984. 95--104.
The classic study of Machiavelli’s ideas as they pertain to Elizabethan drama is Edward Meyer.
Machiavelli and the Elizabethan Drama.
Weimar: Emil Felber, 1897.
It is interesting to note that the archetypal Machivel in Shakespeare’s histories is the more blatantly evil Richard III, a character who, like Iago and Edmund, not only revels in his calculated malice but communicates it directly to the audience.
The ascription of this negatively-regarded trait to Bolingbroke, who is more oriented toward politics than motiveless malignancy, is strange if our sympathies are in any way meant to align with him, and he is as tight-lipped about his motives in Richard II as if he does not have any.}---and is credible enough given its plot---but I want to investigate the possibility that there is more at stake in the tête-à-tête between Richard and Bolingbroke and in the verbal game that they play.
What their encounter illustrates is not so much, I would argue, the supremacy in the modern political dispensation of Machiavellian cunning and pragmatism over medieval absolutism and ceremoniousness but the ineluctable compromise that power, specifically political power, must always make with language.
If the desire to close the gap between language and reality is concomitant with the wielding of power itself, Shakespeare’s \emph{Richard II} suggests that the gap is never completely closable.
Rather, there is a tension between the two, a tension that is always ready to explode as soon as a language of power becomes too far removed from the reality it purports to represent and as soon as those on whom the language impinges discover they are no longer invested in maintaining the illusion that it does so.
Richard’s tragic flaw might be his too self-indulgent belief in the self-representations his kingly speech fashions, but Bolingbroke’s victory over him hardly puts an end to kingly self-representations.
The cynical, politic world which Bolingbroke is so often held to inaugurate might have ironically required such representations even more and might have required that acquiescence to them, sincere or not, be policed to an even greater degree.
As Henry IV, Bolingbroke finds himself uneasily having to occupy the same sort of role as Richard and to perform the same sort of negotiation between his royal speech and the world it attempts to structure.

The play is also an ironic statement about historical discourse.
If characters in other plays seem to know they are in plays (if they aren’t producing plays themselves), Richard seems uncomfortably aware of his role in a ``fall of princes’’ allegory, that he is trapped in a moral exemplum.
What the play relates is his struggle to step out of that role, to determine himself, through language, his own purchase on a history that contains him.
Whereas Bolingbroke often comes out ahead in comparisons of the two leaders, it is arguable that Richard better understands the nature of the kingship.
For all his equivocating and bluster, Richard enacts to perfection the schizophrenic part a king must play.
He behaves as if the world were his to command with words.
He believes (or pretends to believe) that his rule is guaranteed by the historical process that made him king to begin with, even if that historical process is itself a fiction he is enjoined to promulgate for the sake of his authority.
In this way, history bears a relation to divinity.
They both serve as slippery terms that appear to be at the same time universal objects of appeal and yet entirely contingent on human subjectivity.
In Richard’s estimation, God and history ought to be on the same side: his.
At least, he understands that he cannot reign without acting as if this is so, and the line between his belief and his pretense is vanishingly thin.

Bolingbroke demurs to play this game, thinking that shrewd maneuvers and linguistic withholding will spare him (and his successors, whose fates Shakespeare already had written) from the calamities that history teaches us to expect.
There is a reason the play ends a tragedy, with a royal body on the stage---just as \emph{Henry VI, Part One} begins with one---and not with Henry IV triumphant.
Bolingbroke’s achievement is ironically undermined, as we have seen, by Shakespeare in terms remarkably similar, if differently registered, to the victory of Richmond in \emph{Richard III}.
Chronological history might seem to look forward to the Tudors, but Shakespeare’s unique way with endings means that in literary time, the end to bloody wars promised by the future Henry VII is followed immediately by the bloody wars’s beginning.
This circularity undoes, to some extent, myths of historical progress in general or teleological Tudorism in particular.
Instead of universal history proceeding from the creation of the world to the latest victorious bloodletter, we get instead, in Shakespeare, history as a meaningless, repetitive ouroboros.
As if history were not an abstract concept (or even impalpable phenomenon) but an office under royal command, Richard and Bolingbroke both demonstrate that the impulse to mold history out of our own imaginations is as unavoidable as our inability to do so.
In its combining of linguistic fulsomeness with ritualized politics, \emph{Richard II} dramatizes this fruitless search for historical mastery.

Bolingbroke receives his first lesson in performing the role and speaking the language of a king at the moment of his ascendency.
In the abdication sequence, which we have already examined in part, Bolingbroke has Richard summoned to Westminster Hall to formally relinquish the crown.
He wishes for, or consents to, a public deposition of Richard, he says, ``that in common view / He may surrender; / so we shall proceed / Without suspicion’’ (4.1.155--57).
By acting ``in common view,’’ he expects to achieve a public sanction of his authority and also to make of himself a public figure whose actions are above ``suspicion.’’ To advance a public persona, in other words, is, for Bolingbroke, to provide cover for his private thoughts and hidden maneuvers.
Unlike Richard, who seems to tell us everything, Bolingbroke remains restrained and obscure.
Shakespeare enhances these characteristics, and perhaps tempts our interpretive scrutiny, by not writing into the play, or only alluding cryptically to, much of what Bolingbroke must be saying and must be doing to orchestrate his rise to power.
Even the loaded remark that leads to the erstwhile king’s murder is uttered offstage, as it were, dialogue reported at second hand lending itself, as Shakespeare no doubt knew, to the sort of deniability that Bolingbroke wants built-in to his royal language.
And Bolingbroke’s way of transmitting power through his language is indeed to carefully calibrate his speech, as if enough care could be taken with it that his darker purposes will remain hidden even as they are realized.
We may infer his motives, in other words, but his language makes it hard to pin them down definitively.
In his commandment to bring the fallen Richard into his presence, ``He may surrender’’ is not quite ``he must surrender’’---subtly, subjunctively I would say, maintaining the illusion that Richard has chosen to abdicate of his own free will.
He may do so if he wishes.
And he also has permission to do it in public.
``So we shall proceed’’ is so compact, subtle, and bland an expression that it could easily escape notice altogether, except that it insinuates into the royal ``we’’ the collaboration of his fellow rebels; combines the simple fact of futurity, ``shall,’’ with a sense of serious intent; and the verb, ``proceed,’’ is so neutral and non-specific that in its very attempt to slip by it almost begs to be interrogated for some more precise meaning.

Shakespeare clearly characterizes Bolingbroke as cognizant of the fine, linguistic line he is treading in his adoption of a regal bearing.
By having Richard brought in to make a spectacle of himself, Bolingbroke acknowledges that to be legitimately invested with royal power---to be perceived as legitimate---requires more than pedigree or military conquest.
The prize of his inheritance he might have wrested by force from Richard’s control, but the seizing of the kingship, if the pretense and symbolic effectiveness of political legitimacy itself are to be satisfied, demands some kind of display, some occasion of formal utterances.
We may regard Bolingbroke as a self-interested propagandist who does not, or does not yet, believe in the sanctity of kingship, as Richard seems to, who merely exploits it in order to secure his office.
But royal symbols were powerful enough at the time Shakespeare was writing that the abdication is not a scene we should take merely as the exposing of the artifices of monarchy on Shakespeare’s part or an entirely cynical maneuver on Bolingbroke’s.
Bolingbroke understands the necessity of ceremony, of being witnessed being incorporated into the representational system of the English monarchy---of being transformed in people’s minds from Bolingbroke into ``Henry, of that name the fourth.’’ But his understanding of what it means to speak as a king is only partial.
He is clever about exploiting the dark corners of language to his advantage, but he has not yet experienced the extent to which a king’s speech takes on a resonance that reverberates beyond the intentions of the person who physically occupies the throne.
Part of what Shakespeare’s play examines is the gap that opens up between the person who is king and the king-as-speaker, as if the voice of the king-as-speaker is somehow disembodied and superior to even the person of the king himself---not quite the same thing as the king’s two bodies but not entirely unrelated, either.
The success of a monarch, as \emph{Richard II} and Shakespeare’s subsequent history plays demonstrate, has much to do with how well this gap is negotiated.
In the poetically rich linguistic world of \emph{Richard II} specifically, the king really does come to be a sort of poetic maker, whether intentionally or not.

Richard, the poet-maker-king par excellence, speaks, in contrast to Bolingbroke, with full confidence in the illocutionary purchase of his every word; even after his deposition, he cannot entirely bring himself to believe the contrary, that his word never could call to arms a glorious angel ``for every man that Bullingbrook hath press’d’’ (3.2.58).
Even in prison, Richard still yearns to master his own reality with nothing left to him except his ``still-breeding thoughts’’ (5.5.8).
At the deposition, Bolingbroke speaks a language of calculated intrigue---when he speaks at all.
Richard, however, appears to mock his restraint and in so doing mock the notion that a king would not act as master of his own language and thereby master of the reality that language generates.
Bolingbroke’s terse statements only provoke him to this, and the effect of Bolingbroke’s paucity of language is heightened for us as the readers of this scene by Richard’s grandiloquence.
When Bolingbroke says, ``I thought you had been willing to resign,’’ Richard does not take the bait.
Instead, his response, ``My crown I am, but still my griefs are mine.
/ You may my glories and my state depose, / But not my griefs; still am I king of those’’ (4.1.190--93) is even more of a hedge.
``My crown I am’’ could just as well be inverted: ``I am my crown.’’ This is the paradox Richard finds himself in: the language that commands can command anything except its own dissolution.
``You may my glories and my state depose’’ turns the motive force of the abdication back on Bolingbroke: he is not resigning, Bolingbroke is deposing, and not necessarily his state, or kingdom, so much as his achievements and personal condition.
He continues to regard himself the true king despite being in Bolingbroke’s power.
And with these sleights of language proves himself language’s true master.

To some extent, Richard is teaching him what it is to possess a monopoly on linguistic authority.
Easy though it is to dismiss Richard as hopelessly mired in his own imagination, there is much in the play to indicate that he does, in fact, know exactly what he is doing and knows exactly what the performative nature of kingship is, even if he clings to it beyond all hope for himself.
Bolingbroke may never adopt the verbal panache of Richard, but he does come to learn, by necessity, the seriousness of royal speech: how it must create a world for people to inhabit and not simply stand aloof to strategize from the empyrean.
Bolingbroke is vague in his pronouncements perhaps in order to let people betray themselves.
Richard layers meaning atop meaning and ends up meaning everything at once, a king who thinks he is everything at once, a king of ``twenty thousand names.’’ He responds to Bolingbroke’s craft and guile by trying to lure him into a maze of unanswerability.
As the king, Richard naturally feels that he answers to nobody: hardly has he learned, he says earlier, ``to insinuate, flatter, bow, and bend my knee’’ (4.1.165).
And so his answers in their exchange give Bolingbroke everything he wants but also nothing.
His famous response to the question Bolingbroke, still attempting to keep a handle on the situation, finally asks, ``Are you contented to resign the crown?’’ is only the apogee of his frustrated and frustrating linguistic acrobatics: ``Ay, no, no, ay; for I must nothing be; / Therefore no no, for I resign to thee’’ (4.1.200--2).
As Leonard Barkin points out, \emph{Richard II} avoids the kinds of decisive narrative action we see in other history plays.
There are, instead of the duels and battles hinted at, various forms of dithering, equivocation, and delay.
The signature moment of the play, the deposition scene, only conforms to this pattern by cleverly subverting what ought to be an emotionally charged act: ``when the decisive moment actually arrives, it is almost comically anticlimactic.’’\footnote{Leonard Barkin. ``The Theatrical Consistency of Richard II.’’ Shakespeare Quarterly 29.1 (1978): 5--19.} Richard, who cannot wrap his head (or tongue) around the possibility is pitted against Bolingbroke, who does everything he can to avoid falling into the king’s verbal traps.
The result is so deflating to what should be pretty lofty stakes---the monarchy itself---that what Richard ends up handing over seems less a crown than a linguistic carapace that Bolingbroke, for all his inscrutable cleverness, is not necessarily prepared to fill out.

The king may seem to be jesting when he puns, but his synecdoches cut deeper when one realizes that the ultimate synecdoche in this play is the relationship of the king to the ``king.’’ Barkin is right to point out Richard’s tendency to treat the events of the play from a posture of ``aesthetic detachment.’’ This is, I would argue, the effect of Richard’s presumption that he is the master of his language and that reality is determined by his words.
And it is also the reason, theatrics aside, why the avoidance of ``decisive action’’ is built into the structure of the play.
This is a play about linguistic control, not about battling armies.
Bolingbroke’s linguistic style confronts  but also confirms Richard’s.
The volleys are words.
The action is so much in the background, it has to be made trivial, even the most significant actions---even Richard’s death, ultimately, and the newly crowned Henry IV’s reaction to it, as he becomes newly wise to his role and thus contrives a crusade to Jerusalem.
And we, of course, already know his ironic fate, too.

There is a telling metaphor at the exact center of the play.
Bereft of friends, forced to resign the throne, and at his cousin Bolingbroke’s mercy, Richard finds himself unable to bear the sight of his own face.
Dramatically asking at the moment of his deposition for a looking-glass, he tries to apprehend his divided self within it: the image of the king he believed himself to be and the fallen creature now before him.
Looking in the mirror, he gives his final public performance and parting lesson to Bolingbroke:
\begin{vq}
O flatt’ring glass, \\
Like to my followers in prosperity, \\
Thou dost beguile me! Was this face the face \\
That every day under his household roof \\
Did keep ten thousand men? Was this the face \\
That like the sun, did make beholders wink? \\
Is this the face which fac’d so many follies, \\
That was at last out-fac’d by Bullingbrook? \\
A brittle glory shineth in this face, \\
As brittle as the glory is the face, \\
\hspace{9em} [\emph{Dashes the glass against the ground}.] \\
For there it is, crack’d in an hundred shivers. \\
Mark, silent king, the moral of this sport, \\
How soon my sorrow hath destroy’d my face. (4.1.279--91)
\end{vq}
The breaking of the mirror has provided critics a rich occasion for analysis.
The mirror itself recalls the \emph{Mirror for Magistrates}---in which Richard himself appears to narrate his fate---and the entire \emph{speculum regis} genre of advice for princes.
More patently, the mirror serves as a metaphor for deceiving images.
Richard, who found himself deceived by his allies, sees now that his own image does not align with his expectations, that it too is deceitful.
To go only a step further, he also recognizes the idolatry inherent in kingship itself.
Only his insubstantial image---Kantorowicz’s ``body politic’’---is king.
His flesh and blood person, subject to the blows of sorrow, is merely the monarchy’s temporary vehicle.
Hence his subsequent play with the word ``shadow’’, as Bolingbroke, not yet himself a king, appears unmoved by Richard’s histrionics.
Noteworthy, as well, is that he uses the word ``read’’ when he receives the mirror.
Refusing Northumberland’s demand that he read aloud his crimes, he says instead that he will read the mirror, or read himself in the mirror.
What he finds there, however, is deception, because his image has not changed.
No deeper truth has emerged to inscribe his face with the lines betokening a changed state.
The mirror’s breaking can also be interpreted in numerous ways.
By smashing the image of the king, Richard publicly demonstrates the fragility, the insubstantiality, of that image and how easily it can be shattered.
He means this characteristically dramatic gesture as a lesson for Bolingbroke: ``by breaking the mirror he enacts the suddenness with which glory and power may be shattered, leaving the man, the human truth, unchanged and unrevealed.’’\footnote{Harold F. Folland. ``King Richard’s Pallid Victory.’’ Shakespeare Quarterly 24:4 (1973): 398.} Bolingbroke, who is himself quite unrevealed compared to Richard, has not yet come to realize the truth of Richard’s self-abnegating display.
But the shattered mirror designates more than the ``moral’’ that Richard intends by it.
If it did not, it would be fair to describe the play, \emph{Richard II}, as little more than a poetically overwrought rendering of traditional, Boethian philosophizing.
The image in Richard’s mirror is also the image of history.
On the surface, it looks human, but the surface is all we can see.
Much like the play itself, we can only guess at the inner life of these characters by what they say.
History, too, is a similar sort of extrapolation from limited evidence and lacking in the psychological complexity afforded by drama.

The singular mirror that Richard holds reflects back the false image of a unitary account of events.
Where he had once believed his own thoughts capable of producing his reality, the mirror affords him perspective.
Its shattering shows how that perspective is merely his own, and not one in any way enforced upon the world.
The unitary image of kingship he imagined is thus fragmented into a hundred pieces, each reflecting back its own piecemeal, fragmentary truth.
History in Shakespeare’s Richard ends up looking very much like the ``flatt’ring glass’’ that the deposed king throws at Bolingbroke's feet---``crack’d in an hundred shivers’’.
For as many shards it splinters into, there are so many ways to see it.
As Richard discovers to his sorrow, history is neither a scripted guarantee of a divinely-ordered cosmos nor a reflection of human will.
It is, rather, an image that beguiles.
Richard’s revelation here is not that he has changed but that he has not changed.
In outward appearance he is the same even if everything else about him has now become qualitatively different.
The differences that exist but cannot be seen, we are meant to conclude, are as insubstantial as the image that appears in a mirror---and yet somehow more substantial.
If his cares had furrowed wrinkles in his face or otherwise transformed his appearance, he could understand that he is different.
But his image is all he is, and since he is, after all, a creature for us of language, the image in the mirror is really the speech that is reflected back to us by the play.
There is no other, interior dimension for us to look into.
The surface of the play produces an image as the surface of history produces an image, too, and neither are more than shadows.
Richard makes a show of this for Bolingbroke’s sake, though his cousin is as yet unsympathetic.

Moreover, Richard’s punning on his image explicitly makes of it a piece of language: ``Was this face the face / That every day under his household roof / Did keep ten thousand men? / Was this the face / That like the sun, did make beholders wink?’’ He does not quite believe the reality of what he is ``seeing’’ any more than Bolingbroke believes that words themselves are more than shadows.
Richard perceives himself to have been outmaneuvered by Bolingbroke but also ``out-fac’d,’’ that is defeated in the game of linguistic manipulation at Richard excels and even continues to excel in this very exchange.
Such is his virtuosity, it is hard to believe that Richard actually believes what he is saying.
The ironic subtext is, perhaps, extended by the metaphor’s obvious allusion to Christopher Marlowe’s \emph{Doctor Faustus}.
In that play, Faustus also cannot quite believe what he is seeing.
Having asked for everything from absolute knowledge to Helen of Troy herself, he is disappointed to learn nothing he does not already know and that the mysteries he seeks are no more than word games, illusions like Helen.
There is, perhaps, not so great a division between the mage who sells his soul for empty knowledge and the usurper who sells his reputation for an empty crown.
We know Shakespeare dwelled on this figure of recognition, because he repeats it in \emph{King Lear}.
As we will see in that play, Lear has much in common with the deposed Richard.
Deposed himself in all but name, it is Cordelia who wonders at her father’s ``face.’’ Whereas Richard is beguiled that nothing has changed, Cordelia is that so much has.
The difference is that Richard’s kingly name has been taken away but Lear’s has not, and so much the worse for him; the difference is that Richard’s undoing is a solipsistic spectacle of language, because he is the most knowing character in his play, whereas Lear’s is intersubjective, because it proffers the carrot of redemption before violently snatching it away---and, again, so much the worse for him.

The closing events of \emph{Richard II} show that Henry IV will have to be as linguistically astute a king as his predecessor.
When a king’s words flow through the world, after all, they create realities of their own accord, and that despite every effort to keep them under control.
When Aumerle’s treachery is revealed to his father the Duke of York, the entire family rushes to King Henry, the Duke to turn his son over for summary judgment, the Duchess to plead for mercy.
It is the Duchess in this scene, who best articulates the new powers of speech that Bolingbroke has inherited:
\begin{vq}
And if I were thy nurse, they tongue to teach, \\
``Pardon’’ should be the first word of thy speech. \\
I never long’d to hear a word till now, \\
Say ``pardon,’’ King, let pity teach thee how. \\
The word is short, but not so short as sweet, \\
No word like ``pardon’’ for kings’ mouths so meet.’’ (5.3.113--18)
\end{vq}
When Henry finally relents, she insists he repeat the pardon twice as though exulting in the power of his royal word.
``A god on earth thou art.’’ (5.3.136) she exclaims in response to his benevolence.
Divine right is clearly alive and well for at least some people.
This scene is light-hearted enough the way it plays out, but it presages the betrayal of Northumberland---``thou ladder wherewithal / The mounting Bullingbrook ascends my throne,’’ (5.1.55--56) as Richard address him---and his co-conspirators that will plague Henry to the end of his reign, at least in Shakespeare’s version of events.

A darker instance of the king’s speech escaping the king’s control is Exton’s murder of Richard.
Exton reports to his comrades that Henry ``wishtly look’d on me / As who should say, ``I would thou wert the man / That would divorce this terror from my heart’’’’ (5.4.7--9).
Like his namesake, Henry II, this King Henry wonders aloud that he has no friend to relieve him of his ``living fear.’’ But Exton reports this speech at second hand and only imputes to the king the desire that he should do the deed.
This absence of Henry’s direct command dramatizes a gap in the historical record.
Even the dramatized Henry is sneaky enough to operate in the margins of the play, working his will from the other side of the page, as if the omission of his versified will could spare him responsibility for the death of a king.
Killing Richard, however, would be too much for Henry---too much like killing himself.
And he appears ``on record’’ only to abjure Exton’s act and promise atonement for the slander it will bring him.
The body, however, remains on the stage.
Richard remains, before and after he is killed, like an undead version of his former self to haunt future reigns.
The stain on historical memory is even sensed by Exton himself at the very moment of the murder, when he as much as admits the king’s persisting divinity as he laments the shedding of a his blood: ``As full of valure as of royal blood! / Both have I spill’d; O would the deed were good! / For now the devil that told me I did well / Says that this deed is chronicled in hell’’ (5.5.113--16).
Is this a deed chronicled in hell because from this act come all the horrible atrocities of history to come? Or is it because the play, for all its ironizing of kingly discourse, cannot quite disavow its special status entirely? The hell-chronicled deed, and with reference to the royal blood, takes on added resonance in \emph{Henry V}, where the titular Henry of that play makes a show of self-pity in a Richardian, albeit hypocritical, fashion: ``I Richard's body have interred new, / And on it have bestowed more contrite tears, / Than from it issued forced drops of blood’’ (4.1.295--97).
Of course, Henry’s tears can no more wash away Richard’s blood than Lady Macbeth can wash away Duncan’s.
The body remains on stage, even hinted at in Henry’s own language: ``I Richard’s body.’’ For a playwright with a demonstrated capacity to pun on the word ``I’’, and having done so quite famously during Richard’s abdication scene, it would be curious if this word order were merely fortuitous.
Perhaps Shakespeare is, in fact, making a point about historical memory.
It is not the doctrine of the ``king’s two bodies’’ that he is referring to but the history that every king carries with him of all the murdered kings that came before and the necessity of somehow assimilating that history so as to comfortably occupy this odd and historically untenable office.
That Henry V’s body will itself be displayed at the beginning of \emph{Henry VI, Part One} is a reminder of how futile it is.
History is as inexorable as it is, ultimately, unrepresentable.

Returning to the end of \emph{Richard II}, we can hardly know if Shakespeare wants us to think that events turn out as Bolingbroke intends them.
But the new king does start to sound a bit like the old king when he receives the news.
Speaking in couplets, playing with the nuances of language, it is perhaps at this late moment in the play when Bolingbroke finds himself becoming, like Richard, captive to his own self-representation.
He intends a voyage to the Holy Land to cleanse himself of the guilt he feels for Richard’s death.
As usual, we cannot be certain of his sincerity, though we can be sure of his need to appear sincere, and that, as a Shakespearean king, he remains beholden to the insuperable gap between the unreliable, unstable world in which he operates and the language of power that is intended to structure that world.
Bolingbroke’s manner of speech does not supersede Richard’s.
At most, it simply translates it into a different political idiom.
What the confrontation between Bolingbroke and Richard ultimately achieves is not a transhistorical meditation on the uses and abuses of power, but for us, an indication that the nature of power and the operation of historical processes were understood by Shakespeare and his contemporaries in more sophisticated ways than they---outside of a textually-inquisitive, intellectual elite---are often given credit for.
History is not always as simplistic as one side or another relating its version of events, and early modern readers---audiences, too, we may presume---who had many sides to choose from, were scrupulous and skeptical enough in the way they read that they are unlikely to have accepted any given discourse at face value.
Shakespeare’s \emph{Richard II} might be a critique of the beautiful but hollow language of a king so obsessed with his symbolic function and so impressed with his own rhetorical bravura that he has forgotten that the king is only a symbol with which a royal body can only imperfectly combine and that language, even a king’s language, however sanctified by institution and tradition, is not necessarily reality.
But such a reading does not necessarily entail taking sides with Bolingbroke, either, if take seriously Richard’s warnings about the impermanence of power and loyalty and his seeming articulation of the cyclical nature of history.

Shakespeare’s history plays are not definitive representations of events but, perhaps, attempts to realize the promise that historical representation holds out: that each time the story is told, each new form it is told in, we get closer to something useful to present circumstances or else revealing of deeper, human truths.
Each play is, in a way, another attempt to redeem history from itself, from the endless cycle of repetitive violence and its poetically-contrived justifications that it seems inevitably to fold back into.
If, as I would argue, Shakespeare’s engagement with the philosophy of history follows a trajectory from the poetic deconstruction of \emph{Richard II} to the total negation of \emph{King Lear}, we can identify some instructive parallels between the two plays---both styled both ``history’’ and ``tragedy.’’ In one, we see a historical king struggling to overcome through the poetic speech of royalty the fiction that a poetic history has made of him.
In the other, a fictional king is destroyed by the contradictions of kingly speech and ends up taking with him any faith we might have in historical progress, in a social order guaranteed by its narrative continuity.
The aged King Lear, at his most abject, claims in a moment of bitter irony to be ``every inch a king.’’ His youthful alter ago, Richard, at his, asks, ``How can you say to me I am a king?’’ If it takes as much historical imagination to create a king as to keep one in power as to destroy one, these two plays are counterparts in exhausting the possibilities.
The third term that lies between them, however, is \emph{Coriolanus}.
In that play, the idealized history of antiquity is dispersed among a multitude of voices, and the great actors of history shown to be scapegoats for the self-justifications of the present---the introverted Richard II and the abandoned King Lear have an unexpected relation to the historical, mythical, banished Caius Martius.
All seeking their worlds elsewhere, these characters together delineated the fulsomeness of Shakespeare’s historiographic art.